%Races----------------------------------------------------------------------------------------------------------------------------------
        
    \subsection{Race}
        \subsubsection{Race Condition ($RC$)}
            We define \set{RC} as the set of all pairs of events that are in a race. Two events $e$ and $d$ are in a race condition when they are shared memory events ($e,d \in \set{SM}$), having overlapping ranges, not having a $\stck{_\textit{hb}}$ relation with each other, and which are either two writes or the two events are involved in a $\stck{_\textit{rf}}$ relation with each other. This can be stated concisely as,
                \begin{align*}
                    \neg \ (\reln{e}{hb}{d})\ \wedge\ \neg \ (\reln{d}{hb}{e}) 
                    \ \wedge \ 
                     ( \
                     (\event{e,d}{W}\  \wedge\ (\Re(d) \cap_{\Re} \Re(e) \neq \phi)) 
                          \  \vee\ (d \stck{_\textit{rf}} e)\ \vee\ (e \stck{_\textit{rf}} d)
                    \ ).
                \end{align*}
                
        \subsubsection{Data Race ($DR$)}
            We define \set{DR} as the set of all pairs of events that are in a data-race. Two events are in a data race when they are already in a race condition and when the two events are not both of type \textit{sc}, or they have overlapping ranges. This is concisely stated as:  
                \begin{align*}
                    \event{e,d}{RC}  \ \wedge \ 
                    ((\neg\et{e}{sc} \ \vee \ \neg\et{d}{sc}) \ \vee \ 
                    (\Re(e) \cap_{\Re} \Re(d) \neq \Re(e) \cup_{\Re} \Re(d))) 
                \end{align*}

        \subsubsection{Data-Race-Free (DRF) Programs}
            An execution is considered data-race free if none of the above conditions for data-races occur among events. A program is data-race free if all its executions are data race free.          
            \textit{The memory model guarantees Sequential Consistency for all data-race free programs (SC-DRF).}
            