%Valid Execution Rules---------------------------------------------------------------------------------------------------------------------------------
    %MAKE SURE TO PLACE DEFINITONS ON PROGRAM, CANDIDATE, CANDIDATE EXECUTION and OBSERVABLE BEHAVIOURS        

    \subsection{Valid Execution Rules (the Axioms)}
        We now state the memory consistency rules. The rules are on \textit{Candidate Executions} which will place constraints on the possible \textit{Observable behaviors} that may result from it.
         
        %Coherent Reads   
        \subsubsection{Coherent Reads} 
        
            There are certain restrictions of what a read event cannot see in an execution based on $\stck{_\textit{hb}}$ relation with write events.
            
            Consider a read event $e$ and a write event $d$ having at least overlapping ranges:
            \begin{align*}
                \event{e}{R} \ \wedge \ 
                \event{d}{W} \ \wedge \
                (\Re(e) \cap_\Re \Re(d) \neq \phi).
            \end{align*}
            
            A read value cannot come from a write that has happened after it or if there is a write $g$ that happens between them, writing at the same memory.     \begin{gather*}
                    \reln{e}{hb}{d}\ \Rightarrow{}\ \neg \ \reln{e}{rf}{d}. \\
                    \reln{d}{hb}{e}
                    \ \wedge \ 
                    \reln{d}{hb}{g} \ \wedge \  \reln{g}{ao}{e}
                    \ \Rightarrow{} \
                    \forall x \in (\Re(d) \cap_\Re \Re(g) \cap_\Re \Re(e)), \ \neg \ \reln{e}{rbf}{(d,x)}.
                \end{gather*}
     
      \subsubsection{Tear-Free Reads} 
               If two tear free writes $d$ and $g$ and a tear free read $e$ all with equal ranges exist, then $e$ can read only from one of them
                \begin{align*}
                      \et{d}{tf}\ \wedge\ \et{g}{tf} \ \wedge \ \et{e}{tf} 
                        \ \wedge \ 
                        (\Re(d) \!=\! \Re(g) \!=\! \Re(e)) 
                        \ \Rightarrow{} \ 
                            ((\reln{e}{rf}{d}) 
                            \ \wedge \ 
                            (\neg \ \reln{e}{rf}{g})) 
                        \ \vee \  
                            ((\reln{e}{rf}{g}) 
                            \ \wedge \
                            (\neg \ \reln{e}{rf}{d})).
                \end{align*}
                    
        \subsubsection{Sequentially Consistent Atomics} 
            To specifically define how events that are sequentially consistent affects what values a read cannot see, we assume the following memory order among writes $d$ and $g$ and a read $e$ to be the premise for all the rules: 
                \begin{align*}
                    d \stck{_{mo}} g \stck{_{mo}} e.
                \end{align*}
            There are three separate rules that restrict reads-from relation between $d$ and $e$
                \begin{gather*}
                        \et{d}{sc}\ \wedge\ \et{g}{sc}\ \wedge\ \et{e}{sc} 
                        \ \wedge \ (\Re(d) \!=\! \Re(g) \!=\! \Re(e))
                        \ \Rightarrow{} \ 
                        \neg \ \reln{e}{rf}{d}.
                    \\    
                        \et{d}{sc}\ \wedge\ \et{g}{sc}  
                        \ \wedge \ (\Re(d) \!=\! \Re(g)) 
                        \ \wedge \ \reln{d}{hb}{e}
                        \ \wedge \ \reln{g}{hb}{e}
                        \ \Rightarrow{}\  
                        \neg \ \reln{e}{rf}{d}.
                    \\
                        \et{g}{sc}\ \wedge\ \et{e}{sc}  
                        \ \wedge \ (\Re(g) \!= \!\Re(e)) 
                        \ \wedge \ \reln{d}{hb}{g} 
                        \ \wedge \ \reln{d}{hb}{e}
                        \ \Rightarrow \ 
                        \neg \ \reln{e}{rf}{d}.
                \end{gather*}
  