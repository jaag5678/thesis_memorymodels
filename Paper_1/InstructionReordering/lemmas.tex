%A new command to quickly use cons function in formal descriptions
\newcommand{\cons}[2]{\textit{cons}(#1,#2)}
  
%--------------------------------------------------------------------------------------------------------------   
    \subsection{Lemmas to assist our proof}    
    In order to assist our proof, we define two lemmas based on the ordering relations. 
    
    \begin{lemma} Consider three events $e$, $d$, and $k$. \\
    
        If
            \[
                \cons{e}{d} \ \wedge \ \reln{e}{ao}{d} \ \wedge \
                (
                    (\et{d}{uo}) \ \vee \
                    (\et{d}{sc} \ \wedge \ \event{d}{W})
                )
            \]
            
        then,
            \[
                \reln{k}{hb}{d}\ \Rightarrow\ \reln{k}{hb}{e}.
            \]
    \end{lemma}
    
    %An alternative short proof 
    \begin{proof}
        We have the following to be true :
            \begin{align*}
                cons(e,d) \ \wedge \ \reln{e}{ao}{d}.
            \end{align*}
        In both cases where $d$ is unordered or a sequentially consistent write, for any event $k$
        \[
            dir(k,d)\ \Rightarrow\ cons(k,d).
        \]
        
        An event that satisfies the above with $d$ is $e$. Because $\stck{_\textit{ao}}$ is a total order, $e$ will be the only event. This would mean that for any other $k \neq e$,
        \begin{align*}
            \reln{k}{hb}{d}\ \Rightarrow\ \reln{k}{hb}{e}.
        \end{align*}
        
        Note that although there could be a direct \textit{happens-before} relation with some event $k$ from \textit{another} agent, they are only relations satisfying $dir(d,k)$.
        
    \end{proof}

%---------------------------------------------------------------------------------------------------------------    
    
%SHORTER VERSION OF PROOF WITHOUT THE ENGLISH EXPLAINATION IN THE MIDDLE. DISCUSS AND DECIDE ON WHICH FORM IS BETTER
    \begin{lemma}Consider three events $e$, $d$ and $k$. \\
    
        If
            \[
                \cons{e}{d} \ \wedge \ \reln{e}{ao}{d} \ \wedge \
                (
                    (\et{e}{uo}) \ \vee \
                    (\et{e}{sc} \ \wedge \ \event{e}{R})
                )
            \]
            
        then,
            \[
                \reln{e}{hb}{k}\ \Rightarrow\ \reln{d}{hb}{k}.
            \]
    \end{lemma}
    
    %An alternative proof for this 
    \begin{proof}
        The proof is symmetric to that of Lemma 1. 
    \end{proof}

    \emph{Note that the above lemmas are only for events $k$ which are not of type \textit{init}}
    