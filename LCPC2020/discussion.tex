\section{Discussion} 

    
    Theorem 1 and its corollary together give us a set of conditions that just need to be checked in addition while performing reordering of relaxed memory events. Having these set of conditions helps us avoid addressing the data-flow complexity due to different executions of the program using such accesses.    

    It is important to note that our approach is conservative, and one might be able to do reordering without causing new observable behaviors to occur even in cases that do not satisfy our conditions. 
    This is possible because certain happens-before relations may not be essential and hence discarding them will not result in any invalid observable behavior.
    Getting such information would require an analysis that takes into account relations that we cannot obtain using just intra-thread information, which in practice might be infeasible as the number of threads and events increase. (One such well studied analysis is May-Happen-In-Parallel, whose origins come from the work done by Naumovich et al.~\cite{NaumovichA}).
    
    It is also important to note that we focus on Candidates rather than the Program. We do not in this
    work consider the specifics of identifying all possible candidates of a given program, and we assume that whatever candidate considered is a possible one for the original program. 
    This translation from program to a set of candidates is something that would be needed in order to practically incorporate our set of conditions in practice while doing transformations.
    
