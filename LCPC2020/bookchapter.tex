
%%%%%%%%%%%%%%%%%%%%%%% file typeinst.tex %%%%%%%%%%%%%%%%%%%%%%%%%
%
% This is the LaTeX source for the instructions to authors using
% the LaTeX document class 'llncs.cls' for contributions to
% the Lecture Notes in Computer Sciences series.
% http://www.springer.com/lncs       Springer Heidelberg 2006/05/04
%
% It may be used as a template for your own input - copy it
% to a new file with a new name and use it as the basis
% for your article.
%
% NB: the document class 'llncs' has its own and detailed documentation, see
% ftp://ftp.springer.de/data/pubftp/pub/tex/latex/llncs/latex2e/llncsdoc.pdf
%
%%%%%%%%%%%%%%%%%%%%%%%%%%%%%%%%%%%%%%%%%%%%%%%%%%%%%%%%%%%%%%%%%%%


\documentclass[runningheads,a4paper]{llncs}
\usepackage{tasks}
\usepackage{amsmath}
\usepackage{amssymb}
\setcounter{tocdepth}{3}
\usepackage{graphicx}
%\usepackage{amsthm}
\usepackage{float}
\usepackage{centernot}
\usepackage[subtle]{savetrees}
\usepackage{xcolor}


%\setlength{\belowcaptionskip}{-20pt}
\setlength{\intextsep}{3pt}

\usepackage{url}

%Sorting subsubsection spacing
\makeatletter
\renewcommand\subsubsection{\@startsection{subsubsection}{3}{\z@}%
                       {-5\p@ \@plus -4\p@ \@minus -4\p@}% Formerly -18\p@ \@plus -4\p@ \@minus -4\p@
                       {-0.5em \@plus -0.22em \@minus -0.1em}%
                       {\normalfont\normalsize\bfseries\boldmath}}
\makeatother

\newcommand{\keywords}[1]{\par\addvspace\baselineskip
\noindent\keywordname\enspace\ignorespaces#1}

%SOME CRAFTED COMMANDS 

%A short form to write critics of own color
\usepackage{array}
\long\def\authornote#1{%
	\leavevmode\unskip\raisebox{-3.5pt}{\rlap{$\scriptstyle\diamond$}}%
	\marginpar{\raggedright\hbadness=10000
		\def\baselinestretch{0.8}\tiny
		\it #1\par}}
\newcommand{\ati}[1]{\authornote{AK: #1}}
\newcommand{\clark}[1]{\authornote{\textcolor{red}{\textbf{C: #1}}}}

%Short form to use stack_relative
\newcommand{\stck}{\stackrel{\longrightarrow}}

%A different version of the above 
\newcommand{\stckdet}[1]{\stackrel{{#1}}}

%We write a lot of relations between two events using the orderings, so a short form to use that
\newcommand{\reln}[3]{#1\stck{_{\textit{#2}}}#3}

\newcommand{\dblind}[1]{~}
%\newcommand{\dblind}[1]{#1}

%We also will introduce a short form to write an event belongs to some set
\newcommand{\event}[2]{#1\!\in\!#2}

%To make events and their type more close to each other
\newcommand{\typ}[1]{\textit{#1}}
\newcommand{\et}[2]{#1\!:\!\typ{#2}}

%Short form to write color text
\newcommand{\tc}[2]{\textcolor{#1}{#2}}
\newcommand{\needcite}{\textbf{\tc{red}{cite}}}

\begin{document}

\mainmatter  % start of an individual contribution

% first the title is needed
\title{Reordering Under the ECMAScript Memory Consistency Model}

% a short form should be given in case it is too long for the running head
\titlerunning{Reordering Under the ECMAScript Memory Consistency Model}

% the name(s) of the author(s) follow(s) next
%
% NB: Chinese authors should write their first names(s) in front of
% their surnames. This ensures that the names appear correctly in
% the running heads and the author index.
%
\author{{Akshay Gopalakrishnan}
{\and Clark Verbrugge}}
%
%\authorrunning{Lecture Notes in Computer Science: Authors' Instructions}
% (feature abused for this document to repeat the title also on left hand pages)

% the affiliations are given next; don't give your e-mail address
% unless you accept that it will be published
\institute{{McGill University}\\
{Montr\'{e}al, Qu\'{e}bec, Canada}\\
{\email{akshay.akshay@mail.mcgill.ca}, \email{clump@cs.mcgill.ca}}}

%
% NB: a more complex sample for affiliations and the mapping to the
% corresponding authors can be found in the file "llncs.dem"
% (search for the string "\mainmatter" where a contribution starts).
% "llncs.dem" accompanies the document class "llncs.cls".
%

%\toctitle{Lecture Notes in Computer Science}
%\tocauthor{Authors' Instructions}
\maketitle

\begin{abstract}
Relaxed memory accesses are used to gain substantial improvement in the performance of concurrent programs.
A relaxed consistency model specifically describes the semantics of such memory accesses for a particular programming language. 
Historically, such semantics are often ill defined or misunderstood, and have been shown to 
conflict with common compiler optimizations essential for the performance of programs overall.
In this paper, we give a formal description of the ECMAScript relaxed memory consistency model. 
We then analyze the impact of this model on one of the most common compiler optimizations, viz. \textit{instruction reordering}. We give a conservative proof under which such optimization is allowed for relaxed memory accesses. 
Finally, we discuss the advantage of our conservative approach and the gaps needed to be filled in order to incorporate such analysis while doing such optimizations at the program level.  
%\emph{abstract} environment.
\keywords{relaxed memory consistency, optimization, ECMAScript}
\end{abstract}



\section{Introduction}
    Instruction reordering is a common operation done by the compiler / hardware for optimization, essential to instruction scheduling of course, but also implicit in loop invariant removal, partial redundancy elimination, and other optimizations that may move instructions. 
    However, whether we can do such reordering freely given a concurrent program using relaxed memory accesses is a bit unclear. 
     
    \paragraph{Simple reordering is not straightforward under shared memory semantics}
    The main reason is that memory accesses here, do not just perform the desired operation (i.e Read / Write) but also imply certain visibility guarantees across all the other threads.  
    In our observation, we find that, the relaxed memory model of Javascript prescribe semantics for visibility using the $\stck{_{hb}}$ relations. 
    
    \paragraph{Some Examples}

        We show a couple of examples to showcase why reordering may not be that straightforward. 

        Consider the first example in Figure~\ref{reord:example1(a)} below of a Candidate and the resultant candidate after reordering two events.
        The figure on the left is the original candidate and that on the right is after reordering the two reads of $T2$.
        The observable behavior in question is written in the middle. 
        \begin{figure}[H]
            \centering
            \includegraphics[scale=0.7]{5.InstructionReordering/0.Intro/ReorderingExample1(a).pdf}
            \caption{First example for reordering with candidates of the original program and its reordered counterpart.}
            \label{reord:example1(a)} 
        \end{figure}
        
        Figure~\ref{reord:example1(b)} has two sets of relations. 
        The first justifies the outcome for the reordered candidate. 
        While the second justifies the original candidate. 
        Notice that for the second, one may have a read memory ordered before a write that it reads from. 
        This is quite counter intuitive to understand at first. 
        But strictly from the semantics of the model, this justification of the observable behavior is completely valid. 
        \begin{figure}[H]
            \centering
            \includegraphics[scale=0.7]{5.InstructionReordering/0.Intro/ReorderingExample1(b).pdf}
            \caption{The set of partial order relations justifying the observable behavior in question for both the candidates in Figure~\ref{reord:example1(a)}.} 
            \label{reord:example1(b)}
        \end{figure}

        
        Consider another example in Figure~\ref{reord:example2(a)}.
        The figure on the left is the original candidate and that on the right is after reordering the two events of $T1$.
        The observable behavior in question is written in the middle. 
        \begin{figure}[H]
            \centering
            \includegraphics[scale=0.7]{5.InstructionReordering/0.Intro/ReorderingExample2(a).pdf}
            \caption{Second example for reordering with candidates of the original program and its reordered counterpart.} 
            \label{reord:example2(a)}
        \end{figure}

        
        Figure~\ref{reord:example2(b)} has two sets of relations. 
        The first justifies that such an outcome is not possible for the original program candidate due to Axiom \ref{CoRe}. 
        While the second justifies that this outcome is possible for the reordered program.
        Note that we cannnot infer in the reordered candidate the set of relations for any candidate execution to have $\reln{a=x;_{uo}}{hb}{x=1;_{uo}}$. 
        \begin{figure}[H]
            \centering
            \includegraphics[scale=0.7]{5.InstructionReordering/0.Intro/ReorderingExample2(b).pdf}
            \caption{The set of partial order relations justifying the observable behavior in question for both the candidates in Figure~\ref{reord:example2(a)}.} 
            \label{reord:example2(b)}
        \end{figure}

        The above two examples show that we have to be careful while reordering two events in the same thread. 
        By example case analysis, for each observable behavior, one must check all possible candidate executions and assert whether such an observable is possible or not. 
        This method of checking validity of reordering will scale exponentially as the program size increases. 
        It is often also the case that the compiler may not have information on which exact events would be executed in other threads to assert such reordering is valid or not. 

    
    
    
    
    

\section{Related Work}
   Sequential Consistency, which was first formulated by Lamport et al.~\cite{Lamport79}, gives programmers a very intuitive way to reason about their programs running in a multiprocessor environment.
   However, in the practical sense, Sequential Consistency is too ``strict,'' in the sense that it may impede possible performance benefits of using low
   level optimization features, such as instruction reordering, or read/write buffers provided by the hardware.
   A tutorial by Adve et al.~\cite{AdveG}, summarizes the most common hardware features for relaxed memory that are now available in most hardware. What this tutorial also exposed is the difficulty in formalizing such features in a way that we can reason about our programs sanely without getting caught up in the complexity of multiple executions of our programs. 
   Unsurprisingly, relaxed memory model specifications for different hardware / high level programming languages are still sometimes written in informal prose format, which lead to a number of problems in implementation~\cite{Sewell}. 
   
   Sarkar et al.~\cite{SarkarS} showed that the original x86-CC memory model was fairly informal, which they then formalized in their work. This also exposed inconsistencies between the specification and the implementation in hardware. This was shown in their subsequent work done by Owens et al.~\cite{OwensS}, wherein they proposed a new memory model x86-TSO as a remedy. 
   Manson et al.~\cite{JeremyM}, showed that the initial specifications of the Java memory model were quite informal and ill defined, and offered a more precise formalization. Recent works such as that done by Bender et al.~\cite{BenderJ}, also shows us that the recent updates to the java Memory model is still relatively unclear, which they again formalize. Similarly, 
   Batty et al.~\cite{BattyM}, clarified the specification of the C11 memory model. 
   
   Apart from the problems of ill defined / informal specifications, these models also have an impact on the safety of program transformations which were considered safe to do in a sequential program. \u{S}ev\u{c}\'{i}k et al.~\cite{SevcikJ} showed that standard compiler optimizations were rendered invalid under the respective memory model of Java. Vafeiadis et al.~\cite{VafeiadisV} showed that common compiler optimizations under C11 memory model are also invalid, followed by proposing some changes to allow them. 
   
   With respect to instruction reordering in shared memory programs, \u{S}ev\u{c}\'{i}k et al.~\cite{Sevcik2} recently gave a proof design on how to show such optimizations are valid. However, this approach relies on the idea of reconstructing the original execution of a program given the optimized one, while also showing the well known SC-DRF guarantee holds---programs that are \textit{data race free} (DRF) must exhibit SC semantics. Our approach is in fact the other way round; we show that the optimized program does not introduce new behaviours, by explicitly using the consistency rules to show that relevant ordering relations are preserved.
   
   ECMAScript has also had some attention in this respect. Watt et al~\cite{WattC} uncovered and fixed a deficiency in the previous version of the model, repairing the model to guarantee SC-DRF. 
   Our analysis is based on this corrected model which is incorporated in the ECMAScript draft specification. As far as our knowledge goes, no analysis has been done on this model to identify its implications on standard compiler optimizations. 
    
\section{The ECMAScript Memory Consistency Model}
    
    We give a relatively more formal and concise axiomatic description of Section 28 of the ECMAScript standard. The version we are referring to is the current working draft~\cite{ECMA}. It is important to note that this working draft has not changed the memory model specifics since the time we started our work on this.  
    
    %AGENTS----------------------------------------------------------------------------------------------------------------------------------------  
\section{Agents, Events and their Types}

    \subsection{Agents}
        A concurrent program involves different threads/processes running concurrently. 
        Agents could be thought analogous to different threads/processes. 
        
        \critic{red}{Agents actually have more meaning than what we refer to here. However, in terms of reasoning just with memory consistency, we are safely abstracting them to just represent threads/processes.}
        
        \critic{blue}{Technically, one may not map them directly to individual threads as from an implementation standpoint, a single thread can be allowed to execute multiple concurrent processes. As a way of separating implementation from the specifications, we refer to them as Agents}

        %Agent Clusters
        \paragraph{Agent Cluster ($AC$)}
        Collection of agents running concurrently communicating with each other (directly/ indirectly) form an agent cluster.  There can be multiple agent clusters. However, an agent can only belong to one agent cluster.
        
        %PErhaps give an example here later

        \critic{blue}{Note that for the purpose of reasoning with memory model, we stick to assuming that just one agent cluster exists. So we will refrain from defining a function mapping an agent to its respective agent cluster. We also assume that agents in the cluster communicate only through one common shared memory segment.}
        
        %Agent Event Set
        \paragraph{Agent Event List $(ael)$}
        Every agent is mapped to a list of events appended to it during evaluation. We define $ael$ is a mapping of each agent to a list of events.
        
            \[ael(a) = [e_1, e_2, ... e_k ] \]
        
        \critic{blue}{The standard refers this to be an Event List, but we find it a bit misleading as it does not signify a list for each agent. Hence we name it as Agent Event List}
        
        %Ask whether this is actually required as a notation down the line
        \paragraph{}
            When referring to events in an agent cluster, we use the following notation for an event:
        
            \[ e^i_j \ \longrightarrow \  j^{th}\ \textit{instruction}\ \textit{of}\ i^{th} \textit{Agent} \]  
            
        \critic{blue}{We will sometimes forgo the subscript or superscript wherever it may not play a role in understanding a relation or definition.} 
        
        

        

            


    
    %Events------------------------------------------------------------------
\subsection{Events}
        
The memory model is described mainly using a set of events and some ordering relations on them. An evaluation of an operation results in a set of events that are evaluated. An event is either an operation that involves (shared) memory access or that constrains the order of execution of multiple events. The latter are called \textit{Synchronize Events}

\critic{blue}{Synchronizing events are analogous to $lock$ and $unlock$ events that allow exclusive access to critical sections of memory. However, this is not specified in the standard as part of the memory model.}

%Event_sets----------------------------------------------------------------------------------------------------------------------------------
    
    %Useful command syntax
    \newcommand{\rmw}{\textit{rmw}\,}
    \newcommand{\set}[1]{\textbf{\textit{#1}}}

    \subsection{Event Set}
    Given an agent cluster, an \textit{event set} is a collection of all events from the agent event lists. This set is composed of mainly two distinct subsets as follows: 
       
        %Shared Memory Events
        \subsubsection{Shared Memory (\set{SM}) Events} This set is composed of two sets of events: 
            
            \critic{purple}{Use a better listing to enumerate both items below in the same line}
            \begin{enumerate}
                \item Write events (\set{W})
                \item Read events (\set{R}) 
            \end{enumerate}
            Events that belong to both Write and Read events are called Read-Modify-Write. 
        %Synchronize events 
        \subsubsection{Synchronize (\set{S}) Events} These events only restrict the ordering of execution of events by agents. They are of two sets, which are mutually exclusive:
            \begin{enumerate}
                \item Lock events (\set{L})
                \item Unlock events (\set{U}) 
            \end{enumerate}
            
        \critic{blue}{The features of $Lock$ and $Unlock$ events is actually not something given to the programmer to use in Javascript. They are used to implement the feature $wait$ and  $notify$ that the programmer can use which adhere to the semantics of $futexes$ in Linux. Hence, in the original standard of the model, the distinction between lock and unlock is not made, and it is simply stated as Synchronize Event}
 
    \critic{blue}{There is an additional set of events called Host Specific Events, but for our purpose, it is not of any major concern.}  
        
    %Range of events
        \paragraph{Range ($\Re$)}
            Each of the \textit{shared memory events} are associated with a contiguous range of memory on which it operates. Range is a function that maps a shared memory event to the range it operates on. This we represent as a starting index $i$ and a size $s$. So we could represent the range of a write event $w$ as 
                    
                    \[\Re(w) = (i, s) \]
        
            \critic{red}{The range as per the ECMAScript standard denotes only the set of contiguous byte indices. The starting byte index is kept separate. We find this to be unnecessary. Hence we define range to have starting index and size.}
           
            Two Ranges can be \textit{disjoint}, \textit{overlapping} or \textit{equal}. We use the two operators below to define these three possibilities between ranges of events $e$ and $d$ :
            
            \begin{enumerate}
                \item Intersection $(\cap{_\Re})$ - Set of byte indices common to both ranges.
                \item Union $(\cup_\Re)$ - A unique set of byte indices that exist in both the ranges.  
            \end{enumerate}
            
            \begin{enumerate}
                \item Disjoint $\Re(e) \cap_\Re \Re(d) = \phi$ 
                \item Overlapping $(\Re(e)\cap_\Re \Re(d) \neq \phi) \wedge (\Re(e) \cap_\Re  \Re(d) \neq \Re(e) \cup_\Re \Re(d))$ - 
                \item Equal $\Re(e) \cap_\Re  \Re(d) = \Re(e) \cup_\Re \Re(d)$ - In simple terms, we define equality as $\Re(e) = \Re(d)$
            \end{enumerate}
            
            \critic{blue}{Note that two ranges being overlapping is different from them being equal. This distinction is used to define certain things ahead in the model.}
            
         \paragraph{Value($V$)}  
           It is a function that maps a byte address given to the value that is stored in that address.For example, the byte address $k \text{ has the value } x_k$ will be depicted as:
                
                \[V(k) = x_k\]
            
            \critic{red}{We introduce the value function to just map memory to values stored there. Note that we also assume only integer values for the sake of reasoning with memory models.}
            
            Using the above constructs, we represent the three subset of shared memory events with their ranges in the following way:
            
            Consider a chunk of memory {k,k+1...k+10} wherein the values stored are:
            
                \[\forall i \in [0,10], V(k+i) = x_{k+i}\]
                
            \begin{itemize}
                \item $w$ with range $(k,11)$ modifying memory to ${x'_{k}}...{x'_{k+10}}$ will be as : 
                
                        \[{W^i_j}[k...(k+10)] = \{x'_{k}, x'_{k+1}...x'_{k + 10}\}\]
                
                \item $r$ will be represented the same as write with a distinction in semantics that the right hand side is what is read from the range of memory 
                
                        \[{R^i_j}[k...(k+10)] = \{x_{k}, x_{k+1}...x_{k + 10}\}\]
                
                \item $\rmw$ will be mapped to two tuples, the left one indicating the values read and the right one indicating the values written to the same memory. 
                
                        \[{RMW^i_j}[k...(k+10)] = \{(x_{k}, x_{k+1}...x_{k + 10}), (x{'}_{k}, x{'}_{k+1}...x{'}_{k + 10}) \}\]
                
            \end{itemize}
            
            \critic{blue}{Note that some examples will also be like $R[0..4] = 10$, where 10 symbolizes the value stored in 32 bits of memory, which is ideally the form \{0, 0, 0, 32\}. This is because, we are taking decimal equivalent of a 32 bit binary number. It is important to note this fact.}


%Types of Events Based on Order--------------------------------------------------------------------------------------------------------------------
    
    \subsection{Types of events based on Order} 
        Order signifies the sequence in which event actions are visible to different agents as well as the order in which they are executed by the agents themselves. In our context, there are mainly three types for each shared memory event that tells us the kind of ordering that it respects. 
        
        \begin{enumerate}
            \item \textbf{Sequentially Consistent ($sc$)} - Events of this type are $atomic$ in nature. The meaning of sequentially consistent implies that there is a strict global total ordering of such events which is agreed upon by all concurrent processes sharing the same memory. 
            
            \item \textbf{Unordered ($uo$)} - Events of this type are considered non-atomic and can occur in different orders for each concurrent process, meaning there is no fixed global order respected by agents for such events. 
            
            \item \textbf{Initialize ($init$)} - Events of this type are used to initialize the values in memory before events in an agent cluster begin to execute concurrently. Additionally, only write events can be of this type and there is only one init event for each byte address in shared memory. 
        \end{enumerate}
        
        \paragraph{}
        We represent the type of events in the following format - $event : type$ 
        
        \critic{red}{The word \textit{atomic} is actually misleading. It does not imply the events are evaluated using just one instruction. For example, a 64-bit sequentially consistent write on a 32-bit system has to be done with two subsequent memory actions. But its intermediate state of write must not be seen by any other agent. In an abstract sense, this event must appear '\textit{atomic}'.The \textit{atomic} here also refers to implications of whether an event's consequence is visible to all other agents in the same global total order or not. The compiler must ensure that for specific hardware, such guarantees are satisfied.}
        
        \critic{red}{The notion of sequentially consistent has the same semantics of what C++ has for such events. Note that this semantics is not mentioned here explicitly, but by talking with fellow researchers working on the same domain, as well as with careful observation, it has come to our understanding that this is assumed to be true. We will make sure that the semantics is defined here properly as per what is there of C++.}
      
        \critic{red}{We are not sure if $init$ is a type of write that has a range as the range of shared memory involved in the agent cluster or is it individual writes for each byte address. This is not mentioned in the $standard$. We assume them to be for individual byte addresses, as we will see shortly in the rule for $\stck{_{hb}}$ ordering why we consider this assumption to be the best one.  But note that it may not be the case always. As an example, consider a 32-bit hardware not having instructions to support 16 or 8 bit memory actions. In that case we do not know what the $init$ event will sum up to. It would rather be better for it to be just one $init$ write event that ranges through the entire shared memory, and modify the $\stck{_{hb}}$ relation accordingly.}
      
%Tearing factor of events---------------------------------------------------------------------------------------------------------------------------

    \subsection{Tearing (Or not)}
        Additionally, each shared-memory event is also associated with whether they are tear-free operations or not.
        
        \paragraph{Tearing}
            Operations that tear are not aligned accesses with respect to the hardware and can be serviced using two or more memory fetches. 
            
        \paragraph{Tear-Free}
            Operations that are tear-free are aligned with respect to the hardware and should appear to 
            be serviced in one memory fetch. (this might not be possible always, but we are concerned with whether it can appear to be tear free)
     
        \critic{red}{There is a very confusing definition of \textit{tear-free ness} given by ECMAScript. These definitions are part of how the tear factor affects the behavior of programs in a concurrent setting. This is also defined as a set of axioms further below. We make this distinction to avoid confusion : 
        \begin{enumerate}
            \item For every Read event, tear-free-ness questions whether this event is allowed to read from multiple write events on equal range as this event
            \item For every Write event, tear-free-ness questions whether this event is allowed to be read by multiple reads on equal range as this event. 
        \end{enumerate}
        }
        
        For most of our analysis, unless otherwise stated we will assume all events to be tear-free
                       
    
    %Relation among events----------------------------------------------------------------------------------------------------------------------------
    \section{Relation among events}
        There are three basic relations that assist us in reasoning about events and their interaction with memory.
        
        %Read bytes from relation 
        \paragraph{Read-Bytes-From $(\stck{_{rbf}})$}
        
        This relation maps every read event to a list of tuples consisting of write event and their corresponding byte index that is read. For instance, consider a read event $r[i...(i+3)]$ and corresponding write events $w_1[i...(i+3)]$, $w_2[i...(i+4)]$. One possible $\stck{_{rbf}}$ relation would be: \[r \stck{_{rbf}} \{(w_1, i), (w_2, i+1), (w_2, i+2)\}\]
        
        \critic{blue}{We will represent individual rbf relations with read events in our examples. So for the above example, the three rbf pairs are: \[r \stck{_{rbf}} (w_1, i),\ r \stck{_{rbf}} (w_2, i+1),\ r \stck{_{rbf}} (w_2, i+2)\]}
        
        %Reads from relation
        \paragraph{Reads-From $(\stck{_{rf}})$}
        
        This relation, is similar to the above relation, except that the byte index details are not involved in the composite list. So for the above example, the rf relation would be :  \[r \stck{_{rf}} \{w_1,w_2\}\]
        
        \critic{blue}{Similar to rbf, we also represent pair-wise relation in rf : \[r \stck{_{rf}} w_1,\ r \stck{_{rf}} w_2\]}
        
        %Agent sync with relation
        \paragraph{Agent-Synchronizes With (\set{ASW})}
        
        A list for each agent that consist of ordered tuples of synchronize events. These tuples specify ordering constraints among agents at different points of execution. So such a list for an agent $k$ would be represented like:  
        \[ASW_k = \{ \langle s_1, s_2 \rangle, \langle s_3, s_4 \rangle ...\}\]
        
        A property that must hold for each of these lists is: 
        \begin{itemize} 
            \item For every pair in the list, the second event belongs to the parent agent and the first belongs to another agent it synchronized with.
                \[  
                    \forall{i,j>0},\ 
                    \langle s^i, s^j\rangle \in ASW_j 
                    \Rightarrow{} 
                    i \neq j\ \ \wedge \
                    s^j \in ael(k)                        
                \]
        \end{itemize}
        
        \critic{blue}{The analogy is similar to the property that every unlock must be paired with a subsequent lock, which enforces the condition that a lock can be acquired only when it has been released.}
         
    %Ordering Relation among Events----------------------------------------------------------------------------------------------------------------------       
        \subsection{Ordering Relations among Events}
        
        %Agent Order
        \subsubsection{Agent Order ($\stck{_\textit{ao}}$)}
            A total order among events belonging to the same agent event list. It is analogous to intra-thread ordering. For example, if two events $e$ and $d$ belong to the same agent event list , then either $\reln{e}{\textit{ao}}{d}$ or $\reln{d}{\textit{ao}}{e}$. 
            
           % \critic{blue}{Note that the relations are only with respect to events belonging to the same agent. A collection of such relations together form the agent order. This is analogous and meant to be equivalent to what we call as intra-thread sequential order. It is the same as what \textbf{sequenced-before} is defined to be in C++}
        
        %Synchronize With Order
        \subsubsection{Synchronize-With Order ($\stck{_\textit{sw}} $)}
           Represents the synchronizations among different agents through relations between their events. It is a composition of two sets as below: 
                \begin{gather*}
                            \forall{i, j > 0}, \ \langle s_i, s_j \rangle \in ASW\ \Rightarrow{}\ s_i \stck{_\textit{sw}} s_j 
                            \\
                            (\reln{e}{rf}{d}) \ \wedge \ \et{e}{sc} \ \wedge \ \et{d}{sc} \ \wedge \ (\Re(e)\!=\!\Re(d)) \ \Rightarrow{} \ (d \stck{_\textit{sw}} e)
                \end{gather*}
  
       %Happens Before order 
        \subsubsection{Happens Before Order ($\stck{_\textit{hb}}$)}
            A transitive order on events, composed of the following:
                \begin{gather*}
                    \reln{e}{ao}{d} \ \Rightarrow{} \ \reln{e}{hb}{d}
                    \\
                    \reln{e}{sw}{d} \ \Rightarrow{} \ \reln{e}{hb}{d}
                    \\
                    \forall e, d \in SM,\ 
                    \et{e}{init} \ \wedge \ 
                    (\Re(e) \cap_\Re \Re(d) \neq \phi)
                    \ \Rightarrow{} \ 
                    \reln{e}{hb}{d}
                \end{gather*}
        %Memory Order
        
        \subsubsection{Memory Order ($\stck{_\textit{mo}}$)}
            This is a total order on all events which respects happens-before
                \begin{align*}
                    \reln{e}{hb}{d} \ \Rightarrow{} \ \reln{e}{mo}{d}
                \end{align*}
    
    
    \subsection{Some Preliminary Definitions}
        
        Before we go into the consistency rules. we define certain preliminary definitions that create a separation based on a program, the axiomatic events and the various ordering relations defined above. This will help us understand where the consistency rules actually apply. 
        
        \begin{definition}{Program.} 
            A \emph{program} is the source code without abstraction to a set of events and ordering relations. In our context, it is the original ECMAScript program. 
        \end{definition}
        
        %What is one run of a program to us?
        \begin{definition}{Candidate.}
            This is a collection of abstracted sest of shared memory events of a program involved in one possible execution, with the added $\stck{_\textit{ao}}$ relations. We can think of this as each thread having a set of shared memory events to run in a given intra-thread ordering.
        \end{definition}
        
        \begin{definition}{Candidate Execution.}
            A Candidate with the addition of $\stck{_\textit{sw}}$, $\stck{_\textit{hb}}$ and $\stck{_\textit{mo}}$ relations. This can be viewed as the witness/justification of an actual execution of a Program. Note that there can be many Candidate Executions for a given Candidate.
        \end{definition}
        
        %What values are read when the program is run
        \begin{definition}{Observable Behavior.}
        The set of pairwise $\stck{_\textit{rf}}$ and $\stck{_\textit{rbf}}$ relations that result in one execution of the program. Think of this as our outcome of a program execution.
        \end{definition}
    %-----------------------------------------------------------------------------------------------------------------------------------------
        %\emph{The memory consistency rules restrict the possible Observable Behaviors by specifying constraints on $\stck{_{rf}}$ relations based on a  Candidate Execution. For our purpose and flow in which we successively add relations to set of events, this would also include the implication on $\stck{_{rf}}$ relation while having a $\stck{_{sw}}$ relation among two events.}     
    
    %Valid Execution Rules---------------------------------------------------------------------------------------------------------------------------------
    %MAKE SURE TO PLACE DEFINITONS ON PROGRAM, CANDIDATE, CANDIDATE EXECUTION and OBSERVABLE BEHAVIOURS        

    \subsection{Valid Execution Rules (the Axioms)}
        We now state the memory consistency rules. The rules are on \textit{Candidate Executions} which will place constraints on the possible \textit{Observable behaviors} that may result from it.
         
        %Coherent Reads   
        \subsubsection{Coherent Reads} 
        
            There are certain restrictions of what a read event cannot see in an execution based on $\stck{_\textit{hb}}$ relation with write events.
            
            Consider a read event $e$ and a write event $d$ having at least overlapping ranges:
            \begin{align*}
                \event{e}{R} \ \wedge \ 
                \event{d}{W} \ \wedge \
                (\Re(e) \cap_\Re \Re(d) \neq \phi).
            \end{align*}
            
            A read ($e$) value cannot come from a write ($d$) that has happened after it or if there is a write ($g$) that happens between them, writing to the same memory:     
                \begin{gather*}
                    \reln{e}{hb}{d}\ \Rightarrow{}\ \neg \ \reln{e}{rf}{d}. \\
                    \reln{d}{hb}{e}
                    \ \wedge \ 
                    \reln{d}{hb}{g} \ \wedge \  \reln{g}{hb}{e}
                    \ \Rightarrow{} \
                    \forall x \in (\Re(d) \cap_\Re \Re(g) \cap_\Re \Re(e)), \ \neg \ \reln{e}{rbf}{(d,x)}.
                \end{gather*}
     
      \subsubsection{Tear-Free Reads} 
               If two tear-free writes ($d$ and $g$) and a tear-free read ($e$) all with equal ranges exist, then $e$ can read only from one of them
                \begin{align*}
                      \et{d}{tf}\ \wedge\ \et{g}{tf} \ \wedge \ \et{e}{tf} 
                        \ \wedge \ 
                        (\Re(d) \!=\! \Re(g) \!=\! \Re(e)) 
                        \ \Rightarrow{} \ 
                            ((\reln{e}{rf}{d}) 
                            \ \wedge \ 
                            (\neg \ \reln{e}{rf}{g})) 
                        \ \vee \  
                            ((\reln{e}{rf}{g}) 
                            \ \wedge \
                            (\neg \ \reln{e}{rf}{d})).
                \end{align*}
                    
        \subsubsection{Sequentially Consistent Atomics} 
            To specifically define how events that are sequentially consistent affects what values a read cannot see, we assume the following memory order among writes $d$ and $g$ and a read $e$ to be the premise for all the rules: 
                \begin{align*}
                    d \stck{_{mo}} g \stck{_{mo}} e.
                \end{align*}
            There are three separate cases that restrict $e$ to read from $d$, which are as below:
            \begin{itemize}
                \item If all events are sequentially consistent with equal ranges.
                \item If both $g$ and $d$ are sequentially consistent with equal ranges and they happen before $e$.
                \item If both $e$ and $g$ are sequentially consistent with equal ranges and $d$ happens before them. 
            \end{itemize}
            The above cases can be summarized concisely by the rules below:
                \begin{gather*}
                        \et{d}{sc}\ \wedge\ \et{g}{sc}\ \wedge\ \et{e}{sc} 
                        \ \wedge \ (\Re(d) \!=\! \Re(g) \!=\! \Re(e))
                        \ \Rightarrow{} \ 
                        \neg \ \reln{e}{rf}{d}.
                    \\    
                        \et{d}{sc}\ \wedge\ \et{g}{sc}  
                        \ \wedge \ (\Re(d) \!=\! \Re(g)) 
                        \ \wedge \ \reln{d}{hb}{e}
                        \ \wedge \ \reln{g}{hb}{e}
                        \ \Rightarrow{}\  
                        \neg \ \reln{e}{rf}{d}.
                    \\
                        \et{g}{sc}\ \wedge\ \et{e}{sc}  
                        \ \wedge \ (\Re(g) \!= \!\Re(e)) 
                        \ \wedge \ \reln{d}{hb}{g} 
                        \ \wedge \ \reln{d}{hb}{e}
                        \ \Rightarrow \ 
                        \neg \ \reln{e}{rf}{d}.
                \end{gather*}
  
    
    %Races----------------------------------------------------------------------------------------------------------------------------------
        
    \section{Race}
        
        \paragraph{Race Condition $RC$} 
            We assume race condition to be a set of pairs of events that are in a race.
            
            Two events $e$ and $d$ are in a race condition when they are shared memory events 
            
             \[(e \in \set{SM}) \wedge (d \in \set{SM})\]
            
            having at least overlapping ranges, which are either two writes or the two events are involved in a $\stck{_{rf}}$ relation with each other. 
            
               
            \[ 
                 (e,d \in (\set{W} \cup \set{RMW})  \wedge (\Re(d) \cap_{\Re} \Re(e) \neq \phi)) 
                            \vee ((d \stck{_{rf}} e) \vee (e \stck{_{rf}} d)) \Rightarrow (e,d) \in RC 
            \]
            
             \critic{blue}{Though we say it as write events, they also encompass read-modify-write events, as specified by the axiom above.}
            
        \critic{red}{It is interesting to note that the standard specifies only those read-write events that do have a $\stck{_{rf}}$ relation among them to be in a race condition, while the relation signifies an order that has resolved itself because the $\stck{_{rf}}$ relation is there. According to me, their intention was to say that if there \textit{could} exist a $\stck{_{rf}}$ relation among two events, then they would definitely be in a race. This clarification needs to be incorporated in the axiom that defines what is a race condition.}
        
       
        
        \paragraph{Data Race $DR$} 
            Two events are in a data race when they are already in a race condition and when the two events are not of type $sc$ together or they have overlapping ranges: 
            
            \[(e,d) \in RC  
                \wedge ((\neg (e:sc) \vee \neg(d:sc)) 
                \vee (\Re(e) \cap_{\Re} \Re(d) \neq \Re(e) \cup_{\Re} \Re(d)) ) 
                \Rightarrow (e,d) \in DR
                \]
        \critic{red}{The definition for data race also implies that sequentially consistent events with overlapping ranges are considered to be in a data race. This is counter-intuitive in the sense that if all agents observe the same order in which these events happen, then there is no question of it being in a data-race as such, there is a unanimous agreement among all agents about the order in which they take place during the execution.}
        
        
    \paragraph{Data-Race-Free (DRF) Programs}
        An execution is considered data-race free if none of the above conditions for data-races occur among events. A program is data-race free if all its executions are data race free. 
        \newline\newline
        \textit{\textbf{The memory model guarantees sequential consistency for all data-race free programs.}}
        
    
    %Consistent Executions-------------------------------------------------------------------------------------------------------------------

   %Consistent Executions-------------------------------------------------------------------------------------------------------------------

   \section{Consistent Executions (Valid Observables)}
        A valid observable behaviour is when:
        \begin{enumerate}
           \item No $\stck{_\textit{rf}}$ relation violates the above memory consistency rules.
           \item $\stck{_\textit{hb}}$ is a strict partial order.
        \end{enumerate} 

        \textit{The memory model guarantees that every program must have at least one valid observable behaviour.}

        \critic{blue}{There is also some conditions on host-specific events (which we mentioned is not of our main concern) and what is called a chosen read, which is nothing but the reads that the underlying hardware memory model allows. Since we are not concerned with the memory models of different hardware, this restriction on reads is not of our concern.}
    
    
    % A rough draft of the proof of the simple instruction reordering, given the memory model



\section{Instruction Reordering}
    Instruction reordering is a common operation in compiler optimization, essential to instruction scheduling of course, but also implicit in loop invariant removal, partial redundancy elimination, and other optimizations that may move instructions. 
    However, whether we can do such reordering freely given a concurrent program using relaxed memory accesses is a bit unclear. 
     
    
    \paragraph{Simple reordering is not straightforward under shared memory semantics}
    The main reason is that memory accesses here, do not just perform the desired operation (i.e Read / Write) but also imply certain visibility guarantees across all the other threads.  
    In our observation, we find that, the relaxed memory model of Javascript prescribe semantics for visibility using the $\stck{_{hb}}$ relations. 
    
    \critic{purple}{Show an example or multiple examples here that enforces visibility due to having sequentially consistent events involved in a Candidate Execution.}
    
    \paragraph{What can be done?}
    An example-based analysis exposes to us the problems that might exist when we perform such reordering of events. 
    However, such an analysis, though would work for small programs to identify the possible conditions under which reordering can be done, become infeasible as the programs scale in length and complexity. 
    This is because of the exponential increase in possible executions as the number of threads and program size in general increase. 
    Hence,  generalizations by using a small sample size is not something we can afford especially when we want to ensure these program trasnformations are done by the compiler in contrast to being done manually.
    
    \paragraph{Our approach}
    Our solution to this is to construct a proof on Candidate Executions of the original program and the transformed one which exposes the possible observable behaviors it can have.   
    The crux of the proof is to guarantee that reordering does not bring any new $\stck{_{rf}}$ (reads-from) relations that did not exist in any Observable Behavior of the original Candidate Execution. 
    It is important to note however, that a proof in this sense would be generalized to any Candidate and is thus conservative.
    So, it might be the case that for specific programs, reordering can be valid, however, in a general sense may not be valid for others. 

    \paragraph{Assumption}
    We make the following assumptions for every program we consider :
    \begin{enumerate}
        \item All events are tear-free
        \item No synchronize events exist
        \item No Read-Modify-Write events exist
        \item All executions of the candidate before reordering have happens-before as a strict partial order
    \end{enumerate}
    
    We first consider when consecutive events in the same agent can be reordered, followed by non-consecutive cases. The crux of the proof is to guarantee that reordering does not bring any new reads-from relations that did not result due to any execution of the original program. 
    
    %GIVE TWO EXAMPLES TO SHOW THIS. POSSIBLY USE THE EXAMPLE ABOVE AND EXPLAIN

    
    \subsection{Some Preliminary Definitions}
        
        Before we go into the consistency rules. we define certain preliminary definitions that create a separation based on a program, the axiomatic events and the various ordering relations defined above. This will help us understand where the consistency rules actually apply. 
        
        \begin{definition}{Program.} 
            A \emph{program} is the source code without abstraction to a set of events and ordering relations. In our context, it is the original ECMAScript program. 
        \end{definition}
        
        %What is one run of a program to us?
        \begin{definition}{Candidate.}
            This is a collection of abstracted sest of shared memory events of a program involved in one possible execution, with the added $\stck{_\textit{ao}}$ relations. We can think of this as each thread having a set of shared memory events to run in a given intra-thread ordering.
        \end{definition}
        
        \begin{definition}{Candidate Execution.}
            A Candidate with the addition of $\stck{_\textit{sw}}$, $\stck{_\textit{hb}}$ and $\stck{_\textit{mo}}$ relations. This can be viewed as the witness/justification of an actual execution of a Program. Note that there can be many Candidate Executions for a given Candidate.
        \end{definition}
        
        %What values are read when the program is run
        \begin{definition}{Observable Behavior.}
        The set of pairwise $\stck{_\textit{rf}}$ and $\stck{_\textit{rbf}}$ relations that result in one execution of the program. Think of this as our outcome of a program execution.
        \end{definition}
    %-----------------------------------------------------------------------------------------------------------------------------------------
        %\emph{The memory consistency rules restrict the possible Observable Behaviors by specifying constraints on $\stck{_{rf}}$ relations based on a  Candidate Execution. For our purpose and flow in which we successively add relations to set of events, this would also include the implication on $\stck{_{rf}}$ relation while having a $\stck{_{sw}}$ relation among two events.}     

    %A new command to quickly use cons function in formal descriptions
\newcommand{\cons}[2]{\textit{cons}(#1,#2)}
  
%--------------------------------------------------------------------------------------------------------------   
    \subsection{Lemmas to assist our proof}    
    In order to assist our proof, we define two lemmas based on the ordering relations. 
    
    \begin{lemma} Consider three events $e$, $d$, and $k$. \\
    
        If
            \[
                \cons{e}{d} \ \wedge \ \reln{e}{ao}{d} \ \wedge \
                (
                    (\et{d}{uo}) \ \vee \
                    (\et{d}{sc} \ \wedge \ \event{d}{W})
                )
            \]
            
        then,
            \[
                \reln{k}{hb}{d}\ \Rightarrow\ \reln{k}{hb}{e}.
            \]
    \end{lemma}
    
    %An alternative short proof 
    \begin{proof}
        We have the following to be true :
            \begin{align*}
                cons(e,d) \ \wedge \ \reln{e}{ao}{d}.
            \end{align*}
        In both cases where $d$ is unordered or a sequentially consistent write, for any event $k$
        \[
            dir(k,d)\ \Rightarrow\ cons(k,d).
        \]
        
        An event that satisfies the above with $d$ is $e$. Because $\stck{_\textit{ao}}$ is a total order, $e$ will be the only event. This would mean that for any other $k \neq e$,
        \begin{align*}
            \reln{k}{hb}{d}\ \Rightarrow\ \reln{k}{hb}{e}.
        \end{align*}
        
        Note that although there could be a direct \textit{happens-before} relation with some event $k$ from \textit{another} agent, they are only relations satisfying $dir(d,k)$.
        
    \end{proof}

%---------------------------------------------------------------------------------------------------------------    
    
%SHORTER VERSION OF PROOF WITHOUT THE ENGLISH EXPLAINATION IN THE MIDDLE. DISCUSS AND DECIDE ON WHICH FORM IS BETTER
    \begin{lemma}Consider three events $e$, $d$ and $k$. \\
    
        If
            \[
                \cons{e}{d} \ \wedge \ \reln{e}{ao}{d} \ \wedge \
                (
                    (\et{e}{uo}) \ \vee \
                    (\et{e}{sc} \ \wedge \ \event{e}{R})
                )
            \]
            
        then,
            \[
                \reln{e}{hb}{k}\ \Rightarrow\ \reln{d}{hb}{k}.
            \]
    \end{lemma}
    
    %An alternative proof for this 
    \begin{proof}
        The proof is symmetric to that of Lemma 1. 
    \end{proof}

    \emph{Note that the above lemmas are only for events $k$ which are not of type \textit{init}}
    

\subsection{Valid reordering}
    We view reordering as manipulating the agent-order relation among two events. In that sense, reordering two consecutive events $e$ and $d$ such that $e \stck{_{ao}} d$ becomes:
    \[
        e \stck{_{ao}} d 
        \longmapsto
        d \stck{_{ao}} e 
    \]

    What implications this change has on the other ordering relations depends on the type of events $e$ and $d$ are and would require an analysis on each Candidate Execution. 
    The intuition is that the axioms of the memory model rely on certain ordering relations to restrict observable behaviors in a program.
    Hence, preserving these ordering relations would help us in turn not introduce new Observable Behaviors.
    In particular we note that preserving $\stck{_{hb}}$ relations (other than the one we eliminate intentionally i.e $\reln{e}{hb}{d}$) would suffice for our purpose. 
    Since $\stck{_{mo}}$ respects $\stck{_{hb}}$, we in turn even preserve the memory order which is essential.  

    In the end, we want to ensure that the set of possible observable behaviors of a program, remain unchanged after reordering. If that is not feasible, then we would want the set of observable behaviors after reordering at the very least to be a subset. This would ensure that the program does not have some new behaviours that weren't supposed to happen prior to reordering. 
    
    We begin by first defining a reorderable pair of events. We then formulate a theorem (with a proof) on the set of observable behaviors of a Candidate before and after reordering a pair of consecutive events which are reorderable. We consider reordering valid if the set of observable behaviours after reordering are a subset of the original. 

    \begin{definition}{Reorderable Pair (Reord)}
        We define a boolean function \emph{Reord} that takes two ordered pair of events $e$ and $d$ such that $\reln{e}{ao}{d}$ and gives a boolean value indicating if they are a reorderable pair. 
        
        \begin{align*}
            Reord(e,d) = \\
            (
            &((\et{e}{uo} \ \wedge \ \et{d}{uo}) \ \wedge \ 
                    (   
                        (\event{e}{R} \ \wedge \ \event{d}{R}) \ \vee \ 
                        (\Re(e) \cap_\Re \Re(d) = \phi) 
                    )
            ) \\ &\vee \\
            &((\et{e}{sc} \ \wedge \ \et{d}{uo}) \ \wedge \ 
                    (
                        (\event{e}{W} \ \wedge \ (\Re(e) \cap_\Re \Re(d) = \phi)) 
                    )
            ) \\ &\vee \\
            &((\et{e}{uo} \ \wedge \ \et{d}{sc}) \ \wedge \ 
                    (
                        (\event{d}{R} \ \wedge \ (\Re(e) \cap_\Re \Re(d) = \phi)) 
                    )
            )
            )
        \end{align*}

        \critic{purple}{Use the latter for the purpose at the end of the proof for reordering, to emphasize how we approached each case}

         
    \end{definition}

\begin{theorem} 

    Consider a candidate $C$ of a program and its possible \textit{Candidate Executions} where $\stck{_\textit{hb}}$ is strictly partial order. Consider two events $e$ and $d$ such that $\cons{e}{d}$ is true in $C$ and  $\reln{e}{ao}{d}$. Consider another candidate $C'$ resulting after reordering $e$ and $d$. 
    Then if \emph{Reord(e,d)} is true in $C$, the set observable behaviors possible due to Candidate Executions of $C'$ is a subset of that of $C$. 
\end{theorem}

\begin{proof}

    We look at this in terms of performing an instruction reordering on a candidate execution of $C$. We would want the resulting candidate execution to preserve all the other $\stck{_{hb}}$ relations (except $\reln{e}{hb}{d}$) and that any new $\stck{_{hb}}$ relations strictly reduce possible observable behaviors.
    
    The proof is structured as follows. We first show that existing \textit{happens-before} relations in any candidate execution of $C$ except $\reln{e}{hb}{d}$ remain intact after reordering. We then identify the cases where new \textit{happens-before} relations could be established. We identify from these cases whether \textit{happens-before} cycles could be introduced.
    We then show for the remaining cases that new relations do not introduce any new observable behaviors.

    The above steps can be summarized as addressing four main questions for any $Candidate Execution$ of $C'$
    \begin{enumerate}
        \item Apart from $\reln{e}{hb}{d}$, do other \emph{happens-before} relations remain intact?
        \item Apart from $\reln{d}{hb}{e}$, are any new \emph{happens-before} relations established? 
        \item Are any \emph{happens-before} cycles introduced? 
        \item Do the new relations bring new \emph{observable behaviors?}
    \end{enumerate}
    
    %The first two questions ensure that existing happens-before relations are intact
    
    \paragraph{1. Preserving \textit{happens-before} relations}
        
        If some $\stck{_{hb}}$ relations among events are lost after reordering, we may introduce new observable behaviors. 
        
        The relations that could be subject to change can be addressed by considering two disjoint sets of events in any \textit{Candidate Execution} of $C$ as below.
        
        \begin{align*}
            K_e = \{k \ | \ \reln{k}{hb}{e} \}. \\
            K_d = \{k \ | \ \reln{d}{hb}{k} \}. 
        \end{align*}
            
        %Show a figure here  (with Ke and Kd)
        \begin{figure}[H]
            \centering
            \includegraphics[scale=0.7]{Q1(a).pdf}
            \caption{For any Candidate Execution of $C$, the set $K_e$ and $K_d$}
            \label{fig:my_label}
        \end{figure}
        
        The idea is that if these relations are intact, then the relations among events from the first and second set will also hold due to transitivity.
        
        Consider two events $\event{p1}{K_e}$ and $\event{p2}{K_d}$ (When $e$ is the first event or $d$ is the last event, assume dummy events that can act as $p1$ or $p2$.) belonging to the same agent as that of $e$ and $d$ such that in $C$:
        \begin{align*}
            dir(p1,e)\ \wedge\ dir(d,p2).
        \end{align*}

        Note that in terms of direct happens-before relations, on reordering, any $Candidate Execution$ of $C$ will have the following changes
        
        %Show a figure here 
        \begin{figure}[H]
            \centering
            \includegraphics[scale=0.7]{Q1(b).pdf}
            \caption{The direct relation changes that can be observed while reordering events $e$ and $d$}
            \label{fig:my_label}
        \end{figure}
        
        The figure above is to show that, for any $Candidate Execution$ of $C$, the following is true
        \[
            cons(p1,e) \ \wedge dir(p1,e) \ \wedge dir(e,d) \ \wedge cons(d,p2) \ \wedge \ dir(d,p2).
        \]
        and for that of $C'$,
        \[
            cons(p1,d) \ \wedge \ dir(p1,d) \ \wedge \ dir(d,e) \ \wedge cons(e,p2) \ \wedge dir(e,p2).
        \]
        
        We need the following relations among $p1$, $p2$, $e$ and $d$ to be preserved in Candidate executions of $C'$ 
        \[
            \reln{p1}{hb}{e} \ \wedge \ \reln{p1}{hb}{d} \ \wedge \ \reln{d}{hb}{p2} \ \wedge \ \reln{e}{hb}{p2}.
        \]
        
        After reordering, we do have these relations preserved due to transitivity  
        \begin{gather*}
            \reln{p1}{hb}{d} \ \wedge \ \reln{d}{hb}{e} \ \Rightarrow \ \reln{p1}{hb}{e}. \\
            \reln{e}{hb}{p2} \ \wedge \ \reln{d}{hb}{e} \ \Rightarrow \ \reln{d}{hb}{p2}. \\
            \reln{p1}{hb}{d} \ \wedge \ \reln{d}{hb}{e} \ \wedge \ \reln{e}{hb}{p2} \ \Rightarrow \ \reln{p1}{hb}{p2}. 
        \end{gather*}
    
        If we can "pivot" the remaining set $K_e$ to $p1$ and $K_d$ to $p2$, it would ensure that our intended relations remain intact after reordering by transitivity. To state formally, we have a valid pair of pivots $<p1,p2>$ when the following two conditions hold
        \begin{gather*}
            \forall \ k \in K_e - \{p1\}, \ \reln{k}{hb}{p1}. \\
            \forall \ k \in K_d - \{p2\}, \ \reln{p2}{hb}{k}.
        \end{gather*}
        
        %Show a figure here
        \begin{figure}[H]
            \centering
            \includegraphics[scale=0.7]{Q1(d).pdf}
            \caption{For any Candidate execution, the intuition behind valid pivots $<p1,p2>$}
            \label{fig:my_label}
        \end{figure}
        
        By lemma 1 and lemma 2 respectively, we have for $C$, the following condition where $<p1, p2>$ is a valid pivot pair
        \begin{gather*}
            \et{e}{uo} \vee (\et{e}{sc} \wedge \event{e}{W}). \\
            \et{d}{uo} \vee (\et{d}{sc} \wedge \event{d}{R}).
        \end{gather*}
            
        The following table summarizes the cases where we have a valid pair of pivots $<p1,p2>$
        %Show a general table here 
        \begin{figure}[H]
            \centering
            \includegraphics[scale=0.7]{Table1_Final.pdf}
            \caption{Table summarizing whether we have valid pair of pivots based on  $e$ and $d$}
            \label{fig:my_label}
        \end{figure}
                
        We show a simple example where we do not have a valid pair of pivots, particularly because $p1$ is not a valid pivot. Note that in this example, $K_e = K_{e1} + K_{e2} + p1 + p_x$
        %Show figure here of program P
        \begin{figure}[H]
            \centering
            \includegraphics[scale=0.7]{Q1(e).pdf}
            \caption{A Candidate Execution where p1 is not a valid pivot}
            \label{fig:my_label}
        \end{figure}
        
        %Show figure here of program P'
         \begin{figure}[H]
            \centering
            \includegraphics[scale=0.7]{Q1(f).pdf}
            \caption{The resultant Candidate Execution after reordering, exposing the relations with $p_x$, $K_{e2}$ and $d$ that are lost}
            \label{fig:my_label}
        \end{figure}
        
            
        %MAKE SOME MAJOR CHANGES TO THIS
        \paragraph{2. Additional \textit{happens-before} relations}
        Although we have identified the cases when \textit{happens-before} relations are preserved, we also get some additional relations in some of them.
        
        %Show an example here and explain
        As an example, for the case when $d$ is a sequentially consistent read, by lemma 1, in any execution of $C$
        \[
            \reln{k}{hb}{d} \centernot\Rightarrow \reln{k}{hb}{e} 
        \]

        But in $Executions$ of candidate $C'$, by transitivity, we have 
        \[
            \reln{k}{hb}{d} \Rightarrow \reln{k}{hb}{e} 
        \]
        
        This is because, there are sets of relations that come through certain \textit{synchronize-with} relations. Thus, although we are able to preserve relations that existed in any $Candidate Execution$ of $C$, we also in the process, introduce new ones in $Candidate Executions$ of $C'$. The figure below shows pictorially an example of a Candidate Execution of $C$ for the case above 
        
        %Show figure here of program P 
        \begin{figure}[H]
            \centering
            \includegraphics[scale=0.7]{Q2(c).pdf}
            \caption{A Candidate Execution where $d$ is a sequentially consistent read}
            \label{fig:my_label}
        \end{figure}
        
        %Show figure here of program P'
        \begin{figure}[H]
            \centering
            \includegraphics[scale=0.7]{Q2(d).pdf}
            \caption{The Candidate Execution after reordering, exposing the new relations established with $e$, $p3$ and set $k$}
            \label{fig:my_label}
        \end{figure}
        
        \critic{purple}{Explain the above figures or perhaps highlight the new relations that are established.}
       
       
        To summarize, the table below shows the cases where new relations could be introduced. 
        %Show the table here
        \begin{figure}[H]
            \centering
            \includegraphics[scale=0.7]{Table2_Final.pdf}
            \caption{Table summarizing when new \textit{happens-before} relations could be introduced based on having valid pair of pivots }
            \label{fig:my_label}
        \end{figure}

        For these cases, we must know whether these new relations introduce new observable behaviors. 
        
%--------------------------------------------------------------------------------------------------------------------------------------
    \paragraph{4. Presence of cycles?}
        Before we go into analyzing whether new relations introduce observable behaviours, we first ensure there are no $\stck{_\textit{hb}}$ cycles introduced in the process. Consider the example below
        \begin{figure}[H]
            \centering
            \includegraphics[scale=0.7]{Q4(a).pdf}
            \caption{Caption}
            \label{fig:my_label}
        \end{figure}
        
        Notice that here, the axiom of coherent reads restricts $R$ to read from $W'$.
        \[
            \reln{R}{hb}{W'} \Rightarrow \neg \reln{R}{rf}{W'}
        \]
        
        But by transitive property, it is also the case that $\reln{W'}{hb}{R}$. 
        \[
            \reln{W'}{hb}{W} \ \wedge \ \reln{W}{hb}{R} \ 
            \Rightarrow \ 
            \reln{W'}{hb}{R}
        \]
        
        As per this, the axiom of coherent reads shouldn't restrict $\reln{R}{rf}{W'}$. To avoid such cases, we will need to ensure that no Candidate Execution of $C'$ after $e$ and $d$ are reordered have $\stck{_{hb}}$ cycles.
        
        Note that if a cycle exists after reordering, then 
        \begin{enumerate}
            \item The relations preserved do not themselves create a cycle (ref to the theorem)
            \item Additional new relations may introduce cycles
        \end{enumerate}
       
        The first part is straightforward as we assume we can only do reordering on Candidate Exectuions of $C$ not having cycles. 
        
        To address the second part, we first address the cases where $\reln{d}{hb}{e}$ may be part of the cycle. The other event $k$, may be either from the set $K_e$, $K_d$ or a new relation that is formed.
        %Show figure here
        \begin{figure}[H]
            \centering
            \includegraphics[scale=0.7]{Q4(a)_1.pdf}
            \caption{If k belongs to one of the sets $K_e$ or $K_d$}
            \label{fig:my_label}
        \end{figure}
        
        The above figure shows that $k$ cannot belong to either of the sets, as their relations with $e$ and $d$ will not result in a cycle. 
        
        For cases where $\reln{k}{hb}{e}$ is the set of new relations, note that by lemma 1
        \[
            \reln{k}{hb}{e} \Rightarrow \reln{k}{hb}{d}
        \]
        
        For cases where $\reln{d}{hb}{k}$ is the set of new relations, by lemma 2
        \[
            \reln{d}{hb}{k} \Rightarrow \reln{e}{hb}{k}
        \]
        
        So for both these cases also, a cycle with $\reln{d}{hb}{e}$ cannot exist. The following figure shows pictorially this fact. 
        %Show figure here
        \begin{figure}[H]
            \centering
            \includegraphics[scale=0.7]{Q4(a)_2.pdf}
            \caption{If $\reln{k}{hb}{e}$ or $\reln{d}{hb}{k}$ are new sets of relations}
            \label{fig:my_label}
        \end{figure}
        
        
        For the one case where we have two new sets of relations formed, i.e $\reln{d}{hb}{k}$ and $\reln{k}{hb}{e}$, we could have a case where $k$ is a common event for both sets. But, by lemma 1, we also have $\reln{k}{hb}{d}$ and by lemma 2, $\reln{e}{hb}{k}$. Thus, we have a cycle. The following figure shows this pictorially
        
        %Show figure here
        \begin{figure}[H]
            \centering
            \includegraphics[scale=0.7]{Q4(a)_3.pdf}
            \caption{A cycle exists in the case where we have two new sets of relations ($\reln{k}{hb}{e}$ and $\reln{d}{hb}{k}$) }
            \label{fig:my_label}
        \end{figure}
        
        \critic{purple}{Maybe have a better figure, meaning a set of relations where each figure shows clearly which relaiton is implied due to whcih lemma}

        Now for the case when $\reln{d}{hb}{e}$ may not be part of the cycle, we have only two other relations, $\reln{k}{hb}{e}$ or $\reln{d}{hb}{k}$.
        
        Considering the first scenario where the new set of relations are of the form $\reln{k}{hb}{e}$. Suppose a cycle exists with another event $k'$. Then 
        \[
            \reln{k}{hb}{e} \ \wedge \
            \reln{e}{hb}{k'} \ \wedge \
            \reln{k'}{hb}{k}
        \]
        
        Note that the latter two relations are not new, since the only new set of relations are of the first form. Now, by lemma 1 and by transitivity respectively
        \begin{gather*}
            \reln{k}{hb}{e} \Rightarrow \reln{k}{hb}{d} \\
            \reln{e}{hb}{k'} \Rightarrow \reln{d}{hb}{k'}    
        \end{gather*}
        
        So, the following is also a cycle
        \[
            \reln{k}{hb}{d} \ \wedge \
            \reln{d}{hb}{k'} \ \wedge \
            \reln{k'}{hb}{k}
        \]
        
        But these relations already exist in the original Candidate Execution, which implies a cycle existed before reordering. This contradicts our assumption that we only reorder when the Candidate Executions of $C$ have no cycles. Thus, by contradiction such a cycle cannot exist.
        
        In similar lines for the cases where the set of new relations are of the form $\reln{d}{hb}{k}$, we can show by contradiction that a cycle cannot exist.

%-------------------------------------------------------------------------------------------------------------------------------------

        %Needs a few major changes and tables split cases into disjoint, overlapping and equal ranged events 
        \paragraph{4. Do new relations introduce new observable behaviors?}
        In any candidate execution, reordering events $e$ and $d$ eliminates the relation $\reln{e}{hb}{d}$ and introduces the new relation $\reln{d}{hb}{e}$. 
        New behaviours created by the latter directly, if any, are 
        of course intentional (and should normally be avoided by ensuring $e$ and $d$ are independent), but we need to ensure that this does not also result in new behaviours indirectly. 
        
        On observing the role on the axioms on this relation, notice that if both $e$ and $d$ are read events then the range does not matter. For all other cases, if events $e$ and $d$ have overlapping ranges, one could introduce a new observable behavior after reordering them (a simple use of Coherent Reads / Sequentially Consistent Atomics).     
        
        Any other new relations that are introduced can be divided into 4 cases, in terms of our events $e$ and $d$ and the new relation with some event $k$:
        %Show a figure here summarizing the four cases
        \begin{figure}[H]
            \centering
            \includegraphics[scale=0.7]{Q3(a).pdf}
            \caption{Caption}
            \label{fig:my_label}
        \end{figure}
        
        \critic{purple}{Change the figure above to represent only the first four cases}

        In each of the above cases, note firstly that we need to only consider cases where their ranges are overlapping/equal.
        
        %Addressing the first case. 
        Figure below shows a breakdown of sub-cases for the first case (a), varying based
        on the nature of event $k$.
        %Show all cases here for different k
        \begin{figure}[H]
            \centering
            \includegraphics[scale=0.6]{Q3_(b)Case1.pdf}
            \caption{The role of the axioms on introducing a new relation between an unordered Read and some event $k$}
            \label{fig:my_label}
        \end{figure}
        
        %Might have to elaborate this more
        \begin{enumerate}
            \item For (i), when $k$ is a read, none of the rules have any implications on observable behaviors.
            \item For (ii), when $k$ is a write, the rule of coherent reads (ii(a)) or sequentially consistent atomics (ii(b)) could restrict the read ($e$) from reading overlapping ranges of $W'$ with $W$.
        \end{enumerate}
        
        The above case analysis shows us that the new relation could 'trigger' the consistency rules, only to restrict possible reads-from relations, thus restricting possible observable behaviors. 

        The other cases, also have instances which can 'trigger' some cases of the axioms, thus restricting possibly some $\stck{_{rf}}$ relations.b These cases are shown in the figures below: 
        \begin{figure}[H]
            \centering
            \includegraphics[scale=0.6]{Q3_(c)Case2.pdf}
            \caption{(i) and (ii(b)) satisfy the axiom of Coherent Reads}
            \label{fig:my_label}
        \end{figure}
              
        \begin{figure}[H]
            \centering
            \includegraphics[scale=0.6]{Q3_(d)Case3.pdf}
            \caption{(ii) satisfies the axiom of Coherent Reads}
            \label{fig:my_label}
        \end{figure}
        
        \begin{figure}[H]
            \centering
            \includegraphics[scale=0.4]{Q3_(e)Case4.pdf}
            \caption{(i(a)), (ii(a)) satisfy the axiom of Coherent Reads, whereas (i(b)), (ii(b)) satisfy the axiom of Sequentially Consistent Atomics}
            \label{fig:my_label}
        \end{figure}
        
        \critic{blue}{The main reason for this is that we framed he axioms in a form that restricts \textit{reads-from} relations. So in any case where adding an additional \textit{happens-before} relation "triggers" an axiom, we are bound to have some behaviors restricted. It is this fact that is elicited explicitly by going case wise on all relations that are introduced.}
     
        The table below summarizes the valid cases where, we have a pair of valid pivots, where new relations do not introduce new observable behaviors and do not have cycles. 
        %Show the table here
        \begin{figure}[H]
            \centering
            \includegraphics[scale=0.7]{Table4_Final.pdf}
            \caption{The final table summarizing the valid cases where observable behaviors will only be a subset after reordering.}
            \label{fig:my_label}
        \end{figure}

        \critic{blue}{Keep in mind that the comparision of ranges is done while addressing question 3 in the proof, so the table above, implicitly also takes into account only the valid cases where ranges are also correct}
    
        The table above, precisely is the definition of a reorderable pair. If we write the above table in the form of an expression we have an expanded format of our Reorderable pair function. 

        \begin{align*}
            Reord(e,d) = \\
            (
            ((\et{e}{uo} \wedge \et{d}{uo}) \ \wedge \\ 
                \quad ( 
                        &(\event{e}{R} \wedge \event{d}{R}) \vee \\ 
                        &(\event{e}{W} \wedge \event{d}{R} \wedge (\Re(e) \And \Re(d) = \phi)) \vee \\
                        &(\event{e}{R} \wedge \event{d}{W} \wedge (\Re(e) \And \Re(d) = \phi)) \vee \\
                        &(\event{e}{W} \wedge \event{d}{W} \wedge (\Re(e) \And \Re(d) = \phi)) 
                    )
            ) \\ \vee \\
            ((\et{e}{sc} \wedge \et{d}{uo}) \ \wedge \\
                \quad (
                        & (\event{e}{W} \wedge \event{d}{R} \wedge (\Re(e) \And \Re(d) = \phi)) \vee \\
                        & (\event{e}{W} \wedge \event{d}{W} \wedge (\Re(e) \And \Re(d) = \phi)) 
                    )
            ) \\ \vee \\
            ((\et{e}{uo} \wedge \et{d}{sc}) \ \wedge \\
                \quad (
                        & (\event{e}{R} \wedge \event{d}{R} \wedge) \vee \\
                        & (\event{e}{W} \wedge \event{d}{R} \wedge (\Re(e) \And \Re(d) = \phi)) 
                    )
            )
            )
        \end{align*}
        
        \qed  
\end{proof}

\begin{corollary}
    Consider a Candidate C of a program and its Candidate Executions which are valid. Consider two events $e$ and $d$ such that $\neg \cons{e}{d}$ is true in C and $\reln{e}{ao}{d}$. Consider another Candidate C' resulting after reordering $e$ and $d$ in C. Then, the set of Observable behaviors possible in C' is a subset of C only if $Reord(e,d)$ and the following holds true.
    
    \[
        \forall \ k \ \textit{s.t.} \ 
        \reln{e}{ao}{k} \ \wedge \ \reln{k}{ao}{d} \ . \ 
        Reord(e,k) \ \wedge \ Reord(k,d)
    \]
    
\end{corollary}
    
\begin{proof}
    
\end{proof}
    
\subsection{Counter examples for the invalid cases}

\subsection{}

    

    
    
    
    
    
    
    \section{Discussion} 

    %One of the key problems to tackle in relaxed memory consistency is to contain the sheer complexity of possible behaviors of a program by giving a concise set of semantics to while also keeping it as intuitive as possible. It also becomes important to abstract away unnecessary complexity specific to the standard / implementations which may not be necessary to understand the semantics for a specific purpose. Lastly, it is important to frame the semantics in a way appropriate to address specific concern which for our purpose was optimizations. This is exactly what we intended while describing the ECMAScript memory model. 
    
%    Our formal description of the model attempts to abstract away the complexities that may not be required, thus creating a clear separation of concerns while addressing these transformations related to only shared memory events.
    %The consistency rules were framed restricting possible reads-from/read-bytes-from relations, contrast to it being restricting possible happens-before relations in the standard. This made it easier to intuitively think of how the rules affect possible observable behaviors when we attempt reordering two events. 
 
    %While performing optimizations (in our case reordering), the complexity of possible executions may seem overwhelming to take into account in order to ensure that it does not introduce unwanted behaviors. 
    Theorem 1 and its corollary together give us a set of conditions that just need to be checked in addition while performing reordering of relaxed memory events. Having these set of conditions helps us avoid addressing the data-flow complexity due to different executions of the program using such accesses.    
%    We also have counter examples for each case that we consider performing such reordering in general is not safe, thus ensuring that we are not disallowing reordering that must be valid for any program (given our conditions of course).
    
    It is important to note that our approach is conservative, and one might be able to do reordering without causing new observable behaviors to occur even in cases that do not not satisfy our conditions. 
    This is possible because certain happens-before relations may not be essential and hence discarding them will not result in any invalid observable behavior.
    Getting such information would require an analysis that takes into account relations that we cannot obtain using just intra-thread information, which in practice might be infeasible as the number of threads and events increase. (One such well studied analysis is May-Happen-In-Parallel, whose origins come from the work done by Naumovich et al.~\cite{NaumovichA}).
    
    It is also important to note that we focus on Candidates rather than the Program.  We do not in this
    work consider the specifics of identifying all possible candidates of a given program, and we we assume that whatever candidate considered is a possible one for the original program. 
    This translation from program to a set of candidates is something that would be needed in order to practically incorporate our set of conditions in practice while doing transformations.
    
%    Lastly, we would like to mention that we have also looked at redundant read elimination among two reads and redundant write elimination among two writes, by answering the same four basic questions that we addressed in our proof above. 
%    We are still writing it more formally in latex format, but the idea for the proof is done. 
    
    %We do not address the role of synchronize events or host-specific synchronization events in our approach. One would need to account for their role too. 
    
    
    \section{Conclusion and Future work}
Our more declarative approach to the ECMAScript memory consistency model results in a relatively compact and concise description of the semantics.
This better facilitates mathematical reasoning, which we have used to investigate the conditions on basic optimization operations such as instruction reordering.

Future work is aimed at extending our analysis to validate the conditions for redundancy elimination.

We are also interested in further exploring the constraints implied by the potential for multi-byte (non-atomic) accesses.  The current standard imposes only very weak conditions on overlapping accesses, but stronger conditions are likely necessary to reflect actual programming practice.


    There are countless people who have influenced me through these years and have motivated me to keep doing this thesis.
Unfortunately, I cannot mention all of their names due to space constraints as well as the fact that I may not know their influence on me yet.
However, I do want to mention a few important names of people that have in my eyes influenced me the most. 
Firstly, my advisor Professor Clark Verbrugge.
Having not done any rigorous theoretical work or research before and yet wanting to do a theoretical one as a thesis, he gave me high level guidelines to ground me and to keep me from being overwhelmed from framing my own theorems, lemmas, proofs, etc. 
I would not have gotten a better advisor than him, especially during the initial years when I was trying to build an intuitive foundation. 
Also, to advise someone in this sense is quite a challenge in my eyes.
Secondly, my friend, Aarti Kashyap, mainly for being a patient listener, a motivator that made me attend several conferences and for always giving constructive feedback. 
I needed an outside perspective and somebody to discuss with about various aspects of this research at any time; and she being there for it has helped me shape ideas of this thesis better. 
Thirdly, Conrad Watt, mainly for giving me an inside perspective of this topic of research and motivating me to pursue this thesis topic.
Lastly, my family, friends and lab-mates, who always assured me the support (mentally or otherwise) I would need at any time during my masters. 
I would like to also thank Viktor Vafeiadis, my internship advisor, who helped me look at the research I did all this while in a simplistic way.
This has in some way reflected the way I write or present my research ideas.
Finally, God, for guiding me in his/her own way and presenting me with ample opportunities to grow as a human being as well as a researcher these years.

\nocite{*}
    \bibliographystyle{splncs04}
    \bibliography{references}
   
\end{document}
