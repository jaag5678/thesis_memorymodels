     
With the increasing demand for performance in computing, the use of concurrency has become quite ubiquitous. 
In particular, the use of \textit{shared memory} concurrency has shown to give tremendous performance benefits. 
A big part of this is due to using \textit{relaxed memory} accesses, which sacrifice the string ordering properties of sequentially specified actions to gain high performance. 
Lack of understanding of the impact of such accesses, can cause programs to mis-behave.
The semantics describing the impact of such accesses on program outcomes is specified by a memory consistency model. 
In many contexts, there is an absence of clear, well-defined semantics of the memory model, making it difficult for the user to utilize these accesses correctly.
In addition, usage of such accesses may invalidate several aggressive optimizations that the compiler does which contributes to a major chunk of our desired performance.

In current times, there is an increasing trend/need to offload computations on the web. 
One major benefit of this is that portability issues are reduced greatly, which means one can run complex software and use it on one's browser even though a system may not be compatible to run it. 
To ensure that computations done on the web be comparable with those done offline, performance plays a major factor, and this factor has always pushed people to not rely on web-based computations. 

In this scenario, the use of shared memory access (more specifically, relaxed accesses) naturally fits in to provide the lack of performance we so desperately need.
Recently, ECMAScript, the standard for all such web-scripting languages defined their own memory consistency model\cite{ECMA}.
Its semantics, however, is written in prose format, thus making it difficult to correctly utilize such accesses in programs such as that of JavaScript. 
Besides, no formal proof about the validity of common program transformations under such a semantics is given.
There are just suggestions/guidelines for programmers in this respect, but they are example driven.
In addition, this model also attempts to describe semantics for mixed-size accesses, something which is quite recent and not precisely defined for even standard memory models. 

Our focus, therefore, in this thesis is to first offer a clarified, more concise rendition of the core ECMAScript memory model that allows for better abstract reasoning over allowed and disallowed behaviors (outcomes). 
We use our model to provide an intuitive, conservative proof of basic operations core to program optimizations such as reordering and elimination of instructions is permitted, addressing optimization in terms of its impact on observable program behaviors. 
We use this insight to prove validity of a subset of \textit{loop-invariant code-motion}\cite{Muchnick}. 

We do the above by first giving a \textit{Declarative} (commonly known as Axiomatic) specification of the ECMAScript memory model. 
This model is described using \textit{partial-orders} among events that constitute a program.
We use these orders to define a program \textit{execution} and its \textit{outcomes}(referred as Observable Behaviors).
The \textit{axioms} of the model help us define which of these outcomes are valid.
We then use this axiomatic formulation to reason about \textit{reordering} and \textit{elimination} using program executions.
We then move towards program level by reasoning about these transformations in the presence of conditionals and loops. 
Throughout this process, we use examples wherever required to give a better intuitive understanding of the proofs as well as the axioms of the model. 
We conclude with the limitations/advantages of our approach.
We also give further steps that can be taken to address such limitations as well as pending work that could be done.
Later, we critique the model using some examples, hinting towards under-specification in certain parts.
We end by exposing certain foundational problems/directions that we identified in the process of doing this thesis which in our eyes should be addressed/investigated in the immediate future.

