\begin{corollary}
    \label{LoopInvCodeMotWrite1}
    Consider $K$ to be the set of events within a loop in program $P$. 
    Consider $e$ to be a write within the loop. 
    Consider program $P'$ with event $e$ agent ordered before the loop. 
    If
    \begin{gather}
        \et{e}{uo} \label{licmw1:eq1}\\
        \forall k \neq e \in K, \ Reord(k, e) \label{licmw1:eq2} \\ 
        \nexists C^i \in P \ \text{s.t.} \ e^{j<=i} \notin C  \label{licmw1:eq3}                    
    \end{gather}
    then the set of observable behaviors of $P'$ is a subset of $P$.

    \critic{red}{It is interesting to note that we do not need $Reord(e,k)$ which was originally our plan because we needed to eliminate every $e^j$ w.r.t $e^{j+1}$. Because we have $Reord(k,e)$, we can get all the writes to be consecutive to each other. Thus by Theorem 1 directly, we can eliminate them all. We do not require Corollary of Elimination here.}
\end{corollary}             

\begin{proof}


    We first consider the program with just one iteration. Hence for Candidate $C^1$, we have just $e^1$. 
    We need to ensure that the resultant candidate $C'^1$ such that 
    \begin{align*}
        \forall k \in K, \ \reln{e}{ao}{k}
    \end{align*}  
    has observable behaviors as a subset of $C^1$

    For every $k \in K$ such that $\reln{k}{ao}{e}$ we need them to be reorderable with respect to $e$ to bring it outside the loop.
    From condition \ref{licmw1:eq2} we have $Reord(k,e)$ for any such $k$.
    By Corollary \ref{CorollCodeMotion1}, we can infer that $C'^1$ has observable behaviors as a subset of $C^1$.
    To extend this to the program level with one iteration, we also need that $e$ should not be in any conditional branch.
    Condition \ref{licmw1:eq3}, from Prop \ref{CondB1} implies that $e$ is not part of any conditional branch.
    Thus, from Corollary \ref{ReordCond}, we can infer for program with one iteration of the loop that reordering outside the loop is safe. 
    
    Next, we consider the program with more than one iteration of the loop. 
    We prove this case using induction on the number of reads $e$ that exist due to multiple iterations of the loop. 


    \begin{itemize}

        \item Base case : number of $e$ = 2
    
        This case corresponds to candidates of the form $C^2$, thus giving us two writes eads $e^1$ and $e^2$.
        We need candidate $C'^2$ with just one such event $e$, such that:
        \begin{align*}
            \forall k \in K, \ \reln{k}{ao}{e}
        \end{align*}  
        We first reorder both the reads $e^1$, $e^2$ to be outside the loop, naming it Candidate $C''^2$.
        Because we have Condition \ref{licmr2:eq2}, by Corollary \ref{CorollCodeMotion1}, we can infer that $C''^2$ has observable behaviors as a subset of $C^2$.
        To go from $C''^2$ to $C'^2$, note that in $C''^2$ we have $cons(e^1, e^2)$ after reordering them. 
        From Condition \ref{licmr2:eq1}, by Theorem \ref{WriteElim}, we can eliminate $e^1$, thus resulting in $C'^2$ whose observable behaviors is a subset of $C''^2$\footnotemark.

        \footnotetext{Note that here, it is only safe to eliminate the first write ($e^1$) in contrast to having the choice to eliminate either $e^1$ or $e^2$ when both are reads.}
        
        By transitive property of subsets we can infer that $C'^2$ has observable behaviors as a subset of $C^2$.
        
        \item Inductive case : number of $e$ = n

        Assume that for all such candidates with $n$ iterations of the loop, the observable behaviors of $C'^n$ is a subset of $C^n$.

        We now prove using this that it can also hold for number $e$ as $n + 1$. 
        This case corresponds to candidates of the form $C^{n+1}$, thus giving us $n+1$ reads $e^1, e^2,...,e^{n+1}$.
        From Condition \ref{licmw1:eq2} we can infer 
        \begin{align*}
            \forall k \ \text{s.t.} \ \reln{e^n}{ao}{k} \wedge \reln{k}{ao}{e^{n+1}}, \ Reord(k,e^{n+1})
        \end{align*}
        By Corollary \ref{CorollCodeMotion1}, we can infer that $C''^{n+1}$ with $cons(e^n, e^{n+1})$ has observable behaviors as a subset of $C^{n+1}$. 
        From Condition \ref{licmw1:eq1}, by Theorem \ref{WriteElim}, we can eliminate $e^{n}$, thus giving us candidate of the form $C^n$ whose observable behaviors is a subset of $C''^{n+1}$.
        From our inductive assumption, we can then conclude that $C'^{n+1}$ has observable behaviors as a subset of $C^n$. 
        By transitive property of subsets, we can infer that $C'^{n+1}$ has observable behaviors as a subset of $C^{n+1}$.

        \critic{blue}{Mind the notations. Perhaps have another reivew of it to avoid confusion of the reader.}

    \end{itemize}

    From Condition \ref{licmw1:eq3}, by Corollary \ref{ReordCond}, we can infer that the observable behaviors of $P'$ is a subset of $P$.

\end{proof}


\begin{corollary}
    \label{LoopInvCodeMotWrite2}
    Consider $K$ to be the set of events within a loop in program $P$. 
    Consider $e$ to be a write within the loop. 
    Consider program $P'$ with event $e$ agent ordered before the loop. 
    If
    \begin{gather*}
        \et{e}{uo} \label{licmw2:eq1}\\
        \forall k \neq e \in K, \ Reord(e, k) \label{licmw2:eq2}\\ 
        \nexists C^i \in P \ \text{s.t.} \ e^{j<=i} \notin C  \label{licmw2:eq3}                    
    \end{gather*}
    then the set of observable behaviors of $P'$ is a subset of $P$.

    \critic{red}{It is interesting to note that we do not need $Reord(e,k)$ which was originally our plan because we needed to eliminate every $e^j$ w.r.t $e^{j+1}$. Because we have $Reord(k,e)$, we can get all the writes to be consecutive to each other. Thus by Theorem 1 directly, we can eliminate them all. We do not require Corollary of Elimination here.}
\end{corollary}             

\begin{proof}

    We first consider the program with just one iteration. 
    Hence for Candidate $C^1$, we have just $e^1$. 
    We need to ensure that the resultant candidate $C'^1$ such that 
    \begin{align*}
        \forall k \in K, \ \reln{k}{ao}{e}
    \end{align*}  
    has observable behaviors as a subset of $C^1$

    For every $k \in K$ such that $\reln{k}{ao}{e}$ we need them to be reorderable with respect to $e$ to bring it outside the loop.
    From condition \ref{licmw2:eq2} we have $Reord(k,e)$ for any such $k$.
    By Corollary \ref{CorollCodeMotion1}, we can infer that $C'^1$ has observable behaviors as a subset of $C^1$.
    To extend this to the program level with one iteration, we also need that $e$ should not be in any conditional branch.
    Condition \ref{licmw2:eq3}, from Prop \ref{CondB1} implies that $e$ is not part of any conditional branch.
    Thus, from Corollary \ref{ReordCond}, we can infer for program with one iteration of the loop that reordering outside the loop is safe. 
    
    Next, we consider the program with more than one iteration of the loop. 
    We prove this case using induction on the number of reads $e$ that exist due to multiple iterations of the loop. 

    \begin{itemize}

        \item Base case : number of $e$ = 2
    
        This case corresponds to candidates of the form $C^2$, thus giving us two reads $e^1$ and $e^2$.
        We need candidate $C'^2$ with just one such event $e$, such that:
        \begin{align*}
            \forall k \in K, \ \reln{k}{ao}{e}
        \end{align*}  
        We also have from property of loops that $\reln{e^1}{ao}{e^2}$.
        Having Condition \ref{licmw2:eq1}, \ref{licmw2:eq2}, by Corollary \ref{CorolWriteElim}, we can eliminate $e^1$, giving us $C''^2$ whose observable behaviors as a subset of $C^2$.  
        From Corollary \ref{CorollCodeMotion2}, we can reorder $e^2$ outside the loop thus giving us $C'^2$ whose observable behaviors as a subset of $C''^2$.
        
        By transitive property of subsets we can infer that $C'^2$ has observable behaviors as a subset of $C^2$.
        
        \item Inductive case : number of $e$ = n

        Assume that for all such candidates with $n$ iterations of the loop, the observable behaviors of $C'^n$ is a subset of $C^n$.

        We now prove using this that it can also hold for number $e$ as $n + 1$. 
        This case corresponds to candidates of the form $C^{n+1}$, thus giving us $n+1$ reads $e^1, e^2,...,e^{n+1}$.
        From Condition \ref{licmw2:eq2}, we can infer 
        \begin{align*}
            \forall k \ \text{s.t.} \ \reln{e^n}{ao}{k} \wedge \reln{k}{ao}{e^{n+1}}, \ Reord(e^{n}, k)
        \end{align*}
        Having Condition \ref{licmw2:eq1}, by Corollary \ref{CorolWriteElim}, we can eliminate $e^{n}$, thus giving us candidate $C^{n}$ whose observable beahviors are a subset of $C^{n+1}$

        From our inductive assumption, we can then conclude that $C'^{n+1}$ has observable behaviors as a subset of $C^n$. 
        By transitive property of subsets, we can infer that $C'^{n+1}$ has observable behaviors as a subset of $C^{n+1}$.

        \critic{blue}{Mind the notations. Perhaps have another reivew of it to avoid confusion of the reader.}

        \critic{red}{Proof read later.}

    \end{itemize}
    
    From Condition \ref{licmw2:eq3}, by Corollary \ref{ReordCond} and \ref{WriteElimCond}, we can infer that the observable behaviors of $P'$ is a subset of $P$.

\end{proof}
