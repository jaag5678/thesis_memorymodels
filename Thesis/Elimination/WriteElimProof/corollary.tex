\begin{corollary}
    \label{CorolWriteElim}
    Consider a Candidate C of a program and its Candidate Executions which are valid. Consider two events $e$ and $d$ both having equal ranges such that:
    \begin{align*}
        \event{e}{W} \ \wedge \ \event{d}{W} \ \wedge \ \et{e}{uo} \ \wedge \ \reln{e}{ao}{d} \ \wedge \ \neg\cons{e}{d}
    \end{align*} 
    Consider another Candidate C' without the event $e$. If
    \begin{align*}
        \forall k \ \text{s.t.} \ \reln{e}{ao}{k} \wedge \reln{k}{ao}{d} \ , \
        Reord(e, k)
    \end{align*}
    Then, the set of Observable behaviors possible in C' is a subset of C.
\end{corollary}

\begin{proof}
    We prove by induction on the number of events $k$ between $e$ and $d$. We verify that if a $j$ exists that is valid, the Observable behaviors of $C'$ is a subset of $C$.

    \paragraph{Base Case : n = 1}

        We have the case when:
        \begin{align*}
            \reln{e}{ao}{k_1} \ \wedge \ reln{k_1}{ao}{d}
        \end{align*}

        By Theorem of Reordering and Def of consecutive events and agent order, we can reorder $e$ and $k_1$, thus giving us a Candidate $C''$ with :
        \begin{align*}
            \reln{k_1}{ao}{e} \ \wedge \ \reln{e}{ao}{d}
        \end{align*}  
        whose observable behaviors are a subset of $C$.

        By Def of Consecutive instructions and Theorem of Elmination, we can eliminate $e$, thus giving us candidate $C'$  with  
        \begin{align*}
            \reln{k_1}{ao}{d}
        \end{align*} 
        whose observable behaviors are a subset of $C''$.

        By transitive property of subsets, we can conclude that the observable behaviors of $C'$ is a subset of $C$. 
    \paragraph{Inductive Case (n)}

        Let us assume that if the number of events in between are $n$, then the corollary holds. Let us consider the Candidate to be $C_n$ and corresponding candidate after elimination as $C'_n$. The observable behavior of $C'_n$ is a subset of that of $C_n$.

        If we can show the above holds true for $n+1$ events, we are done.
        
        To show this, suppose we have $C_{n+1}$ as the candidate and $C'$ as the one after elimination of $e$. 
        
        Because $\stck{_{ao}}$ is a total order, there is a total order among all $n+1$ events $k$ agent ordered between $e$ and $d$ such that we can label them $k_1, k_2 , ... , k_{n+1}$ with the following properties
        \begin{align*}
            \reln{e}{ao}{\reln{k_1}{ao}{\reln{...}{ao}{\reln{k_{n+1}}{ao}{d}}}} 
            \ \wedge \ 
            cons(e,k_1) \wedge cons(k_1, k_2) \wedge ... \wedge cons(k_{n+1}, d) 
        \end{align*}
        By Theorem of Reordering and Def of consecutive events and agent order, we can reorder $e$ and $k_1$, thus giving a corresponding candidate $C_n$ having observable behaviors as a subset of $C_{n+1}$. 

        By our inductive assumption, we have that the observable behaviors of $C'$ is a subset of $C_n$. By transitive property of subsets, we can then conclude that the observable behaviors of $C'$ are a subset of that of $C_{n+1}$.

\end{proof}

\critic{blue}{The above proof is clear, but it seems to me that I need to label all definitions and lemmas and theorems and corollary so that I can refer them here.}