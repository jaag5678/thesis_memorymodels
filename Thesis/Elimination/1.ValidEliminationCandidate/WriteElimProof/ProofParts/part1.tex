\paragraph{Preserving Happens-before relations}
        
    The relations we want to preserve are those that are dervied through relation with $e$, meaning the following two relations:
    \begin{tasks}(2)
        \task $\reln{k}{hb}{e}$
        \task $\reln{e}{hb}{k}$
    \end{tasks}

    We can divide the events involved in the above into two sets:
    \begin{align*}
        K_b = \{k \ | \ \reln{k}{hb}{e} \}. \\
        K_a = \{k \ | \ \reln{e}{hb}{k} \}. 
    \end{align*}

    
    We need to ensure the following relations hold after elimination.
    \begin{align}
        \forall k_a \in K_a \ \wedge \ \forall k_b \in K_b \ . \ \reln{k_b}{hb}{k_a}
    \end{align}
    
    Similar to reordering, we need to have a valid pivot pair $<p_b, p_a>$ such that 
    \begin{align}
        \forall k_b \neq p_b \in K_b \ . \ \reln{k_b}{hb}{p_b} \\
        \forall k_a \neq p_a \in K_a \ . \ \reln{p_a}{hb}{k_a} 
    \end{align}

    By Lemma \ref{Lemma1}, $\et{e}{uo} \vee \et{e}{sc}$ are the cases where $p_b$ can be a valid pivot. 
    By Lemma \ref{Lemma2}, $\et{e}{uo}$ is the case when $p_a$ (which in our case here is $d$) can be a valid pivot. 
    Since we need a valid pivot pair, $\et{e}{uo}$ is the only case where this is possible. 

    \critic{blue}{Put a figure here for an intuitive understanding of the problem at hand}
