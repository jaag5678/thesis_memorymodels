\paragraph{1. Preserving \emph{happens-before} relations}
        The relations we want to preserve are those that are dervied through relation with $e$, meaning the following two relations:
        \begin{tasks}(2)
            \task $\reln{k}{hb}{e}$
            \task $\reln{e}{hb}{k}$
        \end{tasks}

        We can divide the events involved in the above into two sets:
        \begin{align*}
            K_b = \{k \ | \ \reln{k}{hb}{e} \}. \\
            K_a = \{k \ | \ \reln{e}{hb}{k} \}. 
        \end{align*}

        \critic{blue}{Put a figure here for an intuitive understanding of the problem at hand}

        We need to ensure the following relations hold after elimination.
        \begin{align}
            \forall k_a \in K_a \ \wedge \ \forall k_b \in K_b \ . \ \reln{k_b}{hb}{k_a}
        \end{align}

        \critic{red}{Slight notational confusion}
        \critic{red}{WHat if the eliminated event is a conditional check? That would mean events in the conditional check are also eliminated. Which would mean one has to check if it is okay to eliminate all events within the conditional.}

        Similar to reordering, we need to have a valid pivot pair $<p_b, p_a>$ such that 
        \begin{align}
            \forall k_b \neq p_b \in K_b \ . \ \reln{k_b}{hb}{p_b} \\
            \forall k_a \neq p_a \in K_a \ . \ \reln{p_a}{hb}{k_a} 
        \end{align}

        By Lemma 1, $\et{e}{uo}$ is the only condition that satisfies our requirement. By Lemma 2, $\et{e}{uo} \ \vee \ \et{e}{sc}$ are the options. Condsidering both the above conditions to be satisfied, $\et{e}{uo}$ is the only possibility that holds. 

        \critic{blue}{Write an expression which is the conjunction of both lemmas, and show how the conjunction boils down to the result that we come to.}