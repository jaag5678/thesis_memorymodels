\section{Steps Further}

    \subsection{Addressing Read-Modify-Write}
    \begin{enumerate}
        \item So far we have assumed that no read modify write events exist in programs.
        \item However, this assumption is too strong in general.
        \item Analysis of the validity of reordering/elimination when RMW events are involved should  be done to have a complete analysis of these two transformations as far as shared memory accesses are concerned.
    \end{enumerate}

    \subsection{Incorporating Tearing Factor}
    \begin{enumerate}
        \item The role of tearing is still not clear to us.
        \item Axiom \ref{TfRe} does not rely on any partial order relations other than reads-from. 
        \item Since our approach is mainly reliant on preserving happens-before, our intuition is that our results should ideally be independant of the tearing factor.
        \item However, a proof including tearing aspect for each event is still needed.   
    \end{enumerate}

    \subsection{Role of synchronize/host-specific events}
    \begin{enumerate}
        \item We have not yet considered the role of synchronize events. 
        \item Though for a programmer this is equivalent to wait and notify, reordering and elimination under their presence is something we have not conisdered. 
        \item This we suspect would require understanding the operational aspect of wait / notify procedures.
    \end{enumerate}

    \begin{enumerate}
        \item We do not yet know how Host Specific synchronize events work with relaxed memory acceses.
        \item Strictly speaking, the semantics from a consistency model perspective is same as that of synchronize events. 
        \item However, a detailed analysis must be done before incorporating its role. 
    \end{enumerate}

    \subsection{Addressing other basic program transformations}
    \begin{enumerate}
        \item Addressing redundancy introduction as an immediate next step would prove useful.
        \item Using it, we can analyze reordering of events accross loops. 
        \item This will also give an interesting equivalence to instruction reordering. 
        \item Other program transformations we find important to consider are strengthening/ weakening access modes (fence optimizations), gathering optimization, changing tearing factor of accesess. 
    \end{enumerate}

    