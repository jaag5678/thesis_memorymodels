\begin{corollary}
    \label{LoopInvCodeMotRead1}
    Consider $K$ to be the set of events within a loop in program $P$. 
    Consider $e$ to be a read within the loop, not being a conditional check. 
    Consider program $P'$ with event $e$ agent ordered before the loop. 
    If
    \begin{gather}
        \et{e}{uo}. \label{licmr1:eq1}\\
        \forall k\!\neq\!e \in K, \ Reord(k, e). \label{licmr1:eq2}\\ 
        \nexists C^i \in P \ \text{s.t.} \ e^{j \leq i} \notin C.  \label{licmr1:eq3}                    
    \end{gather}
    then the set of observable behaviors of $P'$ is a subset of $P$.
\end{corollary}

\begin{proof}

    We first consider the programs with just one iteration of the loop; Candidates of the form $C^1$, having just $e^1$. 
    We need to ensure that the resultant candidates $C^{1}'$ such that 
    \begin{align*}
        \forall k \in K, \ \reln{e}{ao}{k}.
    \end{align*}  
    has observable behaviors as a subset of $C^1$

    For every $k \in K$ such that $\reln{k}{ao}{e}$ we need them to be reorderable with respect to $e$ to bring it outside the loop.
    Using Condition \ref{licmr1:eq2}, we have from Corollary \ref{CorollCodeMotion1} that $C^{1}'$ has observable behaviors as a subset of $C^1$.
    To extend this to the program level with one iteration, we also need that $e$ should not be in any conditional branch.
    Using Condition \ref{licmr1:eq3}, we have from Prop \ref{CondB1} that $e$ is not part of any conditional branch.
    Thus, from Corollary \ref{ReordCond}, we can infer that the transformed program has observable behaviors a subset of the original.  
    
    Next, we consider the program with more than one iteration of the loop. 
    We prove this case using induction on the number of reads $e$ that exist due to multiple iterations of the loop. 
    \begin{itemize}

        \item Base case : number of $e$ = 2
    
        This case corresponds to candidates of the form $C^2$, thus giving us possibly two reads $e^1$ and $e^2$.
        We need candidate $C^2'$ with just one such event $e$, such that:
        \begin{align*}
            \forall k \in K, \ \reln{e}{ao}{k}.
        \end{align*}
        We first want tp reorder both the reads $e^1$, $e^2$ to be outside the loop, naming it Candidate $C^{2}''$.
        Using Condition \ref{licmr1:eq2}, by Corollary \ref{CorollCodeMotion1}, we can infer that $C^{2}''$ has observable behaviors as a subset of $C^2$. 
        From Condition \ref{licmr1:eq1}, by Theorem \ref{ReadElim}, we can eliminate either $e^1$ or $e^2$, thus resulting in $C^2'$ whose observable behaviors is a subset of $C^{2}''$.
        By transitive property of subsets we can infer that $C^2'$ has observable behaviors as a subset of $C^2$.
        %Using Condition \ref{licmr1:eq3}, we can infer that neither $e^1$ nor $e^2$ is part of a conditional.
        %Thus, by Corollary \ref{ReordCond}, we can infer that the transformed program has observable behaviors as a subset of the original.
        
        \item Inductive case : number of $e$ = $n$

        Assume that for all such candidates with $n$ iterations of the loop, the observable behaviors of $C^{n}'$ is a subset of $C^n$.

        We now prove using this that it can also hold for number $e$ as $n + 1$. 
        This case corresponds to candidates of the form $C^{n+1}$, thus giving us $n+1$ reads $e^1, e^2,...,e^{n+1}$.
        From Condition \ref{licmr1:eq2}, we can infer 
        \begin{align*}
            \forall k \ \text{s.t.} \ \reln{e^n}{ao}{k} \wedge \reln{k}{ao}{e^{n+1}}, \ Reord(k,e^{n+1}).
        \end{align*}
        By Corollary \ref{CorollCodeMotion1}, we can infer that $C''^{n+1}$ with $cons(e^n, e^{n+1})$ has observable behaviors as a subset of $C^{n+1}$. 
        From Condition \ref{licmr1:eq1}, by Theorem \ref{ReadElim}, we can eliminate $e^{n+1}$, thus giving us candidate of the form $C^n$ whose observable behaviors is a subset of $C^{n+1}''$.

        From our inductive assumption, we can then conclude that $C^{n+1}'$ has observable behaviors as a subset of $C^n$. 
        By transitive property of subsets, we can infer that $C^{n+1}'$ has observable behaviors as a subset of $C^{n+1}$.

    \end{itemize}

    From Condition \ref{licmr1:eq3}, by Corollary \ref{ReordCond}, we can infer that the observable behaviors of $P'$ is a subset of $P$.
    
\end{proof}

%-----------------------------------------------------------------------------------------------------------------------------------------

\begin{corollary}
    \label{LoopInvCodeMotRead2}
    Consider $K$ to be the set of events within a loop in program $P$. 
    Consider $e$ to be a read within the loop, not being a conditional check. 
    Consider program $P'$ with event $e$ agent ordered after the loop. 
    If
    \begin{gather}
        \et{e}{uo}. \label{licmr2:eq1}\\
        \forall k \neq e \in K, \ Reord(e, k). \label{licmr2:eq2}\\ 
        \nexists C^i \in P \ \text{s.t.} \ e^{j<=i} \notin C.  \label{licmr2:eq3}                    
    \end{gather}
    then the set of observable behaviors of $P'$ is a subset of $P$.
\end{corollary}

\begin{proof}

    We first consider the program with just one iteration. 
    Hence for Candidate $C^1$, we have just $e^1$. 
    We need to ensure that the resultant candidate $C^{1}'$ such that 
    \begin{align*}
        \forall k \in K, \ \reln{k}{ao}{e}.
    \end{align*}  
    has observable behaviors as a subset of $C^1$

    For every $k \in K$ such that $\reln{e}{ao}{k}$ we need them to be reorderable with respect to $e$ to bring it outside the loop.
    Using Condition \ref{licmr1:eq2}, we have from Corollary \ref{CorollCodeMotion1} that $C^{1}'$ has observable behaviors as a subset of $C^1$.
    To extend this to the program level with one iteration, we also need that $e$ should not be in any conditional branch.
    Using Condition \ref{licmr1:eq3}, we have from Prop \ref{CondB1} that $e$ is not part of any conditional branch.
    Thus, from Corollary \ref{ReordCond}, we can infer that the transformed program has observable behaviors a subset of the original.  
    
    Next, we consider the program with more than one iteration of the loop. 
    We prove this case using induction on the number of reads $e$ that exist due to multiple iterations of the loop. 
    \begin{itemize}
        
        \item Base case : number of $e$ = 2
    
        This case corresponds to candidates of the form $C^2$, thus giving us possibly two reads $e^1$ and $e^2$.
        We need candidate $C'^2$ with just one such event $e$, such that:
        \begin{align*}
            \forall k \in K, \ \reln{k}{ao}{e}.
        \end{align*} 
        We first reorder both the reads $e^1$, $e^2$ to be outside the loop, naming it Candidate $C''^2$.
        Because we have Condition \ref{licmr2:eq2}, by Corollary \ref{CorollCodeMotion1}, we can infer that $C''^2$ has observable behaviors as a subset of $C^2$.
        To go from $C^{2}''$ to $C^2'$, note that in $C^{2}''$ we have $cons(e^1, e^2)$ after reordering them. 
        From Condition \ref{licmr2:eq1}, by Theorem \ref{ReadElim}, we can eliminate either $e^1$ or $e^2$, thus resulting in $C^2'$ whose observable behaviors is a subset of $C^{2}''$.
        
        By transitive property of subsets we can infer that $C^2'$ has observable behaviors as a subset of $C^2$.
        
        \item Inductive case : number of $e$ = n

        Assume that for all such candidates with $n$ iterations of the loop, the observable behaviors of $C'^n$ is a subset of $C^n$.

        We now prove using this that it can also hold for number $e$ as $n + 1$. 
        This case corresponds to candidates of the form $C^{n+1}$, thus giving us $n+1$ reads $e^1, e^2,...,e^{n+1}$.
        From Condition \ref{licmr1:eq2}, we can infer
        \begin{align*}
            \forall k \ \text{s.t.} \ \reln{e^n}{ao}{k} \wedge \reln{k}{ao}{e^{n+1}}, \ Reord(e^{n}, k)
        \end{align*}
        By Corollary \ref{CorollCodeMotion1}, we can infer that $C''^{n+1}$ with $cons(e^n, e^{n+1})$ has observable behaviors as a subset of $C^{n+1}$. 
        From Condition \ref{licmr2:eq1}, by Theorem \ref{ReadElim}, we can eliminate $e^{n+1}$, thus giving us candidate of the form $C^n$ whose observable behaviors is a subset of $C^{n+1}''$.
        From our inductive assumption, we can then conclude that $C^{n+1}'$ has observable behaviors as a subset of $C^n$. 
        By transitive property of subsets, we can infer that $C^{n+1}'$ has observable behaviors as a subset of $C^{n+1}$.

    \end{itemize}

    From Condition \ref{licmr2:eq3}, by Corollary \ref{ReordCond}, we can infer that the observable behaviors of $P'$ is a subset of $P$.
    
\end{proof}