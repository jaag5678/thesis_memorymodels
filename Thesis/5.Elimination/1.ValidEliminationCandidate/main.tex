\section{Valid Elimination at the Candidate level}

    For addressing the validity of eliminations under our memory model, we separately address first Read elimination, followed by Write elimination, both at the candidate level. 
    At the Candidate level, eliminating an event would imply removal of certain ordering relations at the Candidate Execution level. 
    Given an event $e$ belonging to a Candidate $C$, the candidate $C'$ after eliminating event $e$ from it would imply no relations of the form 
    \begin{align*}
        \reln{k}{ao}{e} \\  
        \reln{e}{ao}{k}.
    \end{align*}
    and no relations of the form
    \begin{align*}
        \reln{k}{hb}{e} \\
        \reln{e}{hb}{k}.
    \end{align*}
    exist in any Candidate Execution of $C'$.
    
    %Read Elimination 
    \subsection{Elimination of Reads}

        The following theorem establishes the condition when we can eliminate a read from a Candidate, while still ensuring that the result Candidate has observable behaviors as a subset. 
            %Read Elimination
    \begin{theorem}
    \label{WriteElim}
    Consider a candidate $C$ of a program and its possible \textit{Candidate Executions} where $\stck{_\textit{hb}}$ is strictly partial order. 
    Consider two \textbf{write} events $e$ and $d$ in $C$ such that 
    \begin{align*}
        \cons{e}{d} \ \wedge \ \reln{e}{ao}{d}. 
    \end{align*}
    Consider a Candidate $C'$ after eliminating the event $e$ from $C$.  
    If
    \begin{align*}
        \et{e}{uo} \ \wedge \ \Re(e) = \Re(d). 
    \end{align*}
    then the set of Observable behaviors of $C'$ is a subset of $C$.  
\end{theorem}

    \begin{proof}
    Once again, we look at this as a write elimination done on a Candidate Execution of $C$. We start by proving when other happens-before relations remain intact. Followed by identifying relations lost due to elimination and a proof for when these relations do not introduce new observable behaviors. 
    
   \paragraph{1. Preserving \emph{happens-before} relations}
        The relations we want to preserve are those that are dervied through relation with $e$, viz. using the following two relations:
        \begin{tasks}(2)
            \task $\reln{k}{hb}{e}$
            \task $\reln{e}{hb}{k}$
        \end{tasks}

        We can divide the events involved in the above into two sets:
        \begin{align*}
            K_b = \{k \ | \ \reln{k}{hb}{e} \}. \\
            K_a = \{k \ | \ \reln{e}{hb}{k} \}. 
        \end{align*}

        We need to ensure the following relations hold after elimination.
        \begin{align*}
            \forall k_a \in K_a \ \wedge \ \forall k_b \in K_b \ . \ \reln{k_b}{hb}{k_a}
        \end{align*}

        Similar to reordering, we need to have a valid pivot pair $<p_b, p_a>$ such that 
        \begin{align*}
            \forall k_b \neq p_b \in K_b \ . \ \reln{k_b}{hb}{p_b} \\
            \forall k_a \neq p_a \in K_a \ . \ \reln{p_a}{hb}{k_a} 
        \end{align*}

        By Lemma \ref{Lemma1}, $\et{e}{uo}$ is the only case where $p_b$ can be a valid pivot. 
        By Lemma \ref{Lemma2}, $\et{e}{uo} \ \vee \ \et{e}{sc}$ are the cases where $p_a$ can be a valid pivot. 
        We need both the above conditions to be satisfied to have a valid pivot pair. 
        Hence, $\et{e}{uo}$ is the only possibility in which a valid pivot pair can exist. 

        \critic{blue}{Put a figure here to show this pivot role.}
   
    \paragraph{2. The \emph{happens-before} relations lost}

    The relations lost are those attached to the event $e$, which are: 
    \begin{align}
        \reln{k}{hb}{e} \ \vee \ \reln{e}{hb}{k}
    \end{align}
    
    \critic{red}{Do we need to prove that these are the only relations lost? Proof part 1 implicitly shows this.}

   \paragraph{3. Presence of Cycles?}
        
Because no new $\stck{_{hb}}$ relations are introduced, and because original candidate executions have $\stck{_{hb}}$ as a strict partial order, no cycles are introduced after elimination. 

\critic{blue}{Perhaps write this argument a bit better.}

   \paragraph{4. Do the lost relations result in New Observable Behaviors?}

        To answer this, we need to see whether the relations removed had an impact on possible $\stck{_{rf}}$ relations other than those with $e$. 
        We divide our argument into two parts, viz. the two types of relations removed:
        \begin{tasks}(2)
            \task $\reln{k}{hb}{R_{uo}}$. 
            \task $\reln{R_{uo}}{hb}{k}$.
        \end{tasks}

        Figure~\ref{elim_read:case1} shows a breakdown of sub-cases for case (a), varying based
        on the nature of event $k$.
        \begin{figure}[H]
            \centering
            \includegraphics[scale=0.5]{5.Elimination/1.ValidEliminationCandidate/ReadElimProof/ProofParts/Part4_Case1.pdf}
            \caption{The impact of lost relation $\reln{k}{hb}{R_{uo}}$ on observable behaviors.}
            \label{elim_read:case1}
        \end{figure}

        Observations:
        \begin{itemize}
            \item (i) is not a pattern forbidden by the consistency rules.
            \item (ii)(a) is a pattern of Axiom \ref{CoRe}, however, only restricting $\stck{_{rf}}$ relation with $R$ and $W'$(which here is our Unordered Read)
            \item (ii)(b) is a pattern of Axiom \ref{SeqCsAt}, however, once again, only restricting $\stck{_{rf}}$ relation with $R$ and $W'$. 
        \end{itemize}

        Figure~\ref{elim_read:case2} shows a breakdown of sub-cases for case (b), varying based
        on the nature of event $k$.
        \begin{figure}[H]
            \centering
            \includegraphics[scale=0.5]{5.Elimination/1.ValidEliminationCandidate/ReadElimProof/ProofParts/Part4_Case2.pdf}
            \caption{The impact of lost relation $\reln{R_{uo}}{hb}{k}$ on observable behaviors.}
            \label{elim_read:case2}
        \end{figure}

        Observations:
        \begin{itemize}
            \item (i) is not a pattern in any Consistency rules
            \item (ii) is a pattern of Axiom \ref{CoRe}, however, only restricting $\stck{_{rf}}$ relation with $R$ and $W$
        \end{itemize}

        From the above observations, we can infer that the relations removed only have restriction on reads-from relations on the event $e$ we eliminate. 
        Thus, we can conclude that no new observable behaviors are introduced due to the removed $\stck{_{hb}}$ relations. 

\end{proof}


    %Write elimination
    \subsection{Elimination of Writes}

        The following theorem establishes the condition when we can eliminate a write from a Candidate, given that we have another write consecutive to it.  
            %Read Elimination
    \begin{theorem}
    \label{WriteElim}
    Consider a candidate $C$ of a program and its possible \textit{Candidate Executions} where $\stck{_\textit{hb}}$ is strictly partial order. 
    Consider two \textbf{write} events $e$ and $d$ in $C$ such that 
    \begin{align*}
        \cons{e}{d} \ \wedge \ \reln{e}{ao}{d}. 
    \end{align*}
    Consider a Candidate $C'$ after eliminating the event $e$ from $C$.  
    If
    \begin{align*}
        \et{e}{uo} \ \wedge \ \Re(e) = \Re(d). 
    \end{align*}
    then the set of Observable behaviors of $C'$ is a subset of $C$.  
\end{theorem}

    \begin{proof}
    Once again, we look at this as a write elimination done on a Candidate Execution of $C$. We start by proving when other happens-before relations remain intact. Followed by identifying relations lost due to elimination and a proof for when these relations do not introduce new observable behaviors. 
    
   \paragraph{1. Preserving \emph{happens-before} relations}
        The relations we want to preserve are those that are dervied through relation with $e$, viz. using the following two relations:
        \begin{tasks}(2)
            \task $\reln{k}{hb}{e}$
            \task $\reln{e}{hb}{k}$
        \end{tasks}

        We can divide the events involved in the above into two sets:
        \begin{align*}
            K_b = \{k \ | \ \reln{k}{hb}{e} \}. \\
            K_a = \{k \ | \ \reln{e}{hb}{k} \}. 
        \end{align*}

        We need to ensure the following relations hold after elimination.
        \begin{align*}
            \forall k_a \in K_a \ \wedge \ \forall k_b \in K_b \ . \ \reln{k_b}{hb}{k_a}
        \end{align*}

        Similar to reordering, we need to have a valid pivot pair $<p_b, p_a>$ such that 
        \begin{align*}
            \forall k_b \neq p_b \in K_b \ . \ \reln{k_b}{hb}{p_b} \\
            \forall k_a \neq p_a \in K_a \ . \ \reln{p_a}{hb}{k_a} 
        \end{align*}

        By Lemma \ref{Lemma1}, $\et{e}{uo}$ is the only case where $p_b$ can be a valid pivot. 
        By Lemma \ref{Lemma2}, $\et{e}{uo} \ \vee \ \et{e}{sc}$ are the cases where $p_a$ can be a valid pivot. 
        We need both the above conditions to be satisfied to have a valid pivot pair. 
        Hence, $\et{e}{uo}$ is the only possibility in which a valid pivot pair can exist. 

        \critic{blue}{Put a figure here to show this pivot role.}
   
    \paragraph{2. The \emph{happens-before} relations lost}

    The relations lost are those attached to the event $e$, which are: 
    \begin{align}
        \reln{k}{hb}{e} \ \vee \ \reln{e}{hb}{k}
    \end{align}
    
    \critic{red}{Do we need to prove that these are the only relations lost? Proof part 1 implicitly shows this.}

   \paragraph{3. Presence of Cycles?}
        
Because no new $\stck{_{hb}}$ relations are introduced, and because original candidate executions have $\stck{_{hb}}$ as a strict partial order, no cycles are introduced after elimination. 

\critic{blue}{Perhaps write this argument a bit better.}

   \paragraph{4. Do the lost relations result in New Observable Behaviors?}

        To answer this, we need to see whether the relations removed had an impact on possible $\stck{_{rf}}$ relations other than those with $e$. 
        We divide our argument into two parts, viz. the two types of relations removed:
        \begin{tasks}(2)
            \task $\reln{k}{hb}{R_{uo}}$. 
            \task $\reln{R_{uo}}{hb}{k}$.
        \end{tasks}

        Figure~\ref{elim_read:case1} shows a breakdown of sub-cases for case (a), varying based
        on the nature of event $k$.
        \begin{figure}[H]
            \centering
            \includegraphics[scale=0.5]{5.Elimination/1.ValidEliminationCandidate/ReadElimProof/ProofParts/Part4_Case1.pdf}
            \caption{The impact of lost relation $\reln{k}{hb}{R_{uo}}$ on observable behaviors.}
            \label{elim_read:case1}
        \end{figure}

        Observations:
        \begin{itemize}
            \item (i) is not a pattern forbidden by the consistency rules.
            \item (ii)(a) is a pattern of Axiom \ref{CoRe}, however, only restricting $\stck{_{rf}}$ relation with $R$ and $W'$(which here is our Unordered Read)
            \item (ii)(b) is a pattern of Axiom \ref{SeqCsAt}, however, once again, only restricting $\stck{_{rf}}$ relation with $R$ and $W'$. 
        \end{itemize}

        Figure~\ref{elim_read:case2} shows a breakdown of sub-cases for case (b), varying based
        on the nature of event $k$.
        \begin{figure}[H]
            \centering
            \includegraphics[scale=0.5]{5.Elimination/1.ValidEliminationCandidate/ReadElimProof/ProofParts/Part4_Case2.pdf}
            \caption{The impact of lost relation $\reln{R_{uo}}{hb}{k}$ on observable behaviors.}
            \label{elim_read:case2}
        \end{figure}

        Observations:
        \begin{itemize}
            \item (i) is not a pattern in any Consistency rules
            \item (ii) is a pattern of Axiom \ref{CoRe}, however, only restricting $\stck{_{rf}}$ relation with $R$ and $W$
        \end{itemize}

        From the above observations, we can infer that the relations removed only have restriction on reads-from relations on the event $e$ we eliminate. 
        Thus, we can conclude that no new observable behaviors are introduced due to the removed $\stck{_{hb}}$ relations. 

\end{proof}

    