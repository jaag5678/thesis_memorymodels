\paragraph{4. Do the lost relations result in New Observable Behaviors?}

    To address this, we divide our cases into two parts; one for each type of relation lost:
    \begin{tasks}[style=enumerate](2)
        \task $\reln{k}{hb}{W_{uo}}.$
        \task $\reln{W_{uo}}{hb}{k}.$
    \end{tasks}

    Figure~\ref{elim_write:case1} shows an exhaustive breakdown of sub-cases for case (1), varying based
    on the nature of event $k$.
    \begin{figure}[H]
        \centering
        \includegraphics[scale=0.5]{5.Elimination/1.ValidEliminationCandidate/WriteElimProof/ProofParts/Part4Case1.pdf}
        \caption{The impact of lost relation $\reln{k}{hb}{W_{uo}}$ on observable behaviors.}
        \label{elim_write:case1}
    \end{figure}

    We can observe the following:
    \begin{itemize}
        \item (i) is a pattern from Axiom \ref{CoRe} that restricts the read $R$ reading from $e$. This reads-from restriction will remain the same after elimination of $e$. 
        \item (ii)(a) is a pattern from Axiom \ref{CoRe}, forbidding $R$ to read some bytes of $W'$. 
        The set of byte-level restrictions will remain the same after elimination, if firstly we have the two relations
        \begin{align*}
            \reln{d}{hb}{R} \wedge \reln{W'}{hb}{d}.
        \end{align*}
        Secondly, we need to ensure that after elimination, Axiom \ref{CoRe} now restricts the exact set of $\stck{_{rbf}}$ relations with $W'$ and $R$ as before. 
        By Lemma \ref{Lemma2} and by Def of happens-before, the first condition holds. 
        For the second, since $R$ or $W'$ can be arbitrary events with arbitrary ranges, we would need the ranges of $e$ and $d$ to be same (i.e $\Re(e) = \Re(d)$).
    \end{itemize}

    Thus, for the case (1), we can conclude that the removed relations do not introduce any new observable behaviors, while also requiring the constraint on the range of events $e$ and $d$.

    Figure~\ref{elim_write:case2} shows an exhaustive breakdown of sub-cases for case (2), varying based
    on the nature of event $k$.
    \begin{figure}[H]
        \centering
        \includegraphics[scale=0.6]{5.Elimination/1.ValidEliminationCandidate/WriteElimProof/ProofParts/Part4Case2.pdf}
        \caption{The impact of lost relation $\reln{W_{uo}}{hb}{k}$ on observable behaviors.}
        \label{elim_write:case2}
    \end{figure}

    We make the following observations:
    \begin{itemize}
        \item (i)(a) has a similar argument to the case (1)'s (ii)(a), requiring $e$ and $d$ to have equal ranges.
        \item (i)(b) is a pattern from Axiom \ref{SeqCsAt}, which restricts $R$ from reading anything of $W$. This reads-from restriction will remain the same after $e$ is eliminated. 
        \item (ii)(a) is a pattern from Axiom \ref{CoRe}, restricting $R$ from reading $W$. This reads-from restriction will remain the same after after eliminating $e$.
        \item (ii)(b) is the same as (i)(b).
    \end{itemize}

    Thus, for the case (2), we can conclude that the removed relations do not introduce any new observable behaviors.

    In all the above cases, we observe that on keeping range of $e$ and $d$ equal, none of the patterns introduce any new observable behavior. Hence, if we have two consecutive writes of equal ranges, of which the first one has access mode unordered, the set of Observable Behaviors without the write is a subset of that with it present. 