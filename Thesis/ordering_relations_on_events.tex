%Ordering Relation among Events----------------------------------------------------------------------------------------------------------------------       
        \section{Ordering Relations among Events}
        
        %Agent Order
        \paragraph{Agent Order ($\stck{_{ao}}$)}
            An relation on all events. It respects the order in which the events are evaluated by a particular agent. 
            \[\forall{i, j > 0},\ e^i_j \stck{_{ao}} e^i_{j+1} \]
        
            \begin{enumerate}
                \item Agent order also respects transitivity. 
            
                    \[ (e \stck{_{ao}} d) \wedge (d \stck{_{ao}} g) \Rightarrow{} (e \stck{_{ao}} g)\]
            
            \end{enumerate}
            
            \critic{blue}{Note that the relations are only with respect to events belonging to the same agent. A collection of such relations together form the agent order. This is analogous and meant to be equivalent to what we call as intra-thread sequential order. It is the same as what \textbf{sequenced-before} is defined to be in C++}
        
        %Synchronize With Order
        \paragraph{Synchronize-With Order ($\stck{_{sw}} $)}
            An ordering relation that represents the synchronizations among different agents.
            \begin{enumerate}
                \item All pairs belonging to $ASW$ of every agent belongs to this ordering relation. 
            
                        \[\forall{i, j > 0}, \ \langle e_i, e_j \rangle \in ASW \Rightarrow{} e_i \stck{_{sw}} e_j \]
            
                \item Specific reads-from pairs also belong to this ordering relation. 
            
                        \[(r \stck{_{rf}} w) \ \wedge \ \et{r}{sc} \ \wedge \ \et{w}{sc} \ \wedge \ (\Re(r)\!=\!\Re(w)) \ \Rightarrow{} \ (w \stck{_{sw}} r)\]
            
            \end{enumerate}
            
            \critic{blue} {Note that for the second condition, both ranges of events have to be equal. However, there is no restriction that the read event cannot read-from other write events.}
            
        %Happens Before order 
        \paragraph{Happens Before Order ($\stck{_{hb}}$)}
             A \textit{partial order} on events which are composed of the following:
            
            \begin{enumerate}
                \item Every agent-ordered relation is also a happens-before relation 
              
                    \[(e \stck{_{ao}} d) \ \Rightarrow{} \ (e \stck{_{hb}} d)\]
              
                \item Every synchronize-with relation is also a happens-before relation 
              
                    \[(e \stck{_{sw}} d) \ \Rightarrow{} \ (e \stck{_{hb}} d)\]
                     
                \item Initialize type of events happen before all shared memory events that have overlapping ranges with them. 
                
                    \[
                        \forall e,d \in SM \ \wedge \ 
                        \et{e}{init} \ \wedge \ 
                        (\Re(e) \cap \Re(d) \neq \phi)
                        \ \Rightarrow{} \ 
                        e \stck{_{hb}} d
                    \]
            \end{enumerate}
        
        \critic{red}{It is also important to note that those $\stck{_{hb}}$ relations that are formed due to Sequentially Consistent events (read-write), imply a more stronger visibility guarantee, in that all the threads observe the same global total order of such events. This however, is not expressed using this relation. Perhaps a better way to represent it may be required.}
        
        %Memory Order
        \paragraph{Memory Order ($\stck{_{mo}}$)}
            This order is a \textit{total order} on all events that are evaluated in an agent cluster. 
            \begin{enumerate}
                \item Happens before order is a part of memory order
                    \[(e \stck{_{hb}} d) \Rightarrow{} (e \stck{_{mo}} d)\]
            \end{enumerate}