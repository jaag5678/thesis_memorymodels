%AXIOMATIC MODEL DEMYSTIFIED----------------------------------------------

\documentclass[10pt]{article}
\usepackage[utf8]{inputenc}
\usepackage{amsmath}
\usepackage{amssymb}
\usepackage[a4paper,margin=1in,footskip=0.25in]{geometry}
\usepackage[dvipsnames]{xcolor}
\usepackage{graphicx}
\usepackage{tikz}
\usepackage{amsthm}
\usepackage{float}
\usepackage{centernot}
\usepackage{tasks}

\title{ECMAScript Axiomatic Memory Consistency Model}
\author{Akshay Gopalakrishnan}
\date{November 2019}

\begin{document}

\maketitle


%Short form to use stack_relative
\newcommand{\stck}{\stackrel{\longrightarrow}}

%A different version of the above 
\newcommand{\stckdet}[1]{\stackrel{{#1}}}

%We write a lot of relations between two events using the orderings, so a short form to use that
\newcommand{\reln}[3]{#1\stck{_{#2}}#3}

%We use a more detailed version of the above relation to indicate direct / indirect relations 

\newcommand{\reldet}[4]{#1\stckdet{_{#2}}{{\stck{_{#3}}}}#4}

%We also will introduce a short form to write an event belongs to some set
\newcommand{\event}[2]{#1\!\in\!#2}

%To make events and their type more close to each other
\newcommand{\typ}[1]{\textit{#1}}
\newcommand{\et}[2]{#1\!:\!\typ{#2}}

%Short form to write color text
\newcommand{\critic}[2]{\textcolor{#1}{\footnotesize #2}}

%Some preliminary latex commands to format writing theorems 

\newtheorem{lemma}{Lemma}

\newtheorem{theorem}{Theorem}[lemma]

\newtheorem{corollary}{Corollary}[theorem]

\newtheorem{definition}{Definition}

%COMMENT FOR NOW AS IT IS NOT OUR PRIORITY 
%
\section{Introduction}
    Instruction reordering is a common operation done by the compiler / hardware for optimization, essential to instruction scheduling of course, but also implicit in loop invariant removal, partial redundancy elimination, and other optimizations that may move instructions. 
    However, whether we can do such reordering freely given a concurrent program using relaxed memory accesses is a bit unclear. 
     
    \paragraph{Simple reordering is not straightforward under shared memory semantics}
    The main reason is that memory accesses here, do not just perform the desired operation (i.e Read / Write) but also imply certain visibility guarantees across all the other threads.  
    In our observation, we find that, the relaxed memory model of Javascript prescribe semantics for visibility using the $\stck{_{hb}}$ relations. 
    
    \paragraph{Some Examples}

        We show a couple of examples to showcase why reordering may not be that straightforward. 

        Consider the first example in Figure~\ref{reord:example1(a)} below of a Candidate and the resultant candidate after reordering two events.
        The figure on the left is the original candidate and that on the right is after reordering the two reads of $T2$.
        The observable behavior in question is written in the middle. 
        \begin{figure}[H]
            \centering
            \includegraphics[scale=0.7]{5.InstructionReordering/0.Intro/ReorderingExample1(a).pdf}
            \caption{First example for reordering with candidates of the original program and its reordered counterpart.}
            \label{reord:example1(a)} 
        \end{figure}
        
        Figure~\ref{reord:example1(b)} has two sets of relations. 
        The first justifies the outcome for the reordered candidate. 
        While the second justifies the original candidate. 
        Notice that for the second, one may have a read memory ordered before a write that it reads from. 
        This is quite counter intuitive to understand at first. 
        But strictly from the semantics of the model, this justification of the observable behavior is completely valid. 
        \begin{figure}[H]
            \centering
            \includegraphics[scale=0.7]{5.InstructionReordering/0.Intro/ReorderingExample1(b).pdf}
            \caption{The set of partial order relations justifying the observable behavior in question for both the candidates in Figure~\ref{reord:example1(a)}.} 
            \label{reord:example1(b)}
        \end{figure}

        
        Consider another example in Figure~\ref{reord:example2(a)}.
        The figure on the left is the original candidate and that on the right is after reordering the two events of $T1$.
        The observable behavior in question is written in the middle. 
        \begin{figure}[H]
            \centering
            \includegraphics[scale=0.7]{5.InstructionReordering/0.Intro/ReorderingExample2(a).pdf}
            \caption{Second example for reordering with candidates of the original program and its reordered counterpart.} 
            \label{reord:example2(a)}
        \end{figure}

        
        Figure~\ref{reord:example2(b)} has two sets of relations. 
        The first justifies that such an outcome is not possible for the original program candidate due to Axiom \ref{CoRe}. 
        While the second justifies that this outcome is possible for the reordered program.
        Note that we cannnot infer in the reordered candidate the set of relations for any candidate execution to have $\reln{a=x;_{uo}}{hb}{x=1;_{uo}}$. 
        \begin{figure}[H]
            \centering
            \includegraphics[scale=0.7]{5.InstructionReordering/0.Intro/ReorderingExample2(b).pdf}
            \caption{The set of partial order relations justifying the observable behavior in question for both the candidates in Figure~\ref{reord:example2(a)}.} 
            \label{reord:example2(b)}
        \end{figure}

        The above two examples show that we have to be careful while reordering two events in the same thread. 
        By example case analysis, for each observable behavior, one must check all possible candidate executions and assert whether such an observable is possible or not. 
        This method of checking validity of reordering will scale exponentially as the program size increases. 
        It is often also the case that the compiler may not have information on which exact events would be executed in other threads to assert such reordering is valid or not. 

    
    
    
    
    


%The ECMAScript Memory Model

%AGENTS----------------------------------------------------------------------------------------------------------------------------------------  
\section{Agents, Events and their Types}

    \subsection{Agents}
        A concurrent program involves different threads/processes running concurrently. 
        Agents could be thought analogous to different threads/processes. 
        
        \critic{red}{Agents actually have more meaning than what we refer to here. However, in terms of reasoning just with memory consistency, we are safely abstracting them to just represent threads/processes.}
        
        \critic{blue}{Technically, one may not map them directly to individual threads as from an implementation standpoint, a single thread can be allowed to execute multiple concurrent processes. As a way of separating implementation from the specifications, we refer to them as Agents}

        %Agent Clusters
        \paragraph{Agent Cluster ($AC$)}
        Collection of agents running concurrently communicating with each other (directly/ indirectly) form an agent cluster.  There can be multiple agent clusters. However, an agent can only belong to one agent cluster.
        
        %PErhaps give an example here later

        \critic{blue}{Note that for the purpose of reasoning with memory model, we stick to assuming that just one agent cluster exists. So we will refrain from defining a function mapping an agent to its respective agent cluster. We also assume that agents in the cluster communicate only through one common shared memory segment.}
        
        %Agent Event Set
        \paragraph{Agent Event List $(ael)$}
        Every agent is mapped to a list of events appended to it during evaluation. We define $ael$ is a mapping of each agent to a list of events.
        
            \[ael(a) = [e_1, e_2, ... e_k ] \]
        
        \critic{blue}{The standard refers this to be an Event List, but we find it a bit misleading as it does not signify a list for each agent. Hence we name it as Agent Event List}
        
        %Ask whether this is actually required as a notation down the line
        \paragraph{}
            When referring to events in an agent cluster, we use the following notation for an event:
        
            \[ e^i_j \ \longrightarrow \  j^{th}\ \textit{instruction}\ \textit{of}\ i^{th} \textit{Agent} \]  
            
        \critic{blue}{We will sometimes forgo the subscript or superscript wherever it may not play a role in understanding a relation or definition.} 
        
        

        

            



%Events------------------------------------------------------------------
    \subsection{Events}
        
        An evaluation of an operation results in a set of events that are evaluated. An event is either an operation that involves (shared) memory access or that constrains the order of execution of multiple events. The latter are called \textit{Synchronize Events}

        \critic{blue}{Synchronizing events are analogous to $lock$ and $unlock$ events that allow exclusive access to critical sections of memory.}  

%Events------------------------------------------------------------------
\subsection{Events}
        
The memory model is described mainly using a set of events and some ordering relations on them. An evaluation of an operation results in a set of events that are evaluated. An event is either an operation that involves (shared) memory access or that constrains the order of execution of multiple events. The latter are called \textit{Synchronize Events}

\critic{blue}{Synchronizing events are analogous to $lock$ and $unlock$ events that allow exclusive access to critical sections of memory. However, this is not specified in the standard as part of the memory model.}

%Event_sets----------------------------------------------------------------------------------------------------------------------------------
    
    %Useful command syntax
    \newcommand{\rmw}{\textit{rmw}\,}
    \newcommand{\set}[1]{\textbf{\textit{#1}}}

    \subsection{Event Set}
    Given an agent cluster, an \textit{event set} is a collection of all events from the agent event lists. This set is composed of mainly two distinct subsets as follows: 
       
        %Shared Memory Events
        \subsubsection{Shared Memory (\set{SM}) Events} This set is composed of two sets of events: 
            
            \critic{purple}{Use a better listing to enumerate both items below in the same line}
            \begin{enumerate}
                \item Write events (\set{W})
                \item Read events (\set{R}) 
            \end{enumerate}
            Events that belong to both Write and Read events are called Read-Modify-Write. 
        %Synchronize events 
        \subsubsection{Synchronize (\set{S}) Events} These events only restrict the ordering of execution of events by agents. They are of two sets, which are mutually exclusive:
            \begin{enumerate}
                \item Lock events (\set{L})
                \item Unlock events (\set{U}) 
            \end{enumerate}
            
        \critic{blue}{The features of $Lock$ and $Unlock$ events is actually not something given to the programmer to use in Javascript. They are used to implement the feature $wait$ and  $notify$ that the programmer can use which adhere to the semantics of $futexes$ in Linux. Hence, in the original standard of the model, the distinction between lock and unlock is not made, and it is simply stated as Synchronize Event}
 
    \critic{blue}{There is an additional set of events called Host Specific Events, but for our purpose, it is not of any major concern.}  
        
    %Range of events
        \paragraph{Range ($\Re$)}
            Each of the \textit{shared memory events} are associated with a contiguous range of memory on which it operates. Range is a function that maps a shared memory event to the range it operates on. This we represent as a starting index $i$ and a size $s$. So we could represent the range of a write event $w$ as 
                    
                    \[\Re(w) = (i, s) \]
        
            \critic{red}{The range as per the ECMAScript standard denotes only the set of contiguous byte indices. The starting byte index is kept separate. We find this to be unnecessary. Hence we define range to have starting index and size.}
           
            Two Ranges can be \textit{disjoint}, \textit{overlapping} or \textit{equal}. We use the two operators below to define these three possibilities between ranges of events $e$ and $d$ :
            
            \begin{enumerate}
                \item Intersection $(\cap{_\Re})$ - Set of byte indices common to both ranges.
                \item Union $(\cup_\Re)$ - A unique set of byte indices that exist in both the ranges.  
            \end{enumerate}
            
            \begin{enumerate}
                \item Disjoint $\Re(e) \cap_\Re \Re(d) = \phi$ 
                \item Overlapping $(\Re(e)\cap_\Re \Re(d) \neq \phi) \wedge (\Re(e) \cap_\Re  \Re(d) \neq \Re(e) \cup_\Re \Re(d))$ - 
                \item Equal $\Re(e) \cap_\Re  \Re(d) = \Re(e) \cup_\Re \Re(d)$ - In simple terms, we define equality as $\Re(e) = \Re(d)$
            \end{enumerate}
            
            \critic{blue}{Note that two ranges being overlapping is different from them being equal. This distinction is used to define certain things ahead in the model.}
            
         \paragraph{Value($V$)}  
           It is a function that maps a byte address given to the value that is stored in that address.For example, the byte address $k \text{ has the value } x_k$ will be depicted as:
                
                \[V(k) = x_k\]
            
            \critic{red}{We introduce the value function to just map memory to values stored there. Note that we also assume only integer values for the sake of reasoning with memory models.}
            
            Using the above constructs, we represent the three subset of shared memory events with their ranges in the following way:
            
            Consider a chunk of memory {k,k+1...k+10} wherein the values stored are:
            
                \[\forall i \in [0,10], V(k+i) = x_{k+i}\]
                
            \begin{itemize}
                \item $w$ with range $(k,11)$ modifying memory to ${x'_{k}}...{x'_{k+10}}$ will be as : 
                
                        \[{W^i_j}[k...(k+10)] = \{x'_{k}, x'_{k+1}...x'_{k + 10}\}\]
                
                \item $r$ will be represented the same as write with a distinction in semantics that the right hand side is what is read from the range of memory 
                
                        \[{R^i_j}[k...(k+10)] = \{x_{k}, x_{k+1}...x_{k + 10}\}\]
                
                \item $\rmw$ will be mapped to two tuples, the left one indicating the values read and the right one indicating the values written to the same memory. 
                
                        \[{RMW^i_j}[k...(k+10)] = \{(x_{k}, x_{k+1}...x_{k + 10}), (x{'}_{k}, x{'}_{k+1}...x{'}_{k + 10}) \}\]
                
            \end{itemize}
            
            \critic{blue}{Note that some examples will also be like $R[0..4] = 10$, where 10 symbolizes the value stored in 32 bits of memory, which is ideally the form \{0, 0, 0, 32\}. This is because, we are taking decimal equivalent of a 32 bit binary number. It is important to note this fact.}


%Types of Events Based on Order--------------------------------------------------------------------------------------------------------------------
    
    \subsection{Types of events based on Order} 
        Order signifies the sequence in which event actions are visible to different agents as well as the order in which they are executed by the agents themselves. In our context, there are mainly three types for each shared memory event that tells us the kind of ordering that it respects. 
        
        \begin{enumerate}
            \item \textbf{Sequentially Consistent ($sc$)} - Events of this type are $atomic$ in nature. The meaning of sequentially consistent implies that there is a strict global total ordering of such events which is agreed upon by all concurrent processes sharing the same memory. 
            
            \item \textbf{Unordered ($uo$)} - Events of this type are considered non-atomic and can occur in different orders for each concurrent process, meaning there is no fixed global order respected by agents for such events. 
            
            \item \textbf{Initialize ($init$)} - Events of this type are used to initialize the values in memory before events in an agent cluster begin to execute concurrently. Additionally, only write events can be of this type and there is only one init event for each byte address in shared memory. 
        \end{enumerate}
        
        \paragraph{}
        We represent the type of events in the following format - $event : type$ 
        
        \critic{red}{The word \textit{atomic} is actually misleading. It does not imply the events are evaluated using just one instruction. For example, a 64-bit sequentially consistent write on a 32-bit system has to be done with two subsequent memory actions. But its intermediate state of write must not be seen by any other agent. In an abstract sense, this event must appear '\textit{atomic}'.The \textit{atomic} here also refers to implications of whether an event's consequence is visible to all other agents in the same global total order or not. The compiler must ensure that for specific hardware, such guarantees are satisfied.}
        
        \critic{red}{The notion of sequentially consistent has the same semantics of what C++ has for such events. Note that this semantics is not mentioned here explicitly, but by talking with fellow researchers working on the same domain, as well as with careful observation, it has come to our understanding that this is assumed to be true. We will make sure that the semantics is defined here properly as per what is there of C++.}
      
        \critic{red}{We are not sure if $init$ is a type of write that has a range as the range of shared memory involved in the agent cluster or is it individual writes for each byte address. This is not mentioned in the $standard$. We assume them to be for individual byte addresses, as we will see shortly in the rule for $\stck{_{hb}}$ ordering why we consider this assumption to be the best one.  But note that it may not be the case always. As an example, consider a 32-bit hardware not having instructions to support 16 or 8 bit memory actions. In that case we do not know what the $init$ event will sum up to. It would rather be better for it to be just one $init$ write event that ranges through the entire shared memory, and modify the $\stck{_{hb}}$ relation accordingly.}
      
%Tearing factor of events---------------------------------------------------------------------------------------------------------------------------

    \subsection{Tearing (Or not)}
        Additionally, each shared-memory event is also associated with whether they are tear-free operations or not.
        
        \paragraph{Tearing}
            Operations that tear are not aligned accesses with respect to the hardware and can be serviced using two or more memory fetches. 
            
        \paragraph{Tear-Free}
            Operations that are tear-free are aligned with respect to the hardware and should appear to 
            be serviced in one memory fetch. (this might not be possible always, but we are concerned with whether it can appear to be tear free)
     
        \critic{red}{There is a very confusing definition of \textit{tear-free ness} given by ECMAScript. These definitions are part of how the tear factor affects the behavior of programs in a concurrent setting. This is also defined as a set of axioms further below. We make this distinction to avoid confusion : 
        \begin{enumerate}
            \item For every Read event, tear-free-ness questions whether this event is allowed to read from multiple write events on equal range as this event
            \item For every Write event, tear-free-ness questions whether this event is allowed to be read by multiple reads on equal range as this event. 
        \end{enumerate}
        }
        
        For most of our analysis, unless otherwise stated we will assume all events to be tear-free
                       

%Relation among events----------------------------------------------------------------------------------------------------------------------------
    \section{Relation among events}
        There are three basic relations that assist us in reasoning about events and their interaction with memory.
        
        %Read bytes from relation 
        \paragraph{Read-Bytes-From $(\stck{_{rbf}})$}
        
        This relation maps every read event to a list of tuples consisting of write event and their corresponding byte index that is read. For instance, consider a read event $r[i...(i+3)]$ and corresponding write events $w_1[i...(i+3)]$, $w_2[i...(i+4)]$. One possible $\stck{_{rbf}}$ relation would be: \[r \stck{_{rbf}} \{(w_1, i), (w_2, i+1), (w_2, i+2)\}\]
        
        \critic{blue}{We will represent individual rbf relations with read events in our examples. So for the above example, the three rbf pairs are: \[r \stck{_{rbf}} (w_1, i),\ r \stck{_{rbf}} (w_2, i+1),\ r \stck{_{rbf}} (w_2, i+2)\]}
        
        %Reads from relation
        \paragraph{Reads-From $(\stck{_{rf}})$}
        
        This relation, is similar to the above relation, except that the byte index details are not involved in the composite list. So for the above example, the rf relation would be :  \[r \stck{_{rf}} \{w_1,w_2\}\]
        
        \critic{blue}{Similar to rbf, we also represent pair-wise relation in rf : \[r \stck{_{rf}} w_1,\ r \stck{_{rf}} w_2\]}
        
        %Agent sync with relation
        \paragraph{Agent-Synchronizes With (\set{ASW})}
        
        A list for each agent that consist of ordered tuples of synchronize events. These tuples specify ordering constraints among agents at different points of execution. So such a list for an agent $k$ would be represented like:  
        \[ASW_k = \{ \langle s_1, s_2 \rangle, \langle s_3, s_4 \rangle ...\}\]
        
        A property that must hold for each of these lists is: 
        \begin{itemize} 
            \item For every pair in the list, the second event belongs to the parent agent and the first belongs to another agent it synchronized with.
                \[  
                    \forall{i,j>0},\ 
                    \langle s^i, s^j\rangle \in ASW_j 
                    \Rightarrow{} 
                    i \neq j\ \ \wedge \
                    s^j \in ael(k)                        
                \]
        \end{itemize}
        
        \critic{blue}{The analogy is similar to the property that every unlock must be paired with a subsequent lock, which enforces the condition that a lock can be acquired only when it has been released.}

%Ordering Relation among Events----------------------------------------------------------------------------------------------------------------------       
        \subsection{Ordering Relations among Events}
        
        %Agent Order
        \subsubsection{Agent Order ($\stck{_\textit{ao}}$)}
            A total order among events belonging to the same agent event list. It is analogous to intra-thread ordering. For example, if two events $e$ and $d$ belong to the same agent event list , then either $\reln{e}{\textit{ao}}{d}$ or $\reln{d}{\textit{ao}}{e}$. 
            
           % \critic{blue}{Note that the relations are only with respect to events belonging to the same agent. A collection of such relations together form the agent order. This is analogous and meant to be equivalent to what we call as intra-thread sequential order. It is the same as what \textbf{sequenced-before} is defined to be in C++}
        
        %Synchronize With Order
        \subsubsection{Synchronize-With Order ($\stck{_\textit{sw}} $)}
           Represents the synchronizations among different agents through relations between their events. It is a composition of two sets as below: 
                \begin{gather*}
                            \forall{i, j > 0}, \ \langle s_i, s_j \rangle \in ASW\ \Rightarrow{}\ s_i \stck{_\textit{sw}} s_j 
                            \\
                            (\reln{e}{rf}{d}) \ \wedge \ \et{e}{sc} \ \wedge \ \et{d}{sc} \ \wedge \ (\Re(e)\!=\!\Re(d)) \ \Rightarrow{} \ (d \stck{_\textit{sw}} e)
                \end{gather*}
  
       %Happens Before order 
        \subsubsection{Happens Before Order ($\stck{_\textit{hb}}$)}
            A transitive order on events, composed of the following:
                \begin{gather*}
                    \reln{e}{ao}{d} \ \Rightarrow{} \ \reln{e}{hb}{d}
                    \\
                    \reln{e}{sw}{d} \ \Rightarrow{} \ \reln{e}{hb}{d}
                    \\
                    \forall e, d \in SM,\ 
                    \et{e}{init} \ \wedge \ 
                    (\Re(e) \cap_\Re \Re(d) \neq \phi)
                    \ \Rightarrow{} \ 
                    \reln{e}{hb}{d}
                \end{gather*}
        %Memory Order
        
        \subsubsection{Memory Order ($\stck{_\textit{mo}}$)}
            This is a total order on all events which respects happens-before
                \begin{align*}
                    \reln{e}{hb}{d} \ \Rightarrow{} \ \reln{e}{mo}{d}
                \end{align*}


    \subsection{Some Preliminary Definitions}
        
        Before we go into the consistency rules. we define certain preliminary definitions that create a separation based on a program, the axiomatic events and the various ordering relations defined above. This will help us understand where the consistency rules actually apply. 
        
        \begin{definition}{Program.} 
            A \emph{program} is the source code without abstraction to a set of events and ordering relations. In our context, it is the original ECMAScript program. 
        \end{definition}
        
        %What is one run of a program to us?
        \begin{definition}{Candidate.}
            This is a collection of abstracted sest of shared memory events of a program involved in one possible execution, with the added $\stck{_\textit{ao}}$ relations. We can think of this as each thread having a set of shared memory events to run in a given intra-thread ordering.
        \end{definition}
        
        \begin{definition}{Candidate Execution.}
            A Candidate with the addition of $\stck{_\textit{sw}}$, $\stck{_\textit{hb}}$ and $\stck{_\textit{mo}}$ relations. This can be viewed as the witness/justification of an actual execution of a Program. Note that there can be many Candidate Executions for a given Candidate.
        \end{definition}
        
        %What values are read when the program is run
        \begin{definition}{Observable Behavior.}
        The set of pairwise $\stck{_\textit{rf}}$ and $\stck{_\textit{rbf}}$ relations that result in one execution of the program. Think of this as our outcome of a program execution.
        \end{definition}
    %-----------------------------------------------------------------------------------------------------------------------------------------
        %\emph{The memory consistency rules restrict the possible Observable Behaviors by specifying constraints on $\stck{_{rf}}$ relations based on a  Candidate Execution. For our purpose and flow in which we successively add relations to set of events, this would also include the implication on $\stck{_{rf}}$ relation while having a $\stck{_{sw}}$ relation among two events.}     

%Valid Execution Rules---------------------------------------------------------------------------------------------------------------------------------
    %MAKE SURE TO PLACE DEFINITONS ON PROGRAM, CANDIDATE, CANDIDATE EXECUTION and OBSERVABLE BEHAVIOURS        

    \subsection{Valid Execution Rules (the Axioms)}
        We now state the memory consistency rules. The rules are on \textit{Candidate Executions} which will place constraints on the possible \textit{Observable behaviors} that may result from it.
         
        %Coherent Reads   
        \subsubsection{Coherent Reads} 
        
            There are certain restrictions of what a read event cannot see in an execution based on $\stck{_\textit{hb}}$ relation with write events.
            
            Consider a read event $e$ and a write event $d$ having at least overlapping ranges:
            \begin{align*}
                \event{e}{R} \ \wedge \ 
                \event{d}{W} \ \wedge \
                (\Re(e) \cap_\Re \Re(d) \neq \phi).
            \end{align*}
            
            A read ($e$) value cannot come from a write ($d$) that has happened after it or if there is a write ($g$) that happens between them, writing to the same memory:     
                \begin{gather*}
                    \reln{e}{hb}{d}\ \Rightarrow{}\ \neg \ \reln{e}{rf}{d}. \\
                    \reln{d}{hb}{e}
                    \ \wedge \ 
                    \reln{d}{hb}{g} \ \wedge \  \reln{g}{hb}{e}
                    \ \Rightarrow{} \
                    \forall x \in (\Re(d) \cap_\Re \Re(g) \cap_\Re \Re(e)), \ \neg \ \reln{e}{rbf}{(d,x)}.
                \end{gather*}
     
      \subsubsection{Tear-Free Reads} 
               If two tear-free writes ($d$ and $g$) and a tear-free read ($e$) all with equal ranges exist, then $e$ can read only from one of them
                \begin{align*}
                      \et{d}{tf}\ \wedge\ \et{g}{tf} \ \wedge \ \et{e}{tf} 
                        \ \wedge \ 
                        (\Re(d) \!=\! \Re(g) \!=\! \Re(e)) 
                        \ \Rightarrow{} \ 
                            ((\reln{e}{rf}{d}) 
                            \ \wedge \ 
                            (\neg \ \reln{e}{rf}{g})) 
                        \ \vee \  
                            ((\reln{e}{rf}{g}) 
                            \ \wedge \
                            (\neg \ \reln{e}{rf}{d})).
                \end{align*}
                    
        \subsubsection{Sequentially Consistent Atomics} 
            To specifically define how events that are sequentially consistent affects what values a read cannot see, we assume the following memory order among writes $d$ and $g$ and a read $e$ to be the premise for all the rules: 
                \begin{align*}
                    d \stck{_{mo}} g \stck{_{mo}} e.
                \end{align*}
            There are three separate cases that restrict $e$ to read from $d$, which are as below:
            \begin{itemize}
                \item If all events are sequentially consistent with equal ranges.
                \item If both $g$ and $d$ are sequentially consistent with equal ranges and they happen before $e$.
                \item If both $e$ and $g$ are sequentially consistent with equal ranges and $d$ happens before them. 
            \end{itemize}
            The above cases can be summarized concisely by the rules below:
                \begin{gather*}
                        \et{d}{sc}\ \wedge\ \et{g}{sc}\ \wedge\ \et{e}{sc} 
                        \ \wedge \ (\Re(d) \!=\! \Re(g) \!=\! \Re(e))
                        \ \Rightarrow{} \ 
                        \neg \ \reln{e}{rf}{d}.
                    \\    
                        \et{d}{sc}\ \wedge\ \et{g}{sc}  
                        \ \wedge \ (\Re(d) \!=\! \Re(g)) 
                        \ \wedge \ \reln{d}{hb}{e}
                        \ \wedge \ \reln{g}{hb}{e}
                        \ \Rightarrow{}\  
                        \neg \ \reln{e}{rf}{d}.
                    \\
                        \et{g}{sc}\ \wedge\ \et{e}{sc}  
                        \ \wedge \ (\Re(g) \!= \!\Re(e)) 
                        \ \wedge \ \reln{d}{hb}{g} 
                        \ \wedge \ \reln{d}{hb}{e}
                        \ \Rightarrow \ 
                        \neg \ \reln{e}{rf}{d}.
                \end{gather*}
  

%Races----------------------------------------------------------------------------------------------------------------------------------
        
    \section{Race}
        
        \paragraph{Race Condition $RC$} 
            We assume race condition to be a set of pairs of events that are in a race.
            
            Two events $e$ and $d$ are in a race condition when they are shared memory events 
            
             \[(e \in \set{SM}) \wedge (d \in \set{SM})\]
            
            having at least overlapping ranges, which are either two writes or the two events are involved in a $\stck{_{rf}}$ relation with each other. 
            
               
            \[ 
                 (e,d \in (\set{W} \cup \set{RMW})  \wedge (\Re(d) \cap_{\Re} \Re(e) \neq \phi)) 
                            \vee ((d \stck{_{rf}} e) \vee (e \stck{_{rf}} d)) \Rightarrow (e,d) \in RC 
            \]
            
             \critic{blue}{Though we say it as write events, they also encompass read-modify-write events, as specified by the axiom above.}
            
        \critic{red}{It is interesting to note that the standard specifies only those read-write events that do have a $\stck{_{rf}}$ relation among them to be in a race condition, while the relation signifies an order that has resolved itself because the $\stck{_{rf}}$ relation is there. According to me, their intention was to say that if there \textit{could} exist a $\stck{_{rf}}$ relation among two events, then they would definitely be in a race. This clarification needs to be incorporated in the axiom that defines what is a race condition.}
        
       
        
        \paragraph{Data Race $DR$} 
            Two events are in a data race when they are already in a race condition and when the two events are not of type $sc$ together or they have overlapping ranges: 
            
            \[(e,d) \in RC  
                \wedge ((\neg (e:sc) \vee \neg(d:sc)) 
                \vee (\Re(e) \cap_{\Re} \Re(d) \neq \Re(e) \cup_{\Re} \Re(d)) ) 
                \Rightarrow (e,d) \in DR
                \]
        \critic{red}{The definition for data race also implies that sequentially consistent events with overlapping ranges are considered to be in a data race. This is counter-intuitive in the sense that if all agents observe the same order in which these events happen, then there is no question of it being in a data-race as such, there is a unanimous agreement among all agents about the order in which they take place during the execution.}
        
        
    \paragraph{Data-Race-Free (DRF) Programs}
        An execution is considered data-race free if none of the above conditions for data-races occur among events. A program is data-race free if all its executions are data race free. 
        \newline\newline
        \textit{\textbf{The memory model guarantees sequential consistency for all data-race free programs.}}
        

%Consistent Executions-------------------------------------------------------------------------------------------------------------------

   %Consistent Executions-------------------------------------------------------------------------------------------------------------------

   \section{Consistent Executions (Valid Observables)}
        A valid observable behaviour is when:
        \begin{enumerate}
           \item No $\stck{_\textit{rf}}$ relation violates the above memory consistency rules.
           \item $\stck{_\textit{hb}}$ is a strict partial order.
        \end{enumerate} 

        \textit{The memory model guarantees that every program must have at least one valid observable behaviour.}

        \critic{blue}{There is also some conditions on host-specific events (which we mentioned is not of our main concern) and what is called a chosen read, which is nothing but the reads that the underlying hardware memory model allows. Since we are not concerned with the memory models of different hardware, this restriction on reads is not of our concern.}
    
    
\input{InstructionReordering/reordering_opt.tex}


\section{Introduction}
    Instruction reordering is a common operation done by the compiler / hardware for optimization, essential to instruction scheduling of course, but also implicit in loop invariant removal, partial redundancy elimination, and other optimizations that may move instructions. 
    However, whether we can do such reordering freely given a concurrent program using relaxed memory accesses is a bit unclear. 
     
    \paragraph{Simple reordering is not straightforward under shared memory semantics}
    The main reason is that memory accesses here, do not just perform the desired operation (i.e Read / Write) but also imply certain visibility guarantees across all the other threads.  
    In our observation, we find that, the relaxed memory model of Javascript prescribe semantics for visibility using the $\stck{_{hb}}$ relations. 
    
    \paragraph{Some Examples}

        We show a couple of examples to showcase why reordering may not be that straightforward. 

        Consider the first example in Figure~\ref{reord:example1(a)} below of a Candidate and the resultant candidate after reordering two events.
        The figure on the left is the original candidate and that on the right is after reordering the two reads of $T2$.
        The observable behavior in question is written in the middle. 
        \begin{figure}[H]
            \centering
            \includegraphics[scale=0.7]{5.InstructionReordering/0.Intro/ReorderingExample1(a).pdf}
            \caption{First example for reordering with candidates of the original program and its reordered counterpart.}
            \label{reord:example1(a)} 
        \end{figure}
        
        Figure~\ref{reord:example1(b)} has two sets of relations. 
        The first justifies the outcome for the reordered candidate. 
        While the second justifies the original candidate. 
        Notice that for the second, one may have a read memory ordered before a write that it reads from. 
        This is quite counter intuitive to understand at first. 
        But strictly from the semantics of the model, this justification of the observable behavior is completely valid. 
        \begin{figure}[H]
            \centering
            \includegraphics[scale=0.7]{5.InstructionReordering/0.Intro/ReorderingExample1(b).pdf}
            \caption{The set of partial order relations justifying the observable behavior in question for both the candidates in Figure~\ref{reord:example1(a)}.} 
            \label{reord:example1(b)}
        \end{figure}

        
        Consider another example in Figure~\ref{reord:example2(a)}.
        The figure on the left is the original candidate and that on the right is after reordering the two events of $T1$.
        The observable behavior in question is written in the middle. 
        \begin{figure}[H]
            \centering
            \includegraphics[scale=0.7]{5.InstructionReordering/0.Intro/ReorderingExample2(a).pdf}
            \caption{Second example for reordering with candidates of the original program and its reordered counterpart.} 
            \label{reord:example2(a)}
        \end{figure}

        
        Figure~\ref{reord:example2(b)} has two sets of relations. 
        The first justifies that such an outcome is not possible for the original program candidate due to Axiom \ref{CoRe}. 
        While the second justifies that this outcome is possible for the reordered program.
        Note that we cannnot infer in the reordered candidate the set of relations for any candidate execution to have $\reln{a=x;_{uo}}{hb}{x=1;_{uo}}$. 
        \begin{figure}[H]
            \centering
            \includegraphics[scale=0.7]{5.InstructionReordering/0.Intro/ReorderingExample2(b).pdf}
            \caption{The set of partial order relations justifying the observable behavior in question for both the candidates in Figure~\ref{reord:example2(a)}.} 
            \label{reord:example2(b)}
        \end{figure}

        The above two examples show that we have to be careful while reordering two events in the same thread. 
        By example case analysis, for each observable behavior, one must check all possible candidate executions and assert whether such an observable is possible or not. 
        This method of checking validity of reordering will scale exponentially as the program size increases. 
        It is often also the case that the compiler may not have information on which exact events would be executed in other threads to assert such reordering is valid or not. 

    
    
    
    
    


%The end ? ------------------------------------------------------------------------------------------------------------------------------    
    
\end{document}
