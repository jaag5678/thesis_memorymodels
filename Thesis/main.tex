%AXIOMATIC MODEL DEMYSTIFIED----------------------------------------------

\documentclass[12pt, TexShade, letterpaper]{report}
\usepackage[utf8]{inputenc}
\usepackage{amsmath}
\usepackage{amssymb}
\usepackage{geometry}
\usepackage[dvipsnames]{xcolor}
\usepackage{graphicx}
\usepackage{tikz}
\usepackage{amsthm}
\usepackage{float}
\usepackage{centernot}
\usepackage{tasks}
\usepackage{fancyhdr}
\usepackage{lmodern}

\usepackage[backend=biber]{biblatex}

\addbibresource{references.bib}


% Overwrite the plain page style with a red line and page numbering
\fancypagestyle{plain}{%
	\fancyhf{} % clear all header and footer fields
	\fancyhead[R]{\textbf{\thepage}} % except the center
}

% Create the fancy page style header
\pagestyle{fancy}
\fancyhf{}
\lhead{\textbf{\nouppercase{\leftmark}}}
\chead{}
\rhead{\textbf{\thepage}}


% Set page numbering to roman
\setcounter{page}{2}\renewcommand{\thepage}{\roman{page}}

\author{\textcopyright Author, August, 2020}
\date{}


\usepackage{xpatch}
\xpretocmd\headrule{\color{red}}{}{\PatchFailed}





%Short form to use stack_relative
\newcommand{\stck}{\stackrel{\longrightarrow}}

%A different version of the above 
\newcommand{\stckdet}[1]{\stackrel{{#1}}}

%We write a lot of relations between two events using the orderings, so a short form to use that
\newcommand{\reln}[3]{#1\stck{_{#2}}#3}

%We use a more detailed version of the above relation to indicate direct / indirect relations 

\newcommand{\reldet}[4]{#1\stckdet{_{#2}}{{\stck{_{#3}}}}#4}

%We also will introduce a short form to write an event belongs to some set
\newcommand{\event}[2]{#1\!\in\!#2}

%To make events and their type more close to each other
\newcommand{\typ}[1]{\textit{#1}}
\newcommand{\et}[2]{#1\!:\!\typ{#2}}

%Short form to write color text
\newcommand{\critic}[2]{\textcolor{#1}{\footnotesize #2}}

%A new command to quickly use cons function in formal descriptions
\newcommand{\cons}[2]{\textit{cons}(#1,#2)}

%Useful command syntax
\newcommand{\rmw}{\textit{rmw}\,}
\newcommand{\set}[1]{\textbf{\textit{#1}}}

%Some preliminary latex commands to format writing theorems 
\newtheorem{axiom}{Axiom}

\newtheorem{lemma}{Lemma}

\newtheorem{theorem}{Theorem}[chapter]

\newtheorem{corollary}{Corollary}[theorem]

\newtheorem{definition}{Definition}[section]

\newtheorem{property}{Property}


\begin{document}

\begin{titlepage}
    \begin{center}
        \vspace*{0.5cm}

        \LARGE
        \textbf{Analysis of the ECMAScript Memory Model : A Program Transformation Perspective}
        
        \vspace{1cm}
        
        \textit{Akshay Gopalakrishnan}
        
        \vspace{7cm}
        
        % \includegraphics[width=0.25\textwidth]{mcglogo.png}
        
        \Large
        School of Computer Science
        
        \vspace{5mm}
        McGill University
        
        \vspace{5mm}
    %	Montr\'eal, Qu\'ebec, Canada
        
        \vspace{5mm}
        August 15, 2020
        \small
        \vspace{0.5cm}
        {\color{red} \hrule height 0.75mm}
        
        \vspace{0.3cm}
        A thesis submitted to McGill University in partial fulfillment of the requirements of the degree of Computer Science
        
        \copyright\hspace{0.5mm}2020 Author
        
    \end{center}
\end{titlepage}

\setlength{\voffset}{2cm}
\renewcommand{\chaptermark}[1]{%
    \markboth{\thechapter.\ #1}{}}
    

    \chapter*{Abstract}\markboth{Abstract}{}
	\label{chap:engAbstract}
%	\addcontentsline{toc}{section}{\nameref{chap:engAbstract}}

\chapter*{Abrégé}\markboth{Abrégé}{}
	\label{chap:frAbstract}
%	\addcontentsline{toc}{section}{\nameref{chap:frAbstract}}

\chapter*{Acknowledgements}\markboth{Acknowledgements}{}
	\label{chap:acknowledgments}
%	\addcontentsline{toc}{section}{\nameref{chap:acknowledgments}}


 % Start of ToC, LoT, gls
 \tableofcontents\thispagestyle{plain}

 \listoffigures\thispagestyle{plain}
%	\addcontentsline{toc}{section}{\listfigurename}
 \listoftables
%	\addcontentsline{toc}{section}{\listtablename}

  \clearpage
 \pagenumbering{arabic} % restart page numbers at one, now in arabic style

    \chapter{Introduction}
    
  
   
   Our analysis is based on this corrected model by $WATTTTTT$ which is incorporated in the ECMAScript draft specification. As far as our knowledge goes, no analysis has been done on this model to identify its implications on standard compiler optimizations. 

    \chapter{Background}
    
  
   
   Our analysis is based on this corrected model by $WATTTTTT$ which is incorporated in the ECMAScript draft specification. As far as our knowledge goes, no analysis has been done on this model to identify its implications on standard compiler optimizations. 

    %\chapter{Related Work}

    %\chapter{The Memory Model}
    %The ECMAScript Memory Model
    %
  
   
   Our analysis is based on this corrected model by $WATTTTTT$ which is incorporated in the ECMAScript draft specification. As far as our knowledge goes, no analysis has been done on this model to identify its implications on standard compiler optimizations. 

    %Instruction reordering
    %\chapter{Instruction Reordering}
    %
  
   
   Our analysis is based on this corrected model by $WATTTTTT$ which is incorporated in the ECMAScript draft specification. As far as our knowledge goes, no analysis has been done on this model to identify its implications on standard compiler optimizations. 

    %Elimination
    %\chapter{Elimination} 
    %
\section{Introduction}
    Instruction reordering is a common operation done by the compiler / hardware for optimization, essential to instruction scheduling of course, but also implicit in loop invariant removal, partial redundancy elimination, and other optimizations that may move instructions. 
    However, whether we can do such reordering freely given a concurrent program using relaxed memory accesses is a bit unclear. 
     
    \paragraph{Simple reordering is not straightforward under shared memory semantics}
    The main reason is that memory accesses here, do not just perform the desired operation (i.e Read / Write) but also imply certain visibility guarantees across all the other threads.  
    In our observation, we find that, the relaxed memory model of Javascript prescribe semantics for visibility using the $\stck{_{hb}}$ relations. 
    
    \paragraph{Some Examples}

        We show a couple of examples to showcase why reordering may not be that straightforward. 

        Consider the first example in Figure~\ref{reord:example1(a)} below of a Candidate and the resultant candidate after reordering two events.
        The figure on the left is the original candidate and that on the right is after reordering the two reads of $T2$.
        The observable behavior in question is written in the middle. 
        \begin{figure}[H]
            \centering
            \includegraphics[scale=0.7]{5.InstructionReordering/0.Intro/ReorderingExample1(a).pdf}
            \caption{First example for reordering with candidates of the original program and its reordered counterpart.}
            \label{reord:example1(a)} 
        \end{figure}
        
        Figure~\ref{reord:example1(b)} has two sets of relations. 
        The first justifies the outcome for the reordered candidate. 
        While the second justifies the original candidate. 
        Notice that for the second, one may have a read memory ordered before a write that it reads from. 
        This is quite counter intuitive to understand at first. 
        But strictly from the semantics of the model, this justification of the observable behavior is completely valid. 
        \begin{figure}[H]
            \centering
            \includegraphics[scale=0.7]{5.InstructionReordering/0.Intro/ReorderingExample1(b).pdf}
            \caption{The set of partial order relations justifying the observable behavior in question for both the candidates in Figure~\ref{reord:example1(a)}.} 
            \label{reord:example1(b)}
        \end{figure}

        
        Consider another example in Figure~\ref{reord:example2(a)}.
        The figure on the left is the original candidate and that on the right is after reordering the two events of $T1$.
        The observable behavior in question is written in the middle. 
        \begin{figure}[H]
            \centering
            \includegraphics[scale=0.7]{5.InstructionReordering/0.Intro/ReorderingExample2(a).pdf}
            \caption{Second example for reordering with candidates of the original program and its reordered counterpart.} 
            \label{reord:example2(a)}
        \end{figure}

        
        Figure~\ref{reord:example2(b)} has two sets of relations. 
        The first justifies that such an outcome is not possible for the original program candidate due to Axiom \ref{CoRe}. 
        While the second justifies that this outcome is possible for the reordered program.
        Note that we cannnot infer in the reordered candidate the set of relations for any candidate execution to have $\reln{a=x;_{uo}}{hb}{x=1;_{uo}}$. 
        \begin{figure}[H]
            \centering
            \includegraphics[scale=0.7]{5.InstructionReordering/0.Intro/ReorderingExample2(b).pdf}
            \caption{The set of partial order relations justifying the observable behavior in question for both the candidates in Figure~\ref{reord:example2(a)}.} 
            \label{reord:example2(b)}
        \end{figure}

        The above two examples show that we have to be careful while reordering two events in the same thread. 
        By example case analysis, for each observable behavior, one must check all possible candidate executions and assert whether such an observable is possible or not. 
        This method of checking validity of reordering will scale exponentially as the program size increases. 
        It is often also the case that the compiler may not have information on which exact events would be executed in other threads to assert such reordering is valid or not. 

    
    
    
    
    

    %\chapter{Conclusion, Summary, Future Work}
    %
  
   
   Our analysis is based on this corrected model by $WATTTTTT$ which is incorporated in the ECMAScript draft specification. As far as our knowledge goes, no analysis has been done on this model to identify its implications on standard compiler optimizations. 
    
    \printbibliography

\end{document}
