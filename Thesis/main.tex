%AXIOMATIC MODEL DEMYSTIFIED----------------------------------------------

\documentclass[12pt, TexShade, letterpaper]{report}
\usepackage[utf8]{inputenc}
\usepackage{amsmath}
\usepackage{amssymb}
\usepackage{geometry}
\usepackage[dvipsnames]{xcolor}
\usepackage{graphicx}
\usepackage{tikz}
\usepackage{amsthm}
\usepackage{float}
\usepackage{centernot}
\usepackage{tasks}
\usepackage{fancyhdr}
\usepackage{lmodern}

\usepackage[backend=biber,
    style=ieee,]{biblatex}

\addbibresource{references.bib}


% Overwrite the plain page style with a red line and page numbering
\fancypagestyle{plain}{%
	\fancyhf{} % clear all header and footer fields
	\fancyhead[R]{\textbf{\thepage}} % except the center
}

% Create the fancy page style header
\pagestyle{fancy}
\fancyhf{}
\lhead{\textbf{\nouppercase{\leftmark}}}
\chead{}
\rhead{\textbf{\thepage}}


% Set page numbering to roman
\setcounter{page}{2}\renewcommand{\thepage}{\roman{page}}

\author{\textcopyright Author, August, 2020}
\date{}

\usepackage{xpatch}
\xpretocmd\headrule{\color{red}}{}{\PatchFailed}

%Short form to use stack_relative
\newcommand{\stck}{\stackrel{\longrightarrow}}

%We write a lot of relations between two events using the orderings, so a short form to use that
\newcommand{\reln}[3]{#1\stck{_{#2}}#3}

%We also will introduce a short form to write an event belongs to some set
\newcommand{\event}[2]{#1\!\in\!#2}

%To make events and their type more close to each other
\newcommand{\typ}[1]{\textit{#1}}
\newcommand{\et}[2]{#1\!:\!\typ{#2}}

%Short form to write color text
\newcommand{\critic}[2]{\textcolor{#1}{\footnotesize #2}}

%A new command to quickly use cons function in formal descriptions
\newcommand{\cons}[2]{\textit{cons}(#1,#2)}

%Useful command syntax
\newcommand{\rmw}{\textit{rmw}\,}
\newcommand{\set}[1]{\textbf{\textit{#1}}}

%Some preliminary latex commands to format writing theorems 
\newtheorem{axiom}{Axiom}

\newtheorem{lemma}{Lemma}

\newtheorem{theorem}{Theorem}[chapter]

\newtheorem{corollary}{Corollary}[theorem]

\newtheorem{definition}{Definition}[section]

\newtheorem{property}{Property}


\begin{document}

\begin{titlepage}
    \begin{center}
        \vspace*{0.5cm}

        \LARGE
        \textbf{Analysis of the ECMAScript Memory Model : A Program Transformation Perspective}
        
        \vspace{1cm}
        
        \textit{Akshay Gopalakrishnan}
        
        \vspace{7cm}
        
        % \includegraphics[width=0.25\textwidth]{mcglogo.png}
        
        \Large
        School of Computer Science
        
        \vspace{5mm}
        McGill University
        
        \vspace{5mm}
    %	Montr\'eal, Qu\'ebec, Canada
        
        \vspace{5mm}
        August 15, 2020
        \small
        \vspace{0.5cm}
        {\color{red} \hrule height 0.75mm}
        
        \vspace{0.3cm}
        A thesis submitted to McGill University in partial fulfillment of the requirements of the degree of Computer Science
        
        \copyright\hspace{0.5mm}2020 Author
        
    \end{center}
\end{titlepage}

\setlength{\voffset}{2cm}
\renewcommand{\chaptermark}[1]{%
    \markboth{\thechapter.\ #1}{}}
    

    \chapter*{Abstract}\markboth{Abstract}{}
	\label{chap:engAbstract}
%	\addcontentsline{toc}{section}{\nameref{chap:engAbstract}}
    Concurrent memory accesses have been shown to give us tremendous performance benefits compared to its sequential counterparts.
WIth the recent addition of several hardware features such as read/write buffers, speculation, etc., more efficient forms of concurrent memory accesss are introduced.
Called relaxed memory accesses, they are used to gain substantial improvement in the performance of concurrent programs. 
A relaxed memory consistency model specifically describes the semantics of such memory accesses for a particular programming language. 
Historically, such semantics are often ill-defined or misunderstood, and have been shown to conflict with common program transformations essential for the performance of programs overall. 
In this thesis, we give a formal declarative(axiomatic) style description of the ECMAScript relaxed memory consistency model. 
We analyze the impact of this model on two of the most common program transformations, viz. instruction reordering and elimination. 
We give a conservative proof under which such optimization is allowed for relaxed memory accesses. 
We use this result to reason about the validity of loop invariant code motion under the same model. 
We conclude this thesis by eliciting the limitations of our approach, critique on the semantics of the model, possible future work using our results and pending foundational questions that we discovered while working on this thesis.

\chapter*{Abrégé}\markboth{Abrégé}{}
	\label{chap:frAbstract}
%	\addcontentsline{toc}{section}{\nameref{chap:frAbstract}}

\chapter*{Acknowledgements}\markboth{Acknowledgements}{}
	\label{chap:acknowledgments}
%	\addcontentsline{toc}{section}{\nameref{chap:acknowledgments}}
    There are countless people who have influenced me through these years and have motivated me to keep doing this thesis.
Unfortunately, I cannot mention all of their names due to space constraints as well as the fact that I may not know their influence on me yet.
However, I do want to mention a few important names of people that have in my eyes influenced me the most. 
Firstly, my advisor Professor Clark Verbrugge.
Having not done any rigorous theoretical work or research before and yet wanting to do a theoretical one as a thesis, he gave me high level guidelines to ground me and to keep me from being overwhelmed from framing my own theorems, lemmas, proofs, etc. 
I would not have gotten a better advisor than him, especially during the initial years when I was trying to build an intuitive foundation. 
Also, to advise someone in this sense is quite a challenge in my eyes.
Secondly, my friend, Aarti Kashyap, mainly for being a patient listener, a motivator that made me attend several conferences and for always giving constructive feedback. 
I needed an outside perspective and somebody to discuss with about various aspects of this research at any time; and she being there for it has helped me shape ideas of this thesis better. 
Thirdly, Conrad Watt, mainly for giving me an inside perspective of this topic of research and motivating me to pursue this thesis topic.
Lastly, my family, friends and lab-mates, who always assured me the support (mentally or otherwise) I would need at any time during my masters. 
I would like to also thank Viktor Vafeiadis, my internship advisor, who helped me look at the research I did all this while in a simplistic way.
This has in some way reflected the way I write or present my research ideas.
Finally, God, for guiding me in his/her own way and presenting me with ample opportunities to grow as a human being as well as a researcher these years.

 % Start of ToC, LoT, gls
 \tableofcontents\thispagestyle{plain}

 \listoffigures\thispagestyle{plain}
%	\addcontentsline{toc}{section}{\listfigurename}
 %\listoftables
%	\addcontentsline{toc}{section}{\listtablename}

  \clearpage
 \pagenumbering{arabic} % restart page numbers at one, now in arabic style

    \chapter{Introduction}
    
  
   
   Our analysis is based on this corrected model by $WATTTTTT$ which is incorporated in the ECMAScript draft specification. As far as our knowledge goes, no analysis has been done on this model to identify its implications on standard compiler optimizations. 

    \chapter{Background}
    
  
   
   Our analysis is based on this corrected model by $WATTTTTT$ which is incorporated in the ECMAScript draft specification. As far as our knowledge goes, no analysis has been done on this model to identify its implications on standard compiler optimizations. 

    %The ECMAScript Memory Model
    \chapter{The Memory Model}
    
  
   
   Our analysis is based on this corrected model by $WATTTTTT$ which is incorporated in the ECMAScript draft specification. As far as our knowledge goes, no analysis has been done on this model to identify its implications on standard compiler optimizations. 

    %Instruction reordering
    \chapter{Instruction Reordering}
    
  
   
   Our analysis is based on this corrected model by $WATTTTTT$ which is incorporated in the ECMAScript draft specification. As far as our knowledge goes, no analysis has been done on this model to identify its implications on standard compiler optimizations. 

    %Elimination
    \chapter{Elimination} 
    
  
   
   Our analysis is based on this corrected model by $WATTTTTT$ which is incorporated in the ECMAScript draft specification. As far as our knowledge goes, no analysis has been done on this model to identify its implications on standard compiler optimizations. 

    \chapter{Conclusion, Summary, Future Work}
    
  
   
   Our analysis is based on this corrected model by $WATTTTTT$ which is incorporated in the ECMAScript draft specification. As far as our knowledge goes, no analysis has been done on this model to identify its implications on standard compiler optimizations. 
    
    \printbibliography

    \appendix 

    \chapter{Counter Examples}
    
  
   
   Our analysis is based on this corrected model by $WATTTTTT$ which is incorporated in the ECMAScript draft specification. As far as our knowledge goes, no analysis has been done on this model to identify its implications on standard compiler optimizations. 

\end{document}
