%AGENTS----------------------------------------------------------------------------------------------------------------------------------------  
\subsection{Agents, Events and their Types}

    \subsubsection{Agents}
        A concurrent program involves different threads/processes running concurrently. Agents are analogous to different threads/processes. Agents actually have more meaning than what we refer to here. However, with respect to the memory consistency model , we can safely abstract them to just mean threads/processes.
        %Agent Clusters
        \paragraph{Agent Cluster}
        Collection of agents concurrently communicating with each other through means of shared memory form an agent cluster.  There can be multiple agent clusters. However, an agent can only belong to one agent cluster. Agents communicating through message passing do not belong in the same agent cluster. 

        For our purpose, we assume just one agent cluster having one shared memory using which agents communicate. 

        \critic{red}{Perhaps draw a figure here to represent the role of agent clusters and different shared memory fragments.}
        %Agent Event Set
        \paragraph{Agent Event List $(ael)$}
        Every agent is mapped to a list of events. Operationally, these events are appended to the list during evaluation. We define $ael$ as a mapping of each agent to a list of events.
        
        The standard refers this to be an Event List, but we find it a bit misleading as it does not signify a list for each agent. Hence we name it as Agent Event List. 

        

            

