This chapter addresses the validity of elimination under the ECMAScript memory model.
We first start by showing some examples of Candidate Executions where write elimination is not safe in the relaxed memory context.
We then give an explaination of why read elimination may not be safe to do.   
We then formulate two theorems, one for read elmination and one for write with a corresponding corollary for write elimination. 
Similar to reordering, we address elimination at the program level (still abstracted to a set of shared memory events) involving loops and conditional branching.
We lastly formulate theorems and their corresponding proof which shows how loop invariant code motion can be described as a combination of elimination and reordering at the Candidate execution level. 
We conclude by mentioning those cases of code motion that cannot be explained yet using our two program transformations.
\ \newline
\ \newline  
\hrule 
\ \newline 
\ \newline 

%Change later 

The model we consider is the current draft specification \cite{ECMA} of the ECMAScript standard. 
The semantics of the model we consider has remain unchanged since the time we started our investigation (2019), so we believe our work will also be of use to those working on it. 
The specification is claimed to be \textit{axiomatic} by definition, which should, in our view remove the complexities of the rest of the standard from the semantics of the model.
However, there are some concerns with it: 

\paragraph{The Model is Quite Algorithmic}
    Although the standard states that the model is not supposed to be operational, the specifications of the model hint otherwise. 
    They are defined as relational constraints on certain \textit{abstract operations} which are not necessary to understand the semantics of the model.  
    As an example, consider one of the \textit{axioms} of the model in Figure~\ref{model:Std1} as stated by the standard. 
    \begin{figure}[H]
        \centering 
        \includegraphics[scale=0.6]{3.ECMAScriptMemoryModel/ECMAScriptStdCoherentReads.pdf}
        \caption{The ECMAScript specification for Coherent Reads.}
        \label{model:Std1}
    \end{figure}
    The definition in Figure~\ref{model:Std1} is specified in terms of a return value from an abstract operation. 
    Understanding this requires one to know the definitions for \textit{Ws, execution, SharedDataBlockEventSet}, etc. although this is not required to understand what the axiom is about, which informally can be stated as below in two points:
    \begin{itemize}
        \item A read's value cannot come from a write that has happened after it. 
        \item A read's value cannot come from a write that has been overwritten by some other write.  
    \end{itemize}
    Axiomatically, we define the above two constraints using binary relations that we derive (also in some sense, take directly) from the specification in Section 2 of this chapter. 
    
\paragraph{Certain Unnecessary Definitions}
    Certain abstract operations are not required to capture the semantics of the model. 
    One such example is shown in Figure~\ref{model:Std2}
    \begin{figure}[H]
        \centering 
        \includegraphics[scale=0.6]{3.ECMAScriptMemoryModel/ECMAScriptStd.pdf}
        \caption{The ECMAScript specification for Compose Write Event Bytes \cite{ECMA}.}
        \label{model:Std2}
    \end{figure}
    Figure~\ref{model:Std2} is the definition of an abstract operation. 
    Understanding this operation would require the meaning of the terms \textit{ModifyOp, Payload, Ws} and \textit{ByteIndex}. 
    In its essence, this operation determines the read-values read by a single event by collecting the values from their corresponding writes. 
    We noticed that one need not know this operation nor understand its function as it is not necessary in the axiomatic semantics of the model. 
    Other such abstract operations which may not be essential are \textit{ValueOfReadEvent} and \textit{ValidChosenReads}\cite{ECMA}. 

\paragraph{Still a bit verbose}
    
    The entire model, though algorithmic in its structure, is still quite verbose in its details, which makes it difficult to understand the model semantics. 
    Figure~\ref{model:Std3} is another \textit{axiom} from the standard. 
    \begin{figure}[H]
        \centering 
        \includegraphics[scale=0.6]{3.ECMAScriptMemoryModel/ECMAScriptStdSeqCnsAt.pdf}
        \caption{The ECMAScript specification for Sequentially Consistent Atomics axiom.}
        \label{model:Std3}
    \end{figure}
    The definition in Figure~\ref{model:Std3}, is not concise enough to reason about it mathematically. 
    In addition, the part after Note1 in Figure~\ref{model:Std3} is not a semantic specification, rather a programming guideline while using 
    atomic memory accesses. 
    We will reduce the above entire axiom into three main patterns using binary relations.

Given the above concerns about the specification in the standard, we found the need to have a concise formal description of the model. 
In the following sections, we define what agents and events are, followed by several binary relations among different events.


%Read Elimination 
    %Read Elimination
    \begin{theorem}
    \label{WriteElim}
    Consider a candidate $C$ of a program and its possible \textit{Candidate Executions} where $\stck{_\textit{hb}}$ is strictly partial order. 
    Consider two \textbf{write} events $e$ and $d$ in $C$ such that 
    \begin{align*}
        \cons{e}{d} \ \wedge \ \reln{e}{ao}{d}. 
    \end{align*}
    Consider a Candidate $C'$ after eliminating the event $e$ from $C$.  
    If
    \begin{align*}
        \et{e}{uo} \ \wedge \ \Re(e) = \Re(d). 
    \end{align*}
    then the set of Observable behaviors of $C'$ is a subset of $C$.  
\end{theorem}

    \begin{proof}
    Once again, we look at this as a write elimination done on a Candidate Execution of $C$. We start by proving when other happens-before relations remain intact. Followed by identifying relations lost due to elimination and a proof for when these relations do not introduce new observable behaviors. 
    
   \paragraph{1. Preserving \emph{happens-before} relations}
        The relations we want to preserve are those that are dervied through relation with $e$, viz. using the following two relations:
        \begin{tasks}(2)
            \task $\reln{k}{hb}{e}$
            \task $\reln{e}{hb}{k}$
        \end{tasks}

        We can divide the events involved in the above into two sets:
        \begin{align*}
            K_b = \{k \ | \ \reln{k}{hb}{e} \}. \\
            K_a = \{k \ | \ \reln{e}{hb}{k} \}. 
        \end{align*}

        We need to ensure the following relations hold after elimination.
        \begin{align*}
            \forall k_a \in K_a \ \wedge \ \forall k_b \in K_b \ . \ \reln{k_b}{hb}{k_a}
        \end{align*}

        Similar to reordering, we need to have a valid pivot pair $<p_b, p_a>$ such that 
        \begin{align*}
            \forall k_b \neq p_b \in K_b \ . \ \reln{k_b}{hb}{p_b} \\
            \forall k_a \neq p_a \in K_a \ . \ \reln{p_a}{hb}{k_a} 
        \end{align*}

        By Lemma \ref{Lemma1}, $\et{e}{uo}$ is the only case where $p_b$ can be a valid pivot. 
        By Lemma \ref{Lemma2}, $\et{e}{uo} \ \vee \ \et{e}{sc}$ are the cases where $p_a$ can be a valid pivot. 
        We need both the above conditions to be satisfied to have a valid pivot pair. 
        Hence, $\et{e}{uo}$ is the only possibility in which a valid pivot pair can exist. 

        \critic{blue}{Put a figure here to show this pivot role.}
   
    \paragraph{2. The \emph{happens-before} relations lost}

    The relations lost are those attached to the event $e$, which are: 
    \begin{align}
        \reln{k}{hb}{e} \ \vee \ \reln{e}{hb}{k}
    \end{align}
    
    \critic{red}{Do we need to prove that these are the only relations lost? Proof part 1 implicitly shows this.}

   \paragraph{3. Presence of Cycles?}
        
Because no new $\stck{_{hb}}$ relations are introduced, and because original candidate executions have $\stck{_{hb}}$ as a strict partial order, no cycles are introduced after elimination. 

\critic{blue}{Perhaps write this argument a bit better.}

   \paragraph{4. Do the lost relations result in New Observable Behaviors?}

        To answer this, we need to see whether the relations removed had an impact on possible $\stck{_{rf}}$ relations other than those with $e$. 
        We divide our argument into two parts, viz. the two types of relations removed:
        \begin{tasks}(2)
            \task $\reln{k}{hb}{R_{uo}}$. 
            \task $\reln{R_{uo}}{hb}{k}$.
        \end{tasks}

        Figure~\ref{elim_read:case1} shows a breakdown of sub-cases for case (a), varying based
        on the nature of event $k$.
        \begin{figure}[H]
            \centering
            \includegraphics[scale=0.5]{5.Elimination/1.ValidEliminationCandidate/ReadElimProof/ProofParts/Part4_Case1.pdf}
            \caption{The impact of lost relation $\reln{k}{hb}{R_{uo}}$ on observable behaviors.}
            \label{elim_read:case1}
        \end{figure}

        Observations:
        \begin{itemize}
            \item (i) is not a pattern forbidden by the consistency rules.
            \item (ii)(a) is a pattern of Axiom \ref{CoRe}, however, only restricting $\stck{_{rf}}$ relation with $R$ and $W'$(which here is our Unordered Read)
            \item (ii)(b) is a pattern of Axiom \ref{SeqCsAt}, however, once again, only restricting $\stck{_{rf}}$ relation with $R$ and $W'$. 
        \end{itemize}

        Figure~\ref{elim_read:case2} shows a breakdown of sub-cases for case (b), varying based
        on the nature of event $k$.
        \begin{figure}[H]
            \centering
            \includegraphics[scale=0.5]{5.Elimination/1.ValidEliminationCandidate/ReadElimProof/ProofParts/Part4_Case2.pdf}
            \caption{The impact of lost relation $\reln{R_{uo}}{hb}{k}$ on observable behaviors.}
            \label{elim_read:case2}
        \end{figure}

        Observations:
        \begin{itemize}
            \item (i) is not a pattern in any Consistency rules
            \item (ii) is a pattern of Axiom \ref{CoRe}, however, only restricting $\stck{_{rf}}$ relation with $R$ and $W$
        \end{itemize}

        From the above observations, we can infer that the relations removed only have restriction on reads-from relations on the event $e$ we eliminate. 
        Thus, we can conclude that no new observable behaviors are introduced due to the removed $\stck{_{hb}}$ relations. 

\end{proof}


%Write elimination
    %Read Elimination
    \begin{theorem}
    \label{WriteElim}
    Consider a candidate $C$ of a program and its possible \textit{Candidate Executions} where $\stck{_\textit{hb}}$ is strictly partial order. 
    Consider two \textbf{write} events $e$ and $d$ in $C$ such that 
    \begin{align*}
        \cons{e}{d} \ \wedge \ \reln{e}{ao}{d}. 
    \end{align*}
    Consider a Candidate $C'$ after eliminating the event $e$ from $C$.  
    If
    \begin{align*}
        \et{e}{uo} \ \wedge \ \Re(e) = \Re(d). 
    \end{align*}
    then the set of Observable behaviors of $C'$ is a subset of $C$.  
\end{theorem}

    \begin{proof}
    Once again, we look at this as a write elimination done on a Candidate Execution of $C$. We start by proving when other happens-before relations remain intact. Followed by identifying relations lost due to elimination and a proof for when these relations do not introduce new observable behaviors. 
    
   \paragraph{1. Preserving \emph{happens-before} relations}
        The relations we want to preserve are those that are dervied through relation with $e$, viz. using the following two relations:
        \begin{tasks}(2)
            \task $\reln{k}{hb}{e}$
            \task $\reln{e}{hb}{k}$
        \end{tasks}

        We can divide the events involved in the above into two sets:
        \begin{align*}
            K_b = \{k \ | \ \reln{k}{hb}{e} \}. \\
            K_a = \{k \ | \ \reln{e}{hb}{k} \}. 
        \end{align*}

        We need to ensure the following relations hold after elimination.
        \begin{align*}
            \forall k_a \in K_a \ \wedge \ \forall k_b \in K_b \ . \ \reln{k_b}{hb}{k_a}
        \end{align*}

        Similar to reordering, we need to have a valid pivot pair $<p_b, p_a>$ such that 
        \begin{align*}
            \forall k_b \neq p_b \in K_b \ . \ \reln{k_b}{hb}{p_b} \\
            \forall k_a \neq p_a \in K_a \ . \ \reln{p_a}{hb}{k_a} 
        \end{align*}

        By Lemma \ref{Lemma1}, $\et{e}{uo}$ is the only case where $p_b$ can be a valid pivot. 
        By Lemma \ref{Lemma2}, $\et{e}{uo} \ \vee \ \et{e}{sc}$ are the cases where $p_a$ can be a valid pivot. 
        We need both the above conditions to be satisfied to have a valid pivot pair. 
        Hence, $\et{e}{uo}$ is the only possibility in which a valid pivot pair can exist. 

        \critic{blue}{Put a figure here to show this pivot role.}
   
    \paragraph{2. The \emph{happens-before} relations lost}

    The relations lost are those attached to the event $e$, which are: 
    \begin{align}
        \reln{k}{hb}{e} \ \vee \ \reln{e}{hb}{k}
    \end{align}
    
    \critic{red}{Do we need to prove that these are the only relations lost? Proof part 1 implicitly shows this.}

   \paragraph{3. Presence of Cycles?}
        
Because no new $\stck{_{hb}}$ relations are introduced, and because original candidate executions have $\stck{_{hb}}$ as a strict partial order, no cycles are introduced after elimination. 

\critic{blue}{Perhaps write this argument a bit better.}

   \paragraph{4. Do the lost relations result in New Observable Behaviors?}

        To answer this, we need to see whether the relations removed had an impact on possible $\stck{_{rf}}$ relations other than those with $e$. 
        We divide our argument into two parts, viz. the two types of relations removed:
        \begin{tasks}(2)
            \task $\reln{k}{hb}{R_{uo}}$. 
            \task $\reln{R_{uo}}{hb}{k}$.
        \end{tasks}

        Figure~\ref{elim_read:case1} shows a breakdown of sub-cases for case (a), varying based
        on the nature of event $k$.
        \begin{figure}[H]
            \centering
            \includegraphics[scale=0.5]{5.Elimination/1.ValidEliminationCandidate/ReadElimProof/ProofParts/Part4_Case1.pdf}
            \caption{The impact of lost relation $\reln{k}{hb}{R_{uo}}$ on observable behaviors.}
            \label{elim_read:case1}
        \end{figure}

        Observations:
        \begin{itemize}
            \item (i) is not a pattern forbidden by the consistency rules.
            \item (ii)(a) is a pattern of Axiom \ref{CoRe}, however, only restricting $\stck{_{rf}}$ relation with $R$ and $W'$(which here is our Unordered Read)
            \item (ii)(b) is a pattern of Axiom \ref{SeqCsAt}, however, once again, only restricting $\stck{_{rf}}$ relation with $R$ and $W'$. 
        \end{itemize}

        Figure~\ref{elim_read:case2} shows a breakdown of sub-cases for case (b), varying based
        on the nature of event $k$.
        \begin{figure}[H]
            \centering
            \includegraphics[scale=0.5]{5.Elimination/1.ValidEliminationCandidate/ReadElimProof/ProofParts/Part4_Case2.pdf}
            \caption{The impact of lost relation $\reln{R_{uo}}{hb}{k}$ on observable behaviors.}
            \label{elim_read:case2}
        \end{figure}

        Observations:
        \begin{itemize}
            \item (i) is not a pattern in any Consistency rules
            \item (ii) is a pattern of Axiom \ref{CoRe}, however, only restricting $\stck{_{rf}}$ relation with $R$ and $W$
        \end{itemize}

        From the above observations, we can infer that the relations removed only have restriction on reads-from relations on the event $e$ we eliminate. 
        Thus, we can conclude that no new observable behaviors are introduced due to the removed $\stck{_{hb}}$ relations. 

\end{proof}

    
%Going towards program level

  
   
   Our analysis is based on this corrected model by $WATTTTTT$ which is incorporated in the ECMAScript draft specification. As far as our knowledge goes, no analysis has been done on this model to identify its implications on standard compiler optimizations. 

\ \newline
\ \newline  
\hrule 
\ \newline 
\ \newline 
To summarize, this chapter addressed the validity of elimination under the ECMAScript Memory Model. 
We first built a conservative proof for elimination of read, followed by write based on candidate executions.
We later extended it to programs abstracted to the set of shared memory events. 
We then proved when loop invariant code motion is valid using both elimination and reordering at a candidate execution level. 
In the next chapter, we conclude this thesis, by discussing limitations, next steps, critique of the semantics of the model and ending with general foundational problems that need to be addressed in this domain of research. 