\begin{corollary}
    Consider a Candidate C of a program and its Candidate Executions which are valid. Consider two events $e$ and $d$ such that $\neg \cons{e}{d}$ is true in C and $\reln{e}{ao}{d}$. 
    Consider another Candidate C' resulting after reordering $e$ and $d$ in C. 
    Then, the set of Observable behaviors possible in C' is a subset of C only if $Reord(e,d)$ and the following holds true.
    
    \[
        \forall \ k \ \textit{s.t.} \ 
        \reln{e}{ao}{k} \ \wedge \ \reln{k}{ao}{d} \ . \ 
        Reord(e,k) \ \wedge \ Reord(k,d)
    \]
    
    \label{corol1}
\end{corollary}
    
\begin{proof}
    We prove this by induction of number of events $k$ between $e$ and $d$. Let $n$ denote the number of events.

    \paragraph{Base Case: $n = 1.$}
        This means we have one event $k$ such that
            \[
                \reln{e}{ao}{\reln{k}{ao}{d}} 
            \]
        What we want after reordering is 
        \[
            \reln{d}{ao}{\reln{k}{ao}{e}} 
        \]
        Without loss of generality, we can choose to first reorder $k$ and $d$. For this we have $cons(k,d)$ to be true. To reorder, what we only need is $Reord(k,d)$ to hold. Thus, after reordering, we have
        \[
            \reln{e}{ao}{\reln{d}{ao}{k}} 
        \]
        Similarly, now we need $Reord(e,d)$ to hold to reorder them above, after doing so, we will get
        \[
            \reln{d}{ao}{\reln{e}{ao}{k}} 
        \]
        Now lastly, we need to reorder $e$ and $k$ for which we need $Reord(e,k)$ to hold, thus giving us our final result
        \[
            \reln{d}{ao}{\reln{k}{ao}{e}} 
        \]
        By transitive property of subsets, we can conclude that the Observable Behavior of the final program after reordering is a subset of the original.  

    \paragraph{2. Inductive Case $n > 1$}
        Assume the above corollary holds for $n = t$. 
        
        We need to show that for $n = t + 1$, the corollary still holds, for this note firstly that, we have the following ordering relations. 
        
        \[
            \reln{e}{ao}{\reln{k_1}{ao}{\reln{k_2}{ao}{\reln{k_3}{ao}{\reln{...}{ao}{\reln{k_t}{ao}{\reln{k_{t+1}}{ao}{d}}}}}}}  
        \]

        Without loss of generality, we can first reorder $k_{t+1}$ and $d$. To do this, we need $Reord(k_{t+1}, d)$ to hold, thus giving us the resultant ordering.
        \[
            \reln{e}{ao}{\reln{k_1}{ao}{\reln{k_2}{ao}{\reln{k_3}{ao}{\reln{...}{ao}{\reln{k_t}{ao}{\reln{d}{ao}{k_{t+1}}}}}}}}  
        \]
        
        Now we have $t$ such events between $e$ and $d$. With our assumption, we can reorder $e$ and $d$, thus giving us 
        \[
            \reln{d}{ao}{\reln{k_1}{ao}{\reln{k_2}{ao}{\reln{k_3}{ao}{\reln{...}{ao}{\reln{k_t}{ao}{\reln{e}{ao}{k_{t+1}}}}}}}}  
        \]

        Finally, we need to reorder $e$ and $k_{t+1}$ to get our final result, for which we need $Reord(e, k_{t+1})$ to hold, thus giving us finally
        \[
            \reln{d}{ao}{\reln{k_1}{ao}{\reln{k_2}{ao}{\reln{k_3}{ao}{\reln{...}{ao}{\reln{k_t}{ao}{\reln{k_{t+1}}{ao}{e}}}}}}}  
        \]

        Thus we can see that we need two more conditions to hold for us to ensure we can reorder $e$ amd $d$ with $n = t + 1$. by transitive property of subsets, we can conclude that the Observable Behavior of the final program after reordering is a subset of the original.

        Hence, by induction the proof is complete. 

        \critic{blue}{Observe that wherever our argument states the requirement of $Reord$ to hold between two events, it is also the case that those two events are consecutive and have an agent ordering exactly as our Theorem states.}

\end{proof}
    

To investigate the validity of reordering at the program level, we first, in terms of candidates, address code motion. The following two corollaries cover them: 
\begin{corollary}
    Consider a Candidate C of a program and its Candidate Executions which are valid. Consider a set of events $k_{i \in[1,n]}$ such that $\reln{k_i}{ao}{k_{i+1}} \wedge \cons{k_i}{k_{i+1}}$.
    Consider an event $e$ such that 
    \begin{align*}
        \cons{e}{k_1} \ \wedge \ \reln{e}{ao}{k_1}  
    \end{align*}
    Consider another candidate $C'$ with the only differnence from $C$ being $\cons{e}{k_n} \ \wedge \ \reln{k_n}{ao}{e}$.
    Then the set of observable behaviors of $C'$ is a subset of that of $C$ if 
    \begin{align}
        \forall i \in [1,n] \ . \ Reord(e,k_i)
    \end{align} 
\end{corollary}

\begin{proof}
    Apply theorem of reordering successively, and by transititvity of subset relations, the corollary holds.
\end{proof}

\begin{corollary}
    Consider a Candidate C of a program and its Candidate Executions which are valid. Consider a set of events $k_{i \in[1,n]}$ such that $\reln{k_i}{ao}{k_{i+1}} \wedge \cons{k_i}{k_{i+1}}$.
    Consider an event $d$ such that 
    \begin{align*}
        \cons{d}{k_n} \ \wedge \ \reln{k_n}{ao}{d}  
    \end{align*}
    Consider another candidate $C'$ with the only differnence from $C$ being $\cons{d}{k_1} \ \wedge \ \reln{d}{ao}{k_1}$.
    Then the set of observable behaviors of $C'$ is a subset of that of $C$ if 
    \begin{align}
        \forall i \in [1,n] \ . \ Reord(k_i,d)
    \end{align} 
\end{corollary}

\begin{proof}
    Apply theorem of reordering successively, and by transititvity of subset relations, the corollary holds.
\end{proof}
