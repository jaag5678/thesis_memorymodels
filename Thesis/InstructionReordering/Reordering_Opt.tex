
\section{Instruction Reordering}
    Instruction reordering is a common operation in compiler optimization, essential to instruction scheduling of course, but also implicit in loop invariant removal, partial redundancy elimination, and other optimizations that may move instructions. 
    However, whether we can do such reordering freely given a concurrent program using relaxed memory accesses is a bit unclear. 
     
    
    \paragraph{Simple reordering is not straightforward under shared memory semantics}
    The main reason is that memory accesses here, do not just perform the desired operation (i.e Read / Write) but also imply certain visibility guarantees across all the other threads.  
    In our observation, we find that, the relaxed memory model of Javascript prescribe semantics for visibility using the $\stck{_{hb}}$ relations. 
    
    \critic{purple}{Show an example or multiple examples here that enforces visibility due to having sequentially consistent events involved in a Candidate Execution.}
    
    \paragraph{What can be done?}
    An example-based analysis exposes to us the problems that might exist when we perform such reordering of events. 
    However, such an analysis, though would work for small programs to identify the possible conditions under which reordering can be done, become infeasible as the programs scale in length and complexity. 
    This is because of the exponential increase in possible executions as the number of threads and program size in general increase. 
    Hence,  generalizations by using a small sample size is not something we can afford especially when we want to ensure these program trasnformations are done by the compiler in contrast to being done manually.
    
    \paragraph{Our approach}
    Our solution to this is to construct a proof on Candidate Executions of the original program and the transformed one which exposes the possible observable behaviors it can have.   
    The crux of the proof is to guarantee that reordering does not bring any new $\stck{_{rf}}$ (reads-from) relations that did not exist in any Observable Behavior of the original Candidate Execution. 
    It is important to note however, that a proof in this sense would be generalized to any Candidate and is thus conservative.
    So, it might be the case that for specific programs, reordering can be valid, however, in a general sense may not be valid for others. 

    \paragraph{Assumption}
    We make the following assumptions for every program we consider :
    \begin{enumerate}
        \item All events are tear-free
        \item No synchronize events exist
        \item No Read-Modify-Write events exist
        \item All executions of the candidate before reordering have happens-before as a strict partial order
    \end{enumerate}
    
    We first consider when consecutive events in the same agent can be reordered, followed by non-consecutive cases. The crux of the proof is to guarantee that reordering does not bring any new reads-from relations that did not result due to any execution of the original program. 
    
    %GIVE TWO EXAMPLES TO SHOW THIS. POSSIBLY USE THE EXAMPLE ABOVE AND EXPLAIN

    \critic{purple}{The following definitions and lemmas are not particular to instruction reordering, so I think we can make it a point to put this in a section that introduces our work on optimizations.}

    
    \subsection{Some Preliminary Definitions}
        
        Before we go into the consistency rules. we define certain preliminary definitions that create a separation based on a program, the axiomatic events and the various ordering relations defined above. This will help us understand where the consistency rules actually apply. 
        
        \begin{definition}{Program.} 
            A \emph{program} is the source code without abstraction to a set of events and ordering relations. In our context, it is the original ECMAScript program. 
        \end{definition}
        
        %What is one run of a program to us?
        \begin{definition}{Candidate.}
            This is a collection of abstracted sest of shared memory events of a program involved in one possible execution, with the added $\stck{_\textit{ao}}$ relations. We can think of this as each thread having a set of shared memory events to run in a given intra-thread ordering.
        \end{definition}
        
        \begin{definition}{Candidate Execution.}
            A Candidate with the addition of $\stck{_\textit{sw}}$, $\stck{_\textit{hb}}$ and $\stck{_\textit{mo}}$ relations. This can be viewed as the witness/justification of an actual execution of a Program. Note that there can be many Candidate Executions for a given Candidate.
        \end{definition}
        
        %What values are read when the program is run
        \begin{definition}{Observable Behavior.}
        The set of pairwise $\stck{_\textit{rf}}$ and $\stck{_\textit{rbf}}$ relations that result in one execution of the program. Think of this as our outcome of a program execution.
        \end{definition}
    %-----------------------------------------------------------------------------------------------------------------------------------------
        %\emph{The memory consistency rules restrict the possible Observable Behaviors by specifying constraints on $\stck{_{rf}}$ relations based on a  Candidate Execution. For our purpose and flow in which we successively add relations to set of events, this would also include the implication on $\stck{_{rf}}$ relation while having a $\stck{_{sw}}$ relation among two events.}     

    %A new command to quickly use cons function in formal descriptions
\newcommand{\cons}[2]{\textit{cons}(#1,#2)}
  
%--------------------------------------------------------------------------------------------------------------   
    \subsection{Lemmas to assist our proof}    
    In order to assist our proof, we define two lemmas based on the ordering relations. 
    
    \begin{lemma} Consider three events $e$, $d$, and $k$. \\
    
        If
            \[
                \cons{e}{d} \ \wedge \ \reln{e}{ao}{d} \ \wedge \
                (
                    (\et{d}{uo}) \ \vee \
                    (\et{d}{sc} \ \wedge \ \event{d}{W})
                )
            \]
            
        then,
            \[
                \reln{k}{hb}{d}\ \Rightarrow\ \reln{k}{hb}{e}.
            \]
    \end{lemma}
    
    %An alternative short proof 
    \begin{proof}
        We have the following to be true :
            \begin{align*}
                cons(e,d) \ \wedge \ \reln{e}{ao}{d}.
            \end{align*}
        In both cases where $d$ is unordered or a sequentially consistent write, for any event $k$
        \[
            dir(k,d)\ \Rightarrow\ cons(k,d).
        \]
        
        An event that satisfies the above with $d$ is $e$. Because $\stck{_\textit{ao}}$ is a total order, $e$ will be the only event. This would mean that for any other $k \neq e$,
        \begin{align*}
            \reln{k}{hb}{d}\ \Rightarrow\ \reln{k}{hb}{e}.
        \end{align*}
        
        Note that although there could be a direct \textit{happens-before} relation with some event $k$ from \textit{another} agent, they are only relations satisfying $dir(d,k)$.
        
    \end{proof}

%---------------------------------------------------------------------------------------------------------------    
    
%SHORTER VERSION OF PROOF WITHOUT THE ENGLISH EXPLAINATION IN THE MIDDLE. DISCUSS AND DECIDE ON WHICH FORM IS BETTER
    \begin{lemma}Consider three events $e$, $d$ and $k$. \\
    
        If
            \[
                \cons{e}{d} \ \wedge \ \reln{e}{ao}{d} \ \wedge \
                (
                    (\et{e}{uo}) \ \vee \
                    (\et{e}{sc} \ \wedge \ \event{e}{R})
                )
            \]
            
        then,
            \[
                \reln{e}{hb}{k}\ \Rightarrow\ \reln{d}{hb}{k}.
            \]
    \end{lemma}
    
    %An alternative proof for this 
    \begin{proof}
        The proof is symmetric to that of Lemma 1. 
    \end{proof}

    \emph{Note that the above lemmas are only for events $k$ which are not of type \textit{init}}
    

\subsection{Valid reordering}
    We view reordering as manipulating the agent-order relation among two events. In that sense, reordering two consecutive events $e$ and $d$ such that $e \stck{_{ao}} d$ becomes:
    \[
        e \stck{_{ao}} d 
        \longmapsto
        d \stck{_{ao}} e 
    \]

    What implications this change has on the other ordering relations depends on the type of events $e$ and $d$ are and would require an analysis on each Candidate Execution. 
    The intuition is that the axioms of the memory model rely on certain ordering relations to restrict observable behaviors in a program.
    Hence, preserving these ordering relations would help us in turn not introduce new Observable Behaviors.
    In particular we note that preserving $\stck{_{hb}}$ relations (other than the one we eliminate intentionally i.e $\reln{e}{hb}{d}$) would suffice for our purpose. 
    Since $\stck{_{mo}}$ respects $\stck{_{hb}}$, we in turn even preserve the memory order which is essential.  

    In the end, we want to ensure that the set of possible observable behaviors of a program, remain unchanged after reordering. If that is not feasible, then we would want the set of observable behaviors after reordering at the very least to be a subset. This would ensure that the program does not have some new behaviours that weren't supposed to happen prior to reordering. 
    
    We begin by first defining a reorderable pair of events. We then formulate a theorem (with a proof) on the set of observable behaviors of a Candidate before and after reordering a pair of consecutive events which are reorderable. We consider reordering valid if the set of observable behaviours after reordering are a subset of the original. 

    \begin{definition}{Reorderable Pair (Reord)}
        We define a boolean function \emph{Reord} that takes two ordered pair of events $e$ and $d$ such that $\reln{e}{ao}{d}$ and gives a boolean value indicating if they are a reorderable pair. 
        
        \begin{align*}
            Reord(e,d) = \\
            (
            &((\et{e}{uo} \ \wedge \ \et{d}{uo}) \ \wedge \ 
                    (   
                        (\event{e}{R} \ \wedge \ \event{d}{R}) \ \vee \ 
                        (\Re(e) \cap_\Re \Re(d) = \phi) 
                    )
            ) \\ &\vee \\
            &((\et{e}{sc} \ \wedge \ \et{d}{uo}) \ \wedge \ 
                    (
                        (\event{e}{W} \ \wedge \ (\Re(e) \cap_\Re \Re(d) = \phi)) 
                    )
            ) \\ &\vee \\
            &((\et{e}{uo} \ \wedge \ \et{d}{sc}) \ \wedge \ 
                    (
                        (\event{d}{R} \ \wedge \ (\Re(e) \cap_\Re \Re(d) = \phi)) 
                    )
            )
            )
        \end{align*}

        \critic{purple}{Use the latter for the purpose at the end of the proof for reordering, to emphasize how we approached each case}

         
    \end{definition}

    \begin{theorem}
    \label{WriteElim}
    Consider a candidate $C$ of a program and its possible \textit{Candidate Executions} where $\stck{_\textit{hb}}$ is strictly partial order. 
    Consider two \textbf{write} events $e$ and $d$ in $C$ such that 
    \begin{align*}
        \cons{e}{d} \ \wedge \ \reln{e}{ao}{d}. 
    \end{align*}
    Consider a Candidate $C'$ after eliminating the event $e$ from $C$.  
    If
    \begin{align*}
        \et{e}{uo} \ \wedge \ \Re(e) = \Re(d). 
    \end{align*}
    then the set of Observable behaviors of $C'$ is a subset of $C$.  
\end{theorem}

    \begin{proof}
    Once again, we look at this as a write elimination done on a Candidate Execution of $C$. We start by proving when other happens-before relations remain intact. Followed by identifying relations lost due to elimination and a proof for when these relations do not introduce new observable behaviors. 
    
   \paragraph{1. Preserving \emph{happens-before} relations}
        The relations we want to preserve are those that are dervied through relation with $e$, viz. using the following two relations:
        \begin{tasks}(2)
            \task $\reln{k}{hb}{e}$
            \task $\reln{e}{hb}{k}$
        \end{tasks}

        We can divide the events involved in the above into two sets:
        \begin{align*}
            K_b = \{k \ | \ \reln{k}{hb}{e} \}. \\
            K_a = \{k \ | \ \reln{e}{hb}{k} \}. 
        \end{align*}

        We need to ensure the following relations hold after elimination.
        \begin{align*}
            \forall k_a \in K_a \ \wedge \ \forall k_b \in K_b \ . \ \reln{k_b}{hb}{k_a}
        \end{align*}

        Similar to reordering, we need to have a valid pivot pair $<p_b, p_a>$ such that 
        \begin{align*}
            \forall k_b \neq p_b \in K_b \ . \ \reln{k_b}{hb}{p_b} \\
            \forall k_a \neq p_a \in K_a \ . \ \reln{p_a}{hb}{k_a} 
        \end{align*}

        By Lemma \ref{Lemma1}, $\et{e}{uo}$ is the only case where $p_b$ can be a valid pivot. 
        By Lemma \ref{Lemma2}, $\et{e}{uo} \ \vee \ \et{e}{sc}$ are the cases where $p_a$ can be a valid pivot. 
        We need both the above conditions to be satisfied to have a valid pivot pair. 
        Hence, $\et{e}{uo}$ is the only possibility in which a valid pivot pair can exist. 

        \critic{blue}{Put a figure here to show this pivot role.}
   
    \paragraph{2. The \emph{happens-before} relations lost}

    The relations lost are those attached to the event $e$, which are: 
    \begin{align}
        \reln{k}{hb}{e} \ \vee \ \reln{e}{hb}{k}
    \end{align}
    
    \critic{red}{Do we need to prove that these are the only relations lost? Proof part 1 implicitly shows this.}

   \paragraph{3. Presence of Cycles?}
        
Because no new $\stck{_{hb}}$ relations are introduced, and because original candidate executions have $\stck{_{hb}}$ as a strict partial order, no cycles are introduced after elimination. 

\critic{blue}{Perhaps write this argument a bit better.}

   \paragraph{4. Do the lost relations result in New Observable Behaviors?}

        To answer this, we need to see whether the relations removed had an impact on possible $\stck{_{rf}}$ relations other than those with $e$. 
        We divide our argument into two parts, viz. the two types of relations removed:
        \begin{tasks}(2)
            \task $\reln{k}{hb}{R_{uo}}$. 
            \task $\reln{R_{uo}}{hb}{k}$.
        \end{tasks}

        Figure~\ref{elim_read:case1} shows a breakdown of sub-cases for case (a), varying based
        on the nature of event $k$.
        \begin{figure}[H]
            \centering
            \includegraphics[scale=0.5]{5.Elimination/1.ValidEliminationCandidate/ReadElimProof/ProofParts/Part4_Case1.pdf}
            \caption{The impact of lost relation $\reln{k}{hb}{R_{uo}}$ on observable behaviors.}
            \label{elim_read:case1}
        \end{figure}

        Observations:
        \begin{itemize}
            \item (i) is not a pattern forbidden by the consistency rules.
            \item (ii)(a) is a pattern of Axiom \ref{CoRe}, however, only restricting $\stck{_{rf}}$ relation with $R$ and $W'$(which here is our Unordered Read)
            \item (ii)(b) is a pattern of Axiom \ref{SeqCsAt}, however, once again, only restricting $\stck{_{rf}}$ relation with $R$ and $W'$. 
        \end{itemize}

        Figure~\ref{elim_read:case2} shows a breakdown of sub-cases for case (b), varying based
        on the nature of event $k$.
        \begin{figure}[H]
            \centering
            \includegraphics[scale=0.5]{5.Elimination/1.ValidEliminationCandidate/ReadElimProof/ProofParts/Part4_Case2.pdf}
            \caption{The impact of lost relation $\reln{R_{uo}}{hb}{k}$ on observable behaviors.}
            \label{elim_read:case2}
        \end{figure}

        Observations:
        \begin{itemize}
            \item (i) is not a pattern in any Consistency rules
            \item (ii) is a pattern of Axiom \ref{CoRe}, however, only restricting $\stck{_{rf}}$ relation with $R$ and $W$
        \end{itemize}

        From the above observations, we can infer that the relations removed only have restriction on reads-from relations on the event $e$ we eliminate. 
        Thus, we can conclude that no new observable behaviors are introduced due to the removed $\stck{_{hb}}$ relations. 

\end{proof}

    \begin{corollary}
    \label{CorolWriteElim}
    Consider a Candidate C of a program and its Candidate Executions which are valid. Consider two events $e$ and $d$ both having equal ranges such that:
    \begin{align*}
        \event{e}{W} \ \wedge \ \event{d}{W} \ \wedge \ \et{e}{uo} \ \wedge \ \reln{e}{ao}{d} \ \wedge \ \neg\cons{e}{d}
    \end{align*} 
    Consider another Candidate C' without the event $e$. If
    \begin{align*}
        \forall k \ \text{s.t.} \ \reln{e}{ao}{k} \wedge \reln{k}{ao}{d} \ , \
        Reord(e, k)
    \end{align*}
    Then, the set of Observable behaviors possible in C' is a subset of C.
\end{corollary}

\begin{proof}
    We prove by induction on the number of events $k$ between $e$ and $d$. We verify that if a $j$ exists that is valid, the Observable behaviors of $C'$ is a subset of $C$.

    \paragraph{Base Case : n = 1}

        We have the case when:
        \begin{align*}
            \reln{e}{ao}{k_1} \ \wedge \ reln{k_1}{ao}{d}
        \end{align*}

        By Theorem of Reordering and Def of consecutive events and agent order, we can reorder $e$ and $k_1$, thus giving us a Candidate $C''$ with :
        \begin{align*}
            \reln{k_1}{ao}{e} \ \wedge \ \reln{e}{ao}{d}
        \end{align*}  
        whose observable behaviors are a subset of $C$.

        By Def of Consecutive instructions and Theorem of Elmination, we can eliminate $e$, thus giving us candidate $C'$  with  
        \begin{align*}
            \reln{k_1}{ao}{d}
        \end{align*} 
        whose observable behaviors are a subset of $C''$.

        By transitive property of subsets, we can conclude that the observable behaviors of $C'$ is a subset of $C$. 
    \paragraph{Inductive Case (n)}

        Let us assume that if the number of events in between are $n$, then the corollary holds. Let us consider the Candidate to be $C_n$ and corresponding candidate after elimination as $C'_n$. The observable behavior of $C'_n$ is a subset of that of $C_n$.

        If we can show the above holds true for $n+1$ events, we are done.
        
        To show this, suppose we have $C_{n+1}$ as the candidate and $C'$ as the one after elimination of $e$. 
        
        Because $\stck{_{ao}}$ is a total order, there is a total order among all $n+1$ events $k$ agent ordered between $e$ and $d$ such that we can label them $k_1, k_2 , ... , k_{n+1}$ with the following properties
        \begin{align*}
            \reln{e}{ao}{\reln{k_1}{ao}{\reln{...}{ao}{\reln{k_{n+1}}{ao}{d}}}} 
            \ \wedge \ 
            cons(e,k_1) \wedge cons(k_1, k_2) \wedge ... \wedge cons(k_{n+1}, d) 
        \end{align*}
        By Theorem of Reordering and Def of consecutive events and agent order, we can reorder $e$ and $k_1$, thus giving a corresponding candidate $C_n$ having observable behaviors as a subset of $C_{n+1}$. 

        By our inductive assumption, we have that the observable behaviors of $C'$ is a subset of $C_n$. By transitive property of subsets, we can then conclude that the observable behaviors of $C'$ are a subset of that of $C_{n+1}$.

\end{proof}

\critic{blue}{The above proof is clear, but it seems to me that I need to label all definitions and lemmas and theorems and corollary so that I can refer them here.}

    
    For cases where reordering is not safe to do, we also show counter examples of programs where new observable behaviors are introduced.
    This additionally will help gain intuition about the proof given. 
    Note that we do not show examples for cases where $\reln{d}{hb}{e}$ itself is sufficient to show a new observable behavior, as this is a trivial exercise that can be done just using sequenital programs.
    We show counterexamples where $\stck{_{hb}}$ relations lost (those accross agents specifically), could introduce new observable behaviors. 

    For all the examples we show here, we only show the ordering relations that are important to observe. 
    Putting all the relations among different events in the example will result in confusion, hence we avoid doing so. 

    \paragraph{Reads to same memory where $e$ is of type $sc$ while $d$ is of either $uo/sc$}

        The following example illustrates when reordering two reads to $x$ as per the specification of their access orders and range results in an observable behavior disallowed.

        \begin{figure}[H]
            \centering
            \includegraphics[scale=0.7]{5.InstructionReordering/4.ValidReorderingCandidate/Example0(Rsc-Ruo,sc).pdf}
            \caption{Case where a = 2 , b = 2, c = 1 is invalid due to Sequentially Consistent Atomics}
        \end{figure}
        
        The figure on the left above shows an example of a candidate where the case of outcome in the red box is not possible. 
        The figure on the right shows the Candidate Execution of such a case.
        Observations:
        \begin{itemize}
            \item We can infer from the Candidate Execution that $\reln{x=1;_{sc}}{hb}{b=x;_{sc}}$.
            \item Because $\reln{\{x=2;_{sc}\}}{sw}{\{b=x;_{sc}\}}$, this means the read value of $b$ is $2$.
            \item From the above two, we can infer $\reln{\{x=2;_{sc}\}}{mo}{\{x=2;_{sc}\}}$.
            \item We can then also infer that $\reln{\{x=1;_{sc}\}}{hb}{\{c=x;_{uo/sc}\}}$ and $\reln{\{x=2;_{sc}\}}{hb}{\{b=x;_{uo/sc}\}}$
            \item By Axiom \ref{SeqCsAt} pattern 1 and 3, the read value for $c$ cannot be $1$.
        \end{itemize}

        \begin{figure}[H]
            \centering
            \includegraphics[scale=0.7]{5.InstructionReordering/4.ValidReorderingCandidate/Example0R(Rsc-Ruo,sc).pdf}
            \caption{Case where the reads are reordered and a = 2 , b = 2, c = 1 is valid}
        \end{figure}

        The figure on the right shows the program after reordering the two reads in $T2$, where the case of reads in the orange box is possible. 
        The figure on the left shows the Candidate Execution of such a case.
        Observations
        \begin{itemize}
            \item We can infer from the Candidate Execution that $\reln{\{x=1;_{sc}\}}{hb}{\{c=x;_{uo/sc}\}}$.
            \item No Axiom has restrictions on $\stck{_{rf}}$ between the above two events.
            \item Hence, the read value of $c$ can be $1$.
            \item Further, the memory order is not inferred yet\footnotemark, hence, the read value for $b$ can be $2$.
            \item Hence the reordering of the two reads is invalid. 
        \end{itemize}

        \footnotetext{Note that if the memory order was reversed in the original candidate execution, Axiom \ref{SeqCsAt} would restrict the value of $b$ to be $1$. Since this is not possible due to the synchronized relation established, it must be the case that $x=1$ is memory ordered before $x=2$.}
        
%---------------------------------------------------------------------------------------------------------------------------------------
    
    \paragraph{A Read $e$ of type $sc$ followed by a Write of either $uo/sc$}
        
        The following is an example of a program with a sequentially consistent read followed by a write of any type. 
        \begin{figure}[H]
            \centering
            \includegraphics[scale=0.7]{5.InstructionReordering/4.ValidReorderingCandidate/Example3(Rsc-Wuo,sc).pdf}
            \caption{Case where a = 1 and b = 1 is invalid due to Coherent Reads.}
        \end{figure}
        The figure on the left above shows an example of a candidate where the case of reads in the red box is not possible. 
        The figure on the right shows the Candidate Execution of such a case. 
        Observations:
        \begin{itemize}
            \item From the Candidate Execution, we can infer $\reln{\{b=y_{uo/sc}\}}{hb}{\{y=1_{uo/sc}\}}$
            \item By Axiom \ref{CoRe}, $b$ cannot read the value of $1$ as $y$. 
            \item This inference was due to $\reln{\{x=1_{sc}\}}{hb}{\{a=x{sc}\}}$
        \end{itemize}

        \begin{figure}[H]
            \centering
            \includegraphics[scale=0.7]{5.InstructionReordering/4.ValidReorderingCandidate/Example3R(Rsc-Wuo,sc).pdf}
            \caption{Case where events of T1 are reordered, resulting in  a = 1 and b = 1 to be valid.}
        \end{figure}
        The figure on the right above shows the program after reordering the two events in $T1$ where case of reads in the orange box is possible. 
        The figure on the left shows the Candidate Execution of such a case. 
        Observations:
        \begin{itemize}
            \item From the Candidate Execution, we can infer $\neg\reln{\{b=y_{uo/sc}\}}{hb}{\{y=1_{uo/sc}\}}$
            \item Since there is no $\stck{_{hb}}$ relation among the above two events, $b$ can read the value of $y$ as $1$.
        \end{itemize}

%--------------------------------------------------------------------------------------------------------------------------------------        
    \paragraph{A Read $e$ of type $uo$ followed by a write $d$ of type $sc$}

        For this we can use the same example for the previous part (tag figure of example), where we just reorder $T2$'s events.
        \begin{figure}[H]
            \centering
            \includegraphics[scale=0.7]{5.InstructionReordering/4.ValidReorderingCandidate/Example4(Ruo-Wsc).pdf}
            \caption{Case where a = 1 and b = 1 is invalid due to Coherent Reads.}
        \end{figure}

        \begin{figure}[H]
            \centering
            \includegraphics[scale=0.7]{5.InstructionReordering/4.ValidReorderingCandidate/Example4R(Ruo-Wsc).pdf}
            \caption{Case where events of T2 are reordered, resulting in  a = 1 and b = 1 to be valid.}
        \end{figure}

%---------------------------------------------------------------------------------------------------------------------------------------
        
    \paragraph{A Write $e$ followed by a Read $d$ both of type $sc$}
        
        A counter example for this is different. It is not the Observable Behavior we are concerned with that is introduced, but that which is allowed but creates a $\stck{_{hb}}$ cycle. The following example is as such:
        \begin{figure}[H]
            \centering
            \includegraphics[scale=0.7]{5.InstructionReordering/4.ValidReorderingCandidate/Example5(Wsc-Rsc).pdf}
            \caption{Case where a = 1 and b = 1 is valid and no happens-before cycles}
        \end{figure}

        After reordering the two events of $T1$ in the above example, the same observable behavior holds, but has a cycle introduced. One might think that simply discarding that execution would do. But this would mean discarding $\stck{_{hb}}$ relations also, which would require more information to infer which relations are going to create such cycles and which are not. Since we place no assumptions on these relations, but that any happens-before relation other than the one we remove explicitly be reordering are all possible. Hence, the following reordered program outcome is something we do not risk to allow.

        \begin{figure}[H]
            \centering
            \includegraphics[scale=0.7]{5.InstructionReordering/4.ValidReorderingCandidate/Example5R(Wsc-Rsc).pdf}
            \caption{Case where a = 1 and b = 1 is creates a happens-before cycle}
        \end{figure}

        Observation:
        \begin{itemize}
            \item From the read values we can infer that the Candidate Execution should have $\reln{\{x=1_{sc}\}}{hb}{\{a=x_{sc}\}}$ and $\reln{\{y=1_{sc}\}}{hb}{\{a=y_{sc}\}}$.
            \item The above relations create the cycle $\reln{\{a=y_{sc}\}}{hb}{\reln{\{x=1_{sc}\}}{hb}{\reln{\{a=x_{sc}\}}{hb}{\reln{\{y=1_{sc}\}}{hb}{\{a=y_{sc}\}}}}}$.
            \item This execution is invalid. 
        \end{itemize}

%---------------------------------------------------------------------------------------------------------------------------------------

    \paragraph{A Write $e$ of type $uo/sc$ followed by a Write $d$ of type $sc$}
        
        The following example shows a program with a thread having a write of any access mode($uo/sc$) followed by a write of type $sc$.
        \begin{figure}[H]
            \centering
            \includegraphics[scale=0.7]{5.InstructionReordering/4.ValidReorderingCandidate/Example7(Wuo,sc-Wsc).pdf}
            \caption{Case where a = 0 and b = 1 is invalid due to Coherent Reads.}
        \end{figure}
        The figure on the left above shows an example of a candidate where the case of reads in the red box is not possible. 
        The figure on the right shows the Candidate Execution of such a case. 
        Observations:
        \begin{itemize}
            \item From the Candidate Execution, we can infer $\reln{\{x=0_{init}\}}{hb}{\reln{\{x=1_{uo/sc}\}}{hb}{\{a=x_{uo/sc}\}}}$
            \item By Axiom \ref{CoRe}, the read of $a$ cannot have the value of $x$ read as $0$. 
            \item This inference was due to $\reln{\{y=1_{sc}\}}{hb}{\{b=y_{sc}\}}$.
        \end{itemize}

        \begin{figure}[H]
            \centering
            \includegraphics[scale=0.7]{5.InstructionReordering/4.ValidReorderingCandidate/Example7R(Wuo,sc-Wsc).pdf}
            \caption{Case where events of T1 are reordered, resulting in  a = 0 and b = 1 to be valid.}
        \end{figure}
        
        The figure on the right above shows the program after reordering the two events in $T1$ where case of reads in the orange box is possible. 
        The figure on the left shows the Candidate Execution that explains the orange box case. 
        Observations:
        \begin{itemize}
            \item From the Candidate Execution, we can infer $\neg\reln{\{x=1_{uo/sc}\}}{hb}{\{a=x_{uo/sc}\}}$
            \item There is no pattern that the Axioms restrict, thus validating $x$ to be read as $0$ by $a$. 
        \end{itemize}

    \subsection{From Candidates to Program}

    Insights:
    \begin{itemize}
        \item At the program level, what is done in general is code motion. 
        \item We can classify differnt cases of code motion. 
        \item The two parts would be that of conditionals and that of loops.
        \item We still did not define general reordering, but we viewed it in terms of agent order to be general enough. 
    \end{itemize}

    General Corollary defining code motion in a Candidate, not a program. 
    \begin{corollary}
        Consider a Candidate C of a program and its Candidate Executions which are valid. Consider a set of events ${k_1, k_2, ... k_n}$ such that 
        \begin{align}
            \forall i \in [1,n-1] \ . \ \reln{k_i}{ao}{k_{i+1}} \wedge \cons{k_i}{k_{i+1}} 
        \end{align}
        Consider an event $e$ such that 
        \begin{align}
            \cons{e}{k_1} \ \wedge \ \reln{e}{ao}{d}  
        \end{align}
        Consider another candidate $C'$ with the only differnence from $C$ being that
        \begin{align}
            \cons{e}{k_n} \ \wedge \ \reln{k_n}{ao}{e}
        \end{align}
        Then the set of observable behaviors of $C'$ is a subset of that of $C$ only if 
        \begin{align}
            \forall i \in [1,n] \ . \ Reord(e,k_i)
        \end{align} 
    \end{corollary}

    \begin{proof}
        Apply theorem of reordering successively, and by transititvity of subset relations, the corollary holds.
    \end{proof}

    \critic{blue}{The above corollary we defined is the most general form of reordering, which also defines code motion. It is interesting to note that we first noted reordering as reversing agent order, but we never considered it as strictly as we should have.}

    \critic{red}{Make the corollary more precise, it can be simplified greatly.}
    
    Preliminaries needed for conditionals:
    \begin{itemize}
        \item One conditional results in two possible candidates. Describe formally that the two candidates are distinct, in that there are some events that aren't present in each of them, but are present in the other. 
        \item There are two types of conditionals; need to have two defintions for conditionals then. 
    \end{itemize}

    Perhaps we need a general corollary for program level 
    \begin{corollary}{Reordering under Program with Conditionals}
        Consider a program $P$ and its candidates $C_1, C_2, ... , C_n$ in which events $e$ and $d$ present in all of them with $\reln{e}{ao}{d}$. Consider the set of corresponding candidates $C'_1, C'_2, ... , C'_n$ after reordering $e$ and $d$ and its corresponding program $P'$. Then the set of observable behaviors of $P'$ is a subset of that of $P$ if the following conditions hold:
        \begin{align*}
            Reord(e,d) \ \wedge \ \\ 
            \forall C_{i \in [1,n]} \ , \ \forall k \in C_i \ \text{s.t.} \ \reln{e}{ao}{k} \wedge \reln{k}{ao}{d} \\ 
            Reord(e,k) \wedge Reord(k,d) 
        \end{align*}
        \critic{red}{The above condition can be simplified as we already have corollary to show reordering of non-consecutive events}
        \critic{blue}{No Candidate of $P$ exists such that only one of $e$ or $d$ exists in them. }
        \begin{align*}
            \nexists C \in P \ s.t. \ 
                (e \in C \ \wedge \ d \notin C) \ \vee \ 
                (e \notin C \ \wedge \ d \in C) 
        \end{align*}
    \end{corollary}

    \begin{proof}
        The proof would go as follows:
        \begin{itemize}
            \item By property of conditionals, there could be a candidate with $e$ existing but $d$ not. This would mean that $d$ is a part of conditional brnaching.
            \item Reordering would then result in removing certain agent order relations with $e$ and adding the same with $d$.
            \item Removal occurs because after reordering, $P'$ will have candidates with $d$ existing but not $e$. 
            \item Adding occurs in the same fashion.
            \item Removal in this case may result in new observable behaviors with events outside the same agent
            \item Adding may also introduce new observable behaviors with events outside the same agent. 
            \item The new observable behaviors can occur due to the other conditional branch having some sc events that syncronize with other threads.
            \item Maybe finding SC events in the other branch may not be so difficult to do. But perhaps if it is difficult, then reordering should not be done. 
            \item Perhaps we could say that those events with whome $d$ has now relations, if it is the case that it has a new relation with some $k$ thats of type $sc$ (perhaps we can narrow it down to a read / write), then we shouldnt be allowed to do reordering. 
        \end{itemize}
    \end{proof}

    \critic{blue}{It has come to my notice that in general reordering may not be fine, if we end up removing something outside a conditional. To show this I have a simple counter example. However, the only way to show this, is to say that there is some candidate which suddenly has an agent order relation between two events which were not supposed to have any relation to them. Simply put, we can say that there is a candidate execution of the reordered program, where both the events exist, where as there isn't any candidate of the original program where such a thing can happen.}

\subsection{}

    

    
    
    
    
    