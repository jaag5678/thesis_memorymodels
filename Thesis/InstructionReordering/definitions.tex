\subsection{Preliminaries}
Before we go about proving when reordering is valid, we would like to have two additional definitions which would prove useful.

%Something we need to define for sake of proofs
\begin{definition}{Consecutive pair of events (\emph{cons})}
    
    We define \emph{cons} as a function, which takes two events as input, and gives us a boolean indicating if they are consecutive pairs. Two events $e$ and $d$ are consecutive if they have an $\stck{_\textit{ao}}$ relation among them and are \emph{next to each other}, which can be defined formally as 
        \begin{align*}
            (
            e \stck{_\textit{ao}} d  \ \wedge \ 
            \nexists k \ \textit{s.t.} \ 
            e \stck{_\textit{ao}} k  \ \wedge \
            k \stck{_\textit{ao}} d 
            )
            \ \vee \
            (
                d \stck{_\textit{ao}} e  \ \wedge \ 
                \nexists k \ \textit{s.t.} \ 
                d \stck{_\textit{ao}} k  \ \wedge \
                k \stck{_\textit{ao}} e  
            )
        \end{align*}
\end{definition}

\begin{definition}{Direct happens-before relation (dir)}
    
    We define \emph{dir} to take an ordered pair of events $(e,d)$ such that $\reln{e}{hb}{d}$ and gives a boolean value to indicate whether this relation is \textit{direct}, i.e those relations that are not derived through transitive property of $\stck{_\textit{hb}}$.
    
    We can infer certain things using this function based on some information on events $e$ and $d$. 
    \begin{itemize}
        \item If $\et{e}{uo}$, then $dir(e,d) \ \Rightarrow \ cons(e,d)$
        \item If $\et{d}{uo}$, then $dir(e,d) \ \Rightarrow \ cons(e,d)$
        \item If $\et{e}{sc}\ \wedge\ e\!\in\!R$, then $dir(e,d) \ \Rightarrow \ cons(e,d)$
        \item If $\et{e}{sc}\ \wedge\ e\!\in\!W$, then $dir(e,d) \ \Rightarrow \ cons(e,d)\ \vee\ \reln{e}{sw}{d}$
        \item If $\et{d}{sc}\ \wedge\ d\!\in\!W$, then $dir(e,d) \ \Rightarrow \ cons(e,d)$
        \item If $\et{d}{sc}\ \wedge\ e\!\in\!R$, then $dir(e,d) \ \Rightarrow \ cons(e,d)\ \vee\ \reln{e}{sw}{d}$
    \end{itemize}
\end{definition}



