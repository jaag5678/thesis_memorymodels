This chapter adddresses the validity of instruction reordering under the ECMAScript memory model.
We first start by showing some examples of Candidate Executions where reordering, while sequentially sane to do, it not safe in the relaxed memory context.  
We then summarize our approach towards a proof to identify when such a reordering is safe with respect to shared memory accesses. 
Next, we introduce two more defintions for our purpose followed by two basic lemmas that will be instrumental for proofs in this chapter and the next. 
Next, we formulate a theorem and a corresponding corollary to assess validity of reorderinng at a Candidate Execution level. 
Lastly, we address conservatively at the program level (still abstracted to a set of shared memory events) involving loops and conditional branching.
Throughout this chapter, we use counter examples to give a better intuitive understanding of the elements of the proof as well as the advantage of our formal model in Chapter 3.
\ \newline
\ \newline  
\hrule 
\ \newline 
\ \newline 


The model we consider is the current draft specification \cite{ECMA} of the ECMAScript standard. 
The semantics of the model we consider has remain unchanged since the time we started our investigation (2019), so we believe our work will also be of use to those working on it. 
The specification is claimed to be \textit{axiomatic} by definition, which should, in our view remove the complexities of the rest of the standard from the semantics of the model.
However, there are some concerns with it: 

\paragraph{The Model is Quite Algorithmic}
    Although the standard states that the model is not supposed to be operational, the specifications of the model hint otherwise. 
    They are defined as relational constraints on certain \textit{abstract operations} which are not necessary to understand the semantics of the model.  
    As an example, consider one of the \textit{axioms} of the model in Figure~\ref{model:Std1} as stated by the standard. 
    \begin{figure}[H]
        \centering 
        \includegraphics[scale=0.6]{3.ECMAScriptMemoryModel/ECMAScriptStdCoherentReads.pdf}
        \caption{The ECMAScript specification for Coherent Reads.}
        \label{model:Std1}
    \end{figure}
    The definition in Figure~\ref{model:Std1} is specified in terms of a return value from an abstract operation. 
    Understanding this requires one to know the definitions for \textit{Ws, execution, SharedDataBlockEventSet}, etc. although this is not required to understand what the axiom is about, which informally can be stated as below in two points:
    \begin{itemize}
        \item A read's value cannot come from a write that has happened after it. 
        \item A read's value cannot come from a write that has been overwritten by some other write.  
    \end{itemize}
    Axiomatically, we define the above two constraints using binary relations that we derive (also in some sense, take directly) from the specification in Section 2 of this chapter. 
    
\paragraph{Certain Unnecessary Definitions}
    Certain abstract operations are not required to capture the semantics of the model. 
    One such example is shown in Figure~\ref{model:Std2}
    \begin{figure}[H]
        \centering 
        \includegraphics[scale=0.6]{3.ECMAScriptMemoryModel/ECMAScriptStd.pdf}
        \caption{The ECMAScript specification for Compose Write Event Bytes \cite{ECMA}.}
        \label{model:Std2}
    \end{figure}
    Figure~\ref{model:Std2} is the definition of an abstract operation. 
    Understanding this operation would require the meaning of the terms \textit{ModifyOp, Payload, Ws} and \textit{ByteIndex}. 
    In its essence, this operation determines the read-values read by a single event by collecting the values from their corresponding writes. 
    We noticed that one need not know this operation nor understand its function as it is not necessary in the axiomatic semantics of the model. 
    Other such abstract operations which may not be essential are \textit{ValueOfReadEvent} and \textit{ValidChosenReads}\cite{ECMA}. 

\paragraph{Still a bit verbose}
    
    The entire model, though algorithmic in its structure, is still quite verbose in its details, which makes it difficult to understand the model semantics. 
    Figure~\ref{model:Std3} is another \textit{axiom} from the standard. 
    \begin{figure}[H]
        \centering 
        \includegraphics[scale=0.6]{3.ECMAScriptMemoryModel/ECMAScriptStdSeqCnsAt.pdf}
        \caption{The ECMAScript specification for Sequentially Consistent Atomics axiom.}
        \label{model:Std3}
    \end{figure}
    The definition in Figure~\ref{model:Std3}, is not concise enough to reason about it mathematically. 
    In addition, the part after Note1 in Figure~\ref{model:Std3} is not a semantic specification, rather a programming guideline while using 
    atomic memory accesses. 
    We will reduce the above entire axiom into three main patterns using binary relations.

Given the above concerns about the specification in the standard, we found the need to have a concise formal description of the model. 
In the following sections, we define what agents and events are, followed by several binary relations among different events.


%Summary of Approach.
\section{Approach}

    We consider the same set of assumptions for reordering here. 
    Similar to reordering, our main objective is to ensure that the set of possible observable behaviors of a program, remain unchanged after elimination. 
    In the case of read elimination, we would want the observable behaviors apart from the specific read eliminated to be a subset.
    In the case of write elimination, there is no such constraint as that for reads.
    For both cases, if preserving all behaviors is not possible, then we would want the set of observable behaviors after elimination at the very least to be a subset.

    The main difference here is that elimination would remove certain happens-before relations, in contrast to having additional ones.
    From our point of view, we would want only the relations with the eliminated read/write to be removed after the transformation.
    The loss of these relations would certainly not have any new happens-before cycle to be introduced. 
    However, we still have to check whether the removed relations result in some new behavior. 
    We prove when it does not, by doing case-wise analysis on the type of relations eliminated.  

    For addressing the validity of eliminations under our memory model, we separately address first Read elimination, followed by Write elimination, both at the candidate level. 
    We finally address them at the program level with conditionals and loops involved. 

%Key definitions 
\section{Some Useful Definitions}
Before we go about proving when reordering is valid, we would like to have two additional definitions which would prove useful\footnotemark.

\footnotetext{The following definitions and lemmas are not particular to instruction reordering, so I think we can make it a point to put this in a section that introduces our work on optimizations.}

%Something we need to define for sake of proofs
\begin{definition}{Consecutive pair of events (\emph{cons})}
    \label{Cons}
    We define \emph{cons} as a function, which takes two events as input, and gives us a boolean indicating if they are consecutive pairs. Two events $e$ and $d$ are consecutive if they have an $\stck{_\textit{ao}}$ relation among them and are \emph{next to each other}, which can be defined formally as 
        \begin{align*}
            (
            e \stck{_\textit{ao}} d  \ \wedge \ 
            \nexists k \ \textit{s.t.} \ 
            e \stck{_\textit{ao}} k  \ \wedge \
            k \stck{_\textit{ao}} d 
            )
            \ \vee \
            (
                d \stck{_\textit{ao}} e  \ \wedge \ 
                \nexists k \ \textit{s.t.} \ 
                d \stck{_\textit{ao}} k  \ \wedge \
                k \stck{_\textit{ao}} e  
            )
        \end{align*}
\end{definition}

\begin{definition}{Direct happens-before relation (dir)}
    \label{Dir}
    We define \emph{dir} to take an ordered pair of events $(e,d)$ such that $\reln{e}{hb}{d}$ and gives a boolean value to indicate whether this relation is \textit{direct}, i.e those relations that are not derived through transitive property of $\stck{_\textit{hb}}$.
    
    We can infer certain things using this function based on some information on events $e$ and $d$. 
    \begin{itemize}
        \item If $\et{e}{uo}$, then $dir(e,d) \ \Rightarrow \ cons(e,d)$
        \item If $\et{d}{uo}$, then $dir(e,d) \ \Rightarrow \ cons(e,d)$
        \item If $\et{e}{sc}\ \wedge\ e\!\in\!R$, then $dir(e,d) \ \Rightarrow \ cons(e,d)$
        \item If $\et{e}{sc}\ \wedge\ e\!\in\!W$, then $dir(e,d) \ \Rightarrow \ cons(e,d)\ \vee\ \reln{e}{sw}{d}$
        \item If $\et{d}{sc}\ \wedge\ d\!\in\!W$, then $dir(e,d) \ \Rightarrow \ cons(e,d)$
        \item If $\et{d}{sc}\ \wedge\ e\!\in\!R$, then $dir(e,d) \ \Rightarrow \ cons(e,d)\ \vee\ \reln{e}{sw}{d}$
    \end{itemize}
\end{definition}


\begin{definition}{Reorderable Pair (Reord)}
    \label{Reord}
    
    We define a boolean function \emph{Reord} that takes two ordered pair of events $e$ and $d$ such that $\reln{e}{ao}{d}$ and gives a boolean value indicating if they are a reorderable pair. 
    
    \begin{align*}
        Reord(e,d) = \\
        (
        &((\et{e}{uo} \ \wedge \ \et{d}{uo}) \ \wedge \ 
                (   
                    (\event{e}{R} \ \wedge \ \event{d}{R}) \ \vee \ 
                    (\Re(e) \cap_\Re \Re(d) = \phi) 
                )
        ) \\ &\vee \\
        &((\et{e}{sc} \ \wedge \ \et{d}{uo}) \ \wedge \ 
                (
                    (\event{e}{W} \ \wedge \ (\Re(e) \cap_\Re \Re(d) = \phi)) 
                )
        ) \\ &\vee \\
        &((\et{e}{uo} \ \wedge \ \et{d}{sc}) \ \wedge \ 
                (
                    (\event{d}{R} \ \wedge \ (\Re(e) \cap_\Re \Re(d) = \phi)) 
                )
        )
        )
    \end{align*}

\end{definition}



%Key Lemmas 

  
   
   Our analysis is based on this corrected model by $WATTTTTT$ which is incorporated in the ECMAScript draft specification. As far as our knowledge goes, no analysis has been done on this model to identify its implications on standard compiler optimizations. 

%Valid reordering at the Candidate Execution level

  
   
   Our analysis is based on this corrected model by $WATTTTTT$ which is incorporated in the ECMAScript draft specification. As far as our knowledge goes, no analysis has been done on this model to identify its implications on standard compiler optimizations. 

%From Candidates to Programs

  
   
   Our analysis is based on this corrected model by $WATTTTTT$ which is incorporated in the ECMAScript draft specification. As far as our knowledge goes, no analysis has been done on this model to identify its implications on standard compiler optimizations. 

%Conclusion (write here itself. No need for a new .tex file)
\ \newline
\ \newline  
\hrule 
\ \newline 
\ \newline 
To summarize, this chapter addressed the validity of instruction reordering under the ECMAScript Memory Model. 
We first built a conservative proof for reordering based on candidate executions.
We later extended it to programs abstracted to the set of shared memory events. 
We discussed throughout the limitation and advantages of our conservative approach. 
We also presented examples throughout this chapter to get a fair intuitive understanding of the ideas behind the proof and the role of the axiomatic model in it.

In the next chapter, we will address the validity of elmination udner the ECMAScript Memory Model.