\section{Summary of Our Approach}

    An example-based analysis exposes to us the problems that might exist when we perform such reordering of events. 
    However, such an analysis, though would work for small programs, become infeasible as the programs scale in length and complexity. 
    This is because of the exponential increase in possible outcomes as the number of threads and program size increases. 
    Hence, generalizations by using a small examples is not something we can afford especially when we want to ensure these program transformations are automated in contrast to being done manually.
    
    Our solution to this is to construct a proof at the candidate level, which can expose for Candidate Executions of the original program and the transformed one the possible observable behaviors it can have.   
    The crux of the proof is to guarantee that reordering does not bring any new $\stck{_{rf}}$ (reads-from) relations that did not exist in any Observable Behavior of the original Candidate's set of Candidate Executions. 
    
    \paragraph{Assumption}
    We make the following assumptions for every program we consider:
    \begin{enumerate}
        \item All events are tear-free.
        \item No synchronize events exist.
        \item No Read-Modify-Write events exist.
        \item All executions of the candidate before reordering have happens-before as a strict partial order.
    \end{enumerate}
    
    We first consider when consecutive events in the same agent can be reordered, followed by non-consecutive cases. 
    We view reordering as manipulating the agent-order relation. 
    In that sense, reordering two consecutive events $e$ and $d$ such that $e \stck{_{ao}} d$ becomes:
    \[
        e \stck{_{ao}} d 
        \longmapsto
        d \stck{_{ao}} e 
    \]

    What implications this change has on the Observable Behaviors depends on the type of events $e$ and $d$ are.
    The intuition is that the axioms of the memory model rely on certain ordering relations to restrict observable behaviors in a program.
    Hence, preserving these ordering relations would help us in turn not introduce new Observable Behaviors.
    In particular we note that preserving $\stck{_{hb}}$ relations (other than the one we eliminate intentionally i.e $\reln{e}{hb}{d}$) would suffice for our purpose. 
    Since $\stck{_{mo}}$ respects $\stck{_{hb}}$, we in turn even preserve memory orders which are essential.  
    

