
\section{Valid reordering}
    

Our main objective is to ensure that the set of possible observable behaviors of a program, remain unchanged after reordering. If that is not feasible, then we would want the set of observable behaviors after reordering at the very least to be a subset. This is to ensure that the Candidate of a program does not have some new behaviours that weren't supposed to happen prior to reordering.   

    \subsection{Reordering of Consecutive Events}
        \begin{theorem}
    \label{WriteElim}
    Consider a candidate $C$ of a program and its possible \textit{Candidate Executions} where $\stck{_\textit{hb}}$ is strictly partial order. 
    Consider two \textbf{write} events $e$ and $d$ in $C$ such that 
    \begin{align*}
        \cons{e}{d} \ \wedge \ \reln{e}{ao}{d}. 
    \end{align*}
    Consider a Candidate $C'$ after eliminating the event $e$ from $C$.  
    If
    \begin{align*}
        \et{e}{uo} \ \wedge \ \Re(e) = \Re(d). 
    \end{align*}
    then the set of Observable behaviors of $C'$ is a subset of $C$.  
\end{theorem}

        \begin{proof}
    Once again, we look at this as a write elimination done on a Candidate Execution of $C$. We start by proving when other happens-before relations remain intact. Followed by identifying relations lost due to elimination and a proof for when these relations do not introduce new observable behaviors. 
    
   \paragraph{1. Preserving \emph{happens-before} relations}
        The relations we want to preserve are those that are dervied through relation with $e$, viz. using the following two relations:
        \begin{tasks}(2)
            \task $\reln{k}{hb}{e}$
            \task $\reln{e}{hb}{k}$
        \end{tasks}

        We can divide the events involved in the above into two sets:
        \begin{align*}
            K_b = \{k \ | \ \reln{k}{hb}{e} \}. \\
            K_a = \{k \ | \ \reln{e}{hb}{k} \}. 
        \end{align*}

        We need to ensure the following relations hold after elimination.
        \begin{align*}
            \forall k_a \in K_a \ \wedge \ \forall k_b \in K_b \ . \ \reln{k_b}{hb}{k_a}
        \end{align*}

        Similar to reordering, we need to have a valid pivot pair $<p_b, p_a>$ such that 
        \begin{align*}
            \forall k_b \neq p_b \in K_b \ . \ \reln{k_b}{hb}{p_b} \\
            \forall k_a \neq p_a \in K_a \ . \ \reln{p_a}{hb}{k_a} 
        \end{align*}

        By Lemma \ref{Lemma1}, $\et{e}{uo}$ is the only case where $p_b$ can be a valid pivot. 
        By Lemma \ref{Lemma2}, $\et{e}{uo} \ \vee \ \et{e}{sc}$ are the cases where $p_a$ can be a valid pivot. 
        We need both the above conditions to be satisfied to have a valid pivot pair. 
        Hence, $\et{e}{uo}$ is the only possibility in which a valid pivot pair can exist. 

        \critic{blue}{Put a figure here to show this pivot role.}
   
    \paragraph{2. The \emph{happens-before} relations lost}

    The relations lost are those attached to the event $e$, which are: 
    \begin{align}
        \reln{k}{hb}{e} \ \vee \ \reln{e}{hb}{k}
    \end{align}
    
    \critic{red}{Do we need to prove that these are the only relations lost? Proof part 1 implicitly shows this.}

   \paragraph{3. Presence of Cycles?}
        
Because no new $\stck{_{hb}}$ relations are introduced, and because original candidate executions have $\stck{_{hb}}$ as a strict partial order, no cycles are introduced after elimination. 

\critic{blue}{Perhaps write this argument a bit better.}

   \paragraph{4. Do the lost relations result in New Observable Behaviors?}

        To answer this, we need to see whether the relations removed had an impact on possible $\stck{_{rf}}$ relations other than those with $e$. 
        We divide our argument into two parts, viz. the two types of relations removed:
        \begin{tasks}(2)
            \task $\reln{k}{hb}{R_{uo}}$. 
            \task $\reln{R_{uo}}{hb}{k}$.
        \end{tasks}

        Figure~\ref{elim_read:case1} shows a breakdown of sub-cases for case (a), varying based
        on the nature of event $k$.
        \begin{figure}[H]
            \centering
            \includegraphics[scale=0.5]{5.Elimination/1.ValidEliminationCandidate/ReadElimProof/ProofParts/Part4_Case1.pdf}
            \caption{The impact of lost relation $\reln{k}{hb}{R_{uo}}$ on observable behaviors.}
            \label{elim_read:case1}
        \end{figure}

        Observations:
        \begin{itemize}
            \item (i) is not a pattern forbidden by the consistency rules.
            \item (ii)(a) is a pattern of Axiom \ref{CoRe}, however, only restricting $\stck{_{rf}}$ relation with $R$ and $W'$(which here is our Unordered Read)
            \item (ii)(b) is a pattern of Axiom \ref{SeqCsAt}, however, once again, only restricting $\stck{_{rf}}$ relation with $R$ and $W'$. 
        \end{itemize}

        Figure~\ref{elim_read:case2} shows a breakdown of sub-cases for case (b), varying based
        on the nature of event $k$.
        \begin{figure}[H]
            \centering
            \includegraphics[scale=0.5]{5.Elimination/1.ValidEliminationCandidate/ReadElimProof/ProofParts/Part4_Case2.pdf}
            \caption{The impact of lost relation $\reln{R_{uo}}{hb}{k}$ on observable behaviors.}
            \label{elim_read:case2}
        \end{figure}

        Observations:
        \begin{itemize}
            \item (i) is not a pattern in any Consistency rules
            \item (ii) is a pattern of Axiom \ref{CoRe}, however, only restricting $\stck{_{rf}}$ relation with $R$ and $W$
        \end{itemize}

        From the above observations, we can infer that the relations removed only have restriction on reads-from relations on the event $e$ we eliminate. 
        Thus, we can conclude that no new observable behaviors are introduced due to the removed $\stck{_{hb}}$ relations. 

\end{proof}

    \subsection{Reordering Non-Consecutive Events}

        \begin{corollary}
    \label{CorolWriteElim}
    Consider a Candidate C of a program and its Candidate Executions which are valid. Consider two events $e$ and $d$ both having equal ranges such that:
    \begin{align*}
        \event{e}{W} \ \wedge \ \event{d}{W} \ \wedge \ \et{e}{uo} \ \wedge \ \reln{e}{ao}{d} \ \wedge \ \neg\cons{e}{d}
    \end{align*} 
    Consider another Candidate C' without the event $e$. If
    \begin{align*}
        \forall k \ \text{s.t.} \ \reln{e}{ao}{k} \wedge \reln{k}{ao}{d} \ , \
        Reord(e, k)
    \end{align*}
    Then, the set of Observable behaviors possible in C' is a subset of C.
\end{corollary}

\begin{proof}
    We prove by induction on the number of events $k$ between $e$ and $d$. We verify that if a $j$ exists that is valid, the Observable behaviors of $C'$ is a subset of $C$.

    \paragraph{Base Case : n = 1}

        We have the case when:
        \begin{align*}
            \reln{e}{ao}{k_1} \ \wedge \ reln{k_1}{ao}{d}
        \end{align*}

        By Theorem of Reordering and Def of consecutive events and agent order, we can reorder $e$ and $k_1$, thus giving us a Candidate $C''$ with :
        \begin{align*}
            \reln{k_1}{ao}{e} \ \wedge \ \reln{e}{ao}{d}
        \end{align*}  
        whose observable behaviors are a subset of $C$.

        By Def of Consecutive instructions and Theorem of Elmination, we can eliminate $e$, thus giving us candidate $C'$  with  
        \begin{align*}
            \reln{k_1}{ao}{d}
        \end{align*} 
        whose observable behaviors are a subset of $C''$.

        By transitive property of subsets, we can conclude that the observable behaviors of $C'$ is a subset of $C$. 
    \paragraph{Inductive Case (n)}

        Let us assume that if the number of events in between are $n$, then the corollary holds. Let us consider the Candidate to be $C_n$ and corresponding candidate after elimination as $C'_n$. The observable behavior of $C'_n$ is a subset of that of $C_n$.

        If we can show the above holds true for $n+1$ events, we are done.
        
        To show this, suppose we have $C_{n+1}$ as the candidate and $C'$ as the one after elimination of $e$. 
        
        Because $\stck{_{ao}}$ is a total order, there is a total order among all $n+1$ events $k$ agent ordered between $e$ and $d$ such that we can label them $k_1, k_2 , ... , k_{n+1}$ with the following properties
        \begin{align*}
            \reln{e}{ao}{\reln{k_1}{ao}{\reln{...}{ao}{\reln{k_{n+1}}{ao}{d}}}} 
            \ \wedge \ 
            cons(e,k_1) \wedge cons(k_1, k_2) \wedge ... \wedge cons(k_{n+1}, d) 
        \end{align*}
        By Theorem of Reordering and Def of consecutive events and agent order, we can reorder $e$ and $k_1$, thus giving a corresponding candidate $C_n$ having observable behaviors as a subset of $C_{n+1}$. 

        By our inductive assumption, we have that the observable behaviors of $C'$ is a subset of $C_n$. By transitive property of subsets, we can then conclude that the observable behaviors of $C'$ are a subset of that of $C_{n+1}$.

\end{proof}

\critic{blue}{The above proof is clear, but it seems to me that I need to label all definitions and lemmas and theorems and corollary so that I can refer them here.}

    \subsection{Counter Examples for all the Invalid Cases}
        
    For cases where reordering is not safe to do, we also show counter examples of programs where new observable behaviors are introduced.
    This additionally will help gain intuition about the proof given. 
    Note that we do not show examples for cases where $\reln{d}{hb}{e}$ itself is sufficient to show a new observable behavior, as this is a trivial exercise that can be done just using sequenital programs.
    We show counterexamples where $\stck{_{hb}}$ relations lost (those accross agents specifically), could introduce new observable behaviors. 

    For all the examples we show here, we only show the ordering relations that are important to observe. 
    Putting all the relations among different events in the example will result in confusion, hence we avoid doing so. 

    \paragraph{Reads to same memory where $e$ is of type $sc$ while $d$ is of either $uo/sc$}

        The following example illustrates when reordering two reads to $x$ as per the specification of their access orders and range results in an observable behavior disallowed.

        \begin{figure}[H]
            \centering
            \includegraphics[scale=0.7]{5.InstructionReordering/4.ValidReorderingCandidate/Example0(Rsc-Ruo,sc).pdf}
            \caption{Case where a = 2 , b = 2, c = 1 is invalid due to Sequentially Consistent Atomics}
        \end{figure}
        
        The figure on the left above shows an example of a candidate where the case of outcome in the red box is not possible. 
        The figure on the right shows the Candidate Execution of such a case.
        Observations:
        \begin{itemize}
            \item We can infer from the Candidate Execution that $\reln{x=1;_{sc}}{hb}{b=x;_{sc}}$.
            \item Because $\reln{\{x=2;_{sc}\}}{sw}{\{b=x;_{sc}\}}$, this means the read value of $b$ is $2$.
            \item From the above two, we can infer $\reln{\{x=2;_{sc}\}}{mo}{\{x=2;_{sc}\}}$.
            \item We can then also infer that $\reln{\{x=1;_{sc}\}}{hb}{\{c=x;_{uo/sc}\}}$ and $\reln{\{x=2;_{sc}\}}{hb}{\{b=x;_{uo/sc}\}}$
            \item By Axiom \ref{SeqCsAt} pattern 1 and 3, the read value for $c$ cannot be $1$.
        \end{itemize}

        \begin{figure}[H]
            \centering
            \includegraphics[scale=0.7]{5.InstructionReordering/4.ValidReorderingCandidate/Example0R(Rsc-Ruo,sc).pdf}
            \caption{Case where the reads are reordered and a = 2 , b = 2, c = 1 is valid}
        \end{figure}

        The figure on the right shows the program after reordering the two reads in $T2$, where the case of reads in the orange box is possible. 
        The figure on the left shows the Candidate Execution of such a case.
        Observations
        \begin{itemize}
            \item We can infer from the Candidate Execution that $\reln{\{x=1;_{sc}\}}{hb}{\{c=x;_{uo/sc}\}}$.
            \item No Axiom has restrictions on $\stck{_{rf}}$ between the above two events.
            \item Hence, the read value of $c$ can be $1$.
            \item Further, the memory order is not inferred yet\footnotemark, hence, the read value for $b$ can be $2$.
            \item Hence the reordering of the two reads is invalid. 
        \end{itemize}

        \footnotetext{Note that if the memory order was reversed in the original candidate execution, Axiom \ref{SeqCsAt} would restrict the value of $b$ to be $1$. Since this is not possible due to the synchronized relation established, it must be the case that $x=1$ is memory ordered before $x=2$.}
        
%---------------------------------------------------------------------------------------------------------------------------------------
    
    \paragraph{A Read $e$ of type $sc$ followed by a Write of either $uo/sc$}
        
        The following is an example of a program with a sequentially consistent read followed by a write of any type. 
        \begin{figure}[H]
            \centering
            \includegraphics[scale=0.7]{5.InstructionReordering/4.ValidReorderingCandidate/Example3(Rsc-Wuo,sc).pdf}
            \caption{Case where a = 1 and b = 1 is invalid due to Coherent Reads.}
        \end{figure}
        The figure on the left above shows an example of a candidate where the case of reads in the red box is not possible. 
        The figure on the right shows the Candidate Execution of such a case. 
        Observations:
        \begin{itemize}
            \item From the Candidate Execution, we can infer $\reln{\{b=y_{uo/sc}\}}{hb}{\{y=1_{uo/sc}\}}$
            \item By Axiom \ref{CoRe}, $b$ cannot read the value of $1$ as $y$. 
            \item This inference was due to $\reln{\{x=1_{sc}\}}{hb}{\{a=x{sc}\}}$
        \end{itemize}

        \begin{figure}[H]
            \centering
            \includegraphics[scale=0.7]{5.InstructionReordering/4.ValidReorderingCandidate/Example3R(Rsc-Wuo,sc).pdf}
            \caption{Case where events of T1 are reordered, resulting in  a = 1 and b = 1 to be valid.}
        \end{figure}
        The figure on the right above shows the program after reordering the two events in $T1$ where case of reads in the orange box is possible. 
        The figure on the left shows the Candidate Execution of such a case. 
        Observations:
        \begin{itemize}
            \item From the Candidate Execution, we can infer $\neg\reln{\{b=y_{uo/sc}\}}{hb}{\{y=1_{uo/sc}\}}$
            \item Since there is no $\stck{_{hb}}$ relation among the above two events, $b$ can read the value of $y$ as $1$.
        \end{itemize}

%--------------------------------------------------------------------------------------------------------------------------------------        
    \paragraph{A Read $e$ of type $uo$ followed by a write $d$ of type $sc$}

        For this we can use the same example for the previous part (tag figure of example), where we just reorder $T2$'s events.
        \begin{figure}[H]
            \centering
            \includegraphics[scale=0.7]{5.InstructionReordering/4.ValidReorderingCandidate/Example4(Ruo-Wsc).pdf}
            \caption{Case where a = 1 and b = 1 is invalid due to Coherent Reads.}
        \end{figure}

        \begin{figure}[H]
            \centering
            \includegraphics[scale=0.7]{5.InstructionReordering/4.ValidReorderingCandidate/Example4R(Ruo-Wsc).pdf}
            \caption{Case where events of T2 are reordered, resulting in  a = 1 and b = 1 to be valid.}
        \end{figure}

%---------------------------------------------------------------------------------------------------------------------------------------
        
    \paragraph{A Write $e$ followed by a Read $d$ both of type $sc$}
        
        A counter example for this is different. It is not the Observable Behavior we are concerned with that is introduced, but that which is allowed but creates a $\stck{_{hb}}$ cycle. The following example is as such:
        \begin{figure}[H]
            \centering
            \includegraphics[scale=0.7]{5.InstructionReordering/4.ValidReorderingCandidate/Example5(Wsc-Rsc).pdf}
            \caption{Case where a = 1 and b = 1 is valid and no happens-before cycles}
        \end{figure}

        After reordering the two events of $T1$ in the above example, the same observable behavior holds, but has a cycle introduced. One might think that simply discarding that execution would do. But this would mean discarding $\stck{_{hb}}$ relations also, which would require more information to infer which relations are going to create such cycles and which are not. Since we place no assumptions on these relations, but that any happens-before relation other than the one we remove explicitly be reordering are all possible. Hence, the following reordered program outcome is something we do not risk to allow.

        \begin{figure}[H]
            \centering
            \includegraphics[scale=0.7]{5.InstructionReordering/4.ValidReorderingCandidate/Example5R(Wsc-Rsc).pdf}
            \caption{Case where a = 1 and b = 1 is creates a happens-before cycle}
        \end{figure}

        Observation:
        \begin{itemize}
            \item From the read values we can infer that the Candidate Execution should have $\reln{\{x=1_{sc}\}}{hb}{\{a=x_{sc}\}}$ and $\reln{\{y=1_{sc}\}}{hb}{\{a=y_{sc}\}}$.
            \item The above relations create the cycle $\reln{\{a=y_{sc}\}}{hb}{\reln{\{x=1_{sc}\}}{hb}{\reln{\{a=x_{sc}\}}{hb}{\reln{\{y=1_{sc}\}}{hb}{\{a=y_{sc}\}}}}}$.
            \item This execution is invalid. 
        \end{itemize}

%---------------------------------------------------------------------------------------------------------------------------------------

    \paragraph{A Write $e$ of type $uo/sc$ followed by a Write $d$ of type $sc$}
        
        The following example shows a program with a thread having a write of any access mode($uo/sc$) followed by a write of type $sc$.
        \begin{figure}[H]
            \centering
            \includegraphics[scale=0.7]{5.InstructionReordering/4.ValidReorderingCandidate/Example7(Wuo,sc-Wsc).pdf}
            \caption{Case where a = 0 and b = 1 is invalid due to Coherent Reads.}
        \end{figure}
        The figure on the left above shows an example of a candidate where the case of reads in the red box is not possible. 
        The figure on the right shows the Candidate Execution of such a case. 
        Observations:
        \begin{itemize}
            \item From the Candidate Execution, we can infer $\reln{\{x=0_{init}\}}{hb}{\reln{\{x=1_{uo/sc}\}}{hb}{\{a=x_{uo/sc}\}}}$
            \item By Axiom \ref{CoRe}, the read of $a$ cannot have the value of $x$ read as $0$. 
            \item This inference was due to $\reln{\{y=1_{sc}\}}{hb}{\{b=y_{sc}\}}$.
        \end{itemize}

        \begin{figure}[H]
            \centering
            \includegraphics[scale=0.7]{5.InstructionReordering/4.ValidReorderingCandidate/Example7R(Wuo,sc-Wsc).pdf}
            \caption{Case where events of T1 are reordered, resulting in  a = 0 and b = 1 to be valid.}
        \end{figure}
        
        The figure on the right above shows the program after reordering the two events in $T1$ where case of reads in the orange box is possible. 
        The figure on the left shows the Candidate Execution that explains the orange box case. 
        Observations:
        \begin{itemize}
            \item From the Candidate Execution, we can infer $\neg\reln{\{x=1_{uo/sc}\}}{hb}{\{a=x_{uo/sc}\}}$
            \item There is no pattern that the Axioms restrict, thus validating $x$ to be read as $0$ by $a$. 
        \end{itemize}

