\begin{proof}

    We look at this in terms of performing an instruction reordering on a candidate execution of $C$. We would want the resulting candidate execution to preserve all the other $\stck{_{hb}}$ relations (except $\reln{e}{hb}{d}$) and that any new $\stck{_{hb}}$ relations strictly reduce possible observable behaviors.
    
    The proof is structured as follows. We first show that existing \textit{happens-before} relations in any candidate execution of $C$ except $\reln{e}{hb}{d}$ remain intact after reordering. We then identify the cases where new \textit{happens-before} relations could be established. We identify from these cases whether \textit{happens-before} cycles could be introduced.
    We then show for the remaining cases that new relations do not introduce any new observable behaviors.

    The above steps can be summarized as addressing four main questions for any \textit{Candidate Execution} of $C'$
    \begin{enumerate}
        \item Apart from $\reln{e}{hb}{d}$, do other \emph{happens-before} relations remain intact?
        \item Apart from $\reln{d}{hb}{e}$, are any new \emph{happens-before} relations established? 
        \item Are any \emph{happens-before} cycles introduced? 
        \item Do the new relations bring new \emph{observable behaviors?}
    \end{enumerate}
    
    \paragraph{1. Preserving \emph{happens-before} relations}
        The relations we want to preserve are those that are dervied through relation with $e$, viz. using the following two relations:
        \begin{tasks}(2)
            \task $\reln{k}{hb}{e}$
            \task $\reln{e}{hb}{k}$
        \end{tasks}

        We can divide the events involved in the above into two sets:
        \begin{align*}
            K_b = \{k \ | \ \reln{k}{hb}{e} \}. \\
            K_a = \{k \ | \ \reln{e}{hb}{k} \}. 
        \end{align*}

        We need to ensure the following relations hold after elimination.
        \begin{align*}
            \forall k_a \in K_a \ \wedge \ \forall k_b \in K_b \ . \ \reln{k_b}{hb}{k_a}
        \end{align*}

        Similar to reordering, we need to have a valid pivot pair $<p_b, p_a>$ such that 
        \begin{align*}
            \forall k_b \neq p_b \in K_b \ . \ \reln{k_b}{hb}{p_b} \\
            \forall k_a \neq p_a \in K_a \ . \ \reln{p_a}{hb}{k_a} 
        \end{align*}

        By Lemma \ref{Lemma1}, $\et{e}{uo}$ is the only case where $p_b$ can be a valid pivot. 
        By Lemma \ref{Lemma2}, $\et{e}{uo} \ \vee \ \et{e}{sc}$ are the cases where $p_a$ can be a valid pivot. 
        We need both the above conditions to be satisfied to have a valid pivot pair. 
        Hence, $\et{e}{uo}$ is the only possibility in which a valid pivot pair can exist. 

        \critic{blue}{Put a figure here to show this pivot role.}
    
    \paragraph{2. The \emph{happens-before} relations lost}

    The relations lost are those attached to the event $e$, which are: 
    \begin{align}
        \reln{k}{hb}{e} \ \vee \ \reln{e}{hb}{k}
    \end{align}
    
    \critic{red}{Do we need to prove that these are the only relations lost? Proof part 1 implicitly shows this.}

    \paragraph{3. Presence of Cycles?}
        
Because no new $\stck{_{hb}}$ relations are introduced, and because original candidate executions have $\stck{_{hb}}$ as a strict partial order, no cycles are introduced after elimination. 

\critic{blue}{Perhaps write this argument a bit better.}

    \paragraph{4. Do the lost relations result in New Observable Behaviors?}

        To answer this, we need to see whether the relations removed had an impact on possible $\stck{_{rf}}$ relations other than those with $e$. 
        We divide our argument into two parts, viz. the two types of relations removed:
        \begin{tasks}(2)
            \task $\reln{k}{hb}{R_{uo}}$. 
            \task $\reln{R_{uo}}{hb}{k}$.
        \end{tasks}

        Figure~\ref{elim_read:case1} shows a breakdown of sub-cases for case (a), varying based
        on the nature of event $k$.
        \begin{figure}[H]
            \centering
            \includegraphics[scale=0.5]{5.Elimination/1.ValidEliminationCandidate/ReadElimProof/ProofParts/Part4_Case1.pdf}
            \caption{The impact of lost relation $\reln{k}{hb}{R_{uo}}$ on observable behaviors.}
            \label{elim_read:case1}
        \end{figure}

        Observations:
        \begin{itemize}
            \item (i) is not a pattern forbidden by the consistency rules.
            \item (ii)(a) is a pattern of Axiom \ref{CoRe}, however, only restricting $\stck{_{rf}}$ relation with $R$ and $W'$(which here is our Unordered Read)
            \item (ii)(b) is a pattern of Axiom \ref{SeqCsAt}, however, once again, only restricting $\stck{_{rf}}$ relation with $R$ and $W'$. 
        \end{itemize}

        Figure~\ref{elim_read:case2} shows a breakdown of sub-cases for case (b), varying based
        on the nature of event $k$.
        \begin{figure}[H]
            \centering
            \includegraphics[scale=0.5]{5.Elimination/1.ValidEliminationCandidate/ReadElimProof/ProofParts/Part4_Case2.pdf}
            \caption{The impact of lost relation $\reln{R_{uo}}{hb}{k}$ on observable behaviors.}
            \label{elim_read:case2}
        \end{figure}

        Observations:
        \begin{itemize}
            \item (i) is not a pattern in any Consistency rules
            \item (ii) is a pattern of Axiom \ref{CoRe}, however, only restricting $\stck{_{rf}}$ relation with $R$ and $W$
        \end{itemize}

        From the above observations, we can infer that the relations removed only have restriction on reads-from relations on the event $e$ we eliminate. 
        Thus, we can conclude that no new observable behaviors are introduced due to the removed $\stck{_{hb}}$ relations. 

    The table above, precisely is the definition of a reorderable pair (after including the constraints on ranges). If we write the above table in the form of an expression we have an expanded format of our reorderable pair function. 

    \begin{align*}
        Reord(e,d) = \\
        (
        ((\et{e}{uo} \wedge \et{d}{uo}) \ \wedge \\ 
            \quad ( 
                    &(\event{e}{R} \wedge \event{d}{R}) \vee \\ 
                    &(\event{e}{W} \wedge \event{d}{R} \wedge (\Re(e) \And \Re(d) = \phi)) \vee \\
                    &(\event{e}{R} \wedge \event{d}{W} \wedge (\Re(e) \And \Re(d) = \phi)) \vee \\
                    &(\event{e}{W} \wedge \event{d}{W} \wedge (\Re(e) \And \Re(d) = \phi)) 
                )
        ) \\ \vee \\
        ((\et{e}{sc} \wedge \et{d}{uo}) \ \wedge \\
            \quad (
                    & (\event{e}{W} \wedge \event{d}{R} \wedge (\Re(e) \And \Re(d) = \phi)) \vee \\
                    & (\event{e}{W} \wedge \event{d}{W} \wedge (\Re(e) \And \Re(d) = \phi)) 
                )
        ) \\ \vee \\
        ((\et{e}{uo} \wedge \et{d}{sc}) \ \wedge \\
            \quad (
                    & (\event{e}{R} \wedge \event{d}{R} \wedge) \vee \\
                    & (\event{e}{W} \wedge \event{d}{R} \wedge (\Re(e) \And \Re(d) = \phi)) 
                )
        )
        )
    \end{align*}
        
        \qed  
\end{proof}