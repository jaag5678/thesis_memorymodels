\subsection{Addressing Programs with Loops}
        
        Addressing reordering of events in programs with loops is relatively straightforward, leaving one special case. 
        For simplicity (and also Without loss of generality), let us consider our program has only one loop.

        There will be one Candidate for each iteration of loop. For convenience, let us define $C^i$ to be a candidate of program with $i$ iterations of the same loop. Let us also define $e_l^i$ to be an event within the loop of the program which in a candidate signifies the $i^{th}$ iteration of the event. 

        Using the above notation, we have the following property for any consecutive events $e$ and $d$ belonging in a loop:
        \begin{property}
            \label{LoopReordEq}
            \begin{align*}
                \forall i \neq j, \ Reord(e_l^i, d_l^i) \ \Rightarrow \ Reord(e_l^j, d_l^j)
            \end{align*}
        \end{property}
        
        Using the above notation, we have the following corollary for reordering events $e$ and $d$ within a loop 
        
        \begin{corollary}
            Consider a program $P$ with a loop and its candidates $C^1, C^2, ... , C^n$ in which events $e$ and $d$ are parts of the loop and present in all of them with $\reln{e}{ao}{d}$. Consider the set of corresponding candidates $C'^1, C'^2, ... , C'^n$ after reordering $e$ and $d$ in $P$ and its corresponding program $P'$. If the following three conditions hold:
            \begin{gather*}
                Reord(e, d) \\ 
                \forall C^i \in P , \ \forall j \in [1,i], \forall k \ \text{s.t.} \ \reln{e^j}{ao}{k} \wedge \reln{k}{ao}{d^j} \ . \ Reord(e^j, k) \wedge Reord(k, d^j)  \\ 
                \nexists C^i \in P \ s.t. \ 
                    \forall j <= i , \ (e^j \in C \ \wedge \ d^j \notin C) \ \vee \ 
                    (e^j \notin C \ \wedge \ d^j \in C) 
            \end{gather*}
            then the set of observable behaviors of Program $P'$ is a subset of program $P$.     
        \end{corollary}

        \begin{proof}
        
            The proof for this is fairly striaghtforward. 

            Condition 1 corresponds to Theorem \ref{ThmReord}. 
            From Prop \ref{LoopReordEq}, it is sufficient to show Condition 1 to represent all events $e^i, d^i$ in a loop iteration. 
            (rephrase the last sentence)

            Condition 2 and 3 correspond to Corollary \ref{CorollReord} with a slight difference. Because we reorder $e$ and $d$ within a loop, the resultant program's Candidates $C'^i$ will have for each iteration of the loop the events $e$ and $d$ reordered within them. 
            Hence, we need to ensure that reordering is possible in every possible iteration. 
            Condition 2 and 3 is precisely the set of conditions where we can assure that such a reordering is possible in any iteration of the loop. 
            
            \critic{blue}{Since the compiler cannot practically check for all iterations the set of conditions we have, one might assume that this does not hold in practice. On the contrary, its practical application would just involve checking Reord(e,k) and $Reord(k,d)$ for all such events $k$ that can exist between $e$ and $d$. This set can be obtained using a striaghtforward flow analysis. Additionally, Condition 3 can always be checked beforehand as it corresponds to checking whether events $e$ and $d$ belong in different conditional branches.}

            \critic{red}{Discuss the proof with Clark.}
            
        \end{proof}


        \paragraph{Loop invariant code motion}
            What is not so obvious is the case when events are reordered out of the loop. We will not construct a proof for this, rather use a direct counterexample to show our point. 

            \critic{blue}{Show figure here.}

            \critic{blue}{Explain figure here.}

            The problem is that we cannot use the notion of reordering of events in a Candidate to generate the candidate that is intended. Reordering at the Candidate level does not map directly to Reordering that is done to perform something like Loop invariant code motion. 

            We will later show int he next chapter that we can infact define one of the forms of loop invariant code motion using both Reordering and Elimination at the Candidate level.
             
    