\section{Helper Definitions}
    
    Before we go into the consistency rules. we define certain preliminary definitions that create a separation based on a program, the axiomatic events and the various ordering relations defined above. This will help us understand where the consistency rules actually apply.    
    
    %What is a program 
    \begin{definition}{Program} 
        A \emph{program} is the source code without abstraction to a set of events and ordering relations. In our context, it is the original Javascript program. 
        
        %Note that here we need to supply a Javascript program using shared array buffers and atomic objects. A thing to do after we give examples for the ones below
    \end{definition}
    
%-----------------------------------------------------------------------------------------------------------------------------------------    
    %What is one run of a program to us?
    \begin{definition}{Candidate}
        A a collection of abstracted set of shared memory events of a program involved in one possible execution, with the $\stck{_\textit{ao}}$ relations. We can think of this as each thread having a set of shared memory events to run in a given intra-thread ordering. An example of a candidate is shown in figure~\ref{fig:candidate}.
        
        \begin{figure}[H]
            \centering
            \includegraphics[scale=0.7]{4.ECMAScriptMemoryModel/candidate.pdf}
            \caption{An example of a Candidate}
            \label{fig:candidate}
        \end{figure}
        
    \end{definition}

%-----------------------------------------------------------------------------------------------------------------------------------------  
    \begin{definition}{Candidate Execution}
        A Candidate with the addition of $\stck{_\textit{sw}}$, $\stck{_\textit{hb}}$ and $\stck{_\textit{mo}}$ relations. This can be viewed as the witness/justification of an actual execution of a Program. Note that there can be many Candidate Executions for a given Candidate. The following figure shows an example of a candidate execution. 
        
        \begin{figure}[H]
            \centering
            \includegraphics[scale=0.7]{4.ECMAScriptMemoryModel/CandidateExecution.pdf}
            \caption{An example of an Execution based on Candidate above}
            \label{fig:my_label}
        \end{figure}
        
    \end{definition}

 %-----------------------------------------------------------------------------------------------------------------------------------------   
    %What values are read when the program is run
    \begin{definition}{Observable Behavior}
    
        The set of pairwise $\stck{_\textit{rf}}$/$\stck{_\textit{rbf}}$ relations that result in one execution of the program. Think of this as our outcome of a program execution.
    
        \begin{figure}[H]
            \centering
            \includegraphics[scale=0.7]{4.ECMAScriptMemoryModel/Observables.pdf}
            \caption{Observable Behavior}
            \label{fig:my_label}
        \end{figure}
        
    \end{definition}

%-----------------------------------------------------------------------------------------------------------------------------------------
    
    \begin{definition}{Obs}
        We define $Obs_P, Obs_C, Obs_{CE}$ as functions that take a program, candidate and candidate execution respectively and give the set of observable behaviors possible by them. We are not concerned with the specific elements in this set, but the relation between the output of each of these functions among each other. 

        Consider a program $P$ whose candidates are $C_1, C_2, ... , C_n$. Consider for each candidate $C_i$, the candidate executions $CE_1, CE_2, ... CE_m$\footnotemark. Then, we have the following properties that hold:
        \begin{align*}
            Obs_P(P) = \cup_{i=1}^{n}Obs_C(C_i) \\ 
            Obs_C(C_i) = \cup_{j=1}^{m}Obs_C(CE_j)
        \end{align*}

        \footnotetext{Note that the variables $n$ and $m$ need not be finite.}
    \end{definition}

%-----------------------------------------------------------------------------------------------------------------------------------------
    