This chapter starts with exposing some problems with the existing specifications of the model. 
The latter part is our declarative specification of the model. 
We start by introducing \textit{agents} and \textit{event sets}, followed by various \textit{binary relations} defined between events. 
We then introduce certain helper definitions that prove useful in understanding the Axioms of the model. 
We then use the above elements to specify the \textit{axioms} of the model.
Lastly, we define \textit{races} followed by defining what a \textit{Consistent Execution} is as per the specification of the model.  
\ \newline
\ \newline  
\hrule 
\ \newline 
\ \newline 


The model we consider is the current draft specification \cite{ECMA} of the ECMAScript standard. 
The semantics of the model we consider has remain unchanged since the time we started our investigation (2019), so we believe our work will also be of use to those working on it. 
The specification is claimed to be \textit{axiomatic} by definition, which should, in our view remove the complexities of the rest of the standard from the semantics of the model.
However, there are some concerns with it: 

\paragraph{The Model is Quite Algorithmic}
    Although the standard states that the model is not supposed to be operational, the specifications of the model hint otherwise. 
    They are defined as relational constraints on certain \textit{abstract operations} which are not necessary to understand the semantics of the model.  
    As an example, consider one of the \textit{axioms} of the model in Figure~\ref{model:Std1} as stated by the standard. 
    \begin{figure}[H]
        \centering 
        \includegraphics[scale=0.6]{3.ECMAScriptMemoryModel/ECMAScriptStdCoherentReads.pdf}
        \caption{The ECMAScript specification for Coherent Reads.}
        \label{model:Std1}
    \end{figure}
    The definition in Figure~\ref{model:Std1} is specified in terms of a return value from an abstract operation. 
    Understanding this requires one to know the definitions for \textit{Ws, execution, SharedDataBlockEventSet}, etc. although this is not required to understand what the axiom is about, which informally can be stated as below in two points:
    \begin{itemize}
        \item A read's value cannot come from a write that has happened after it. 
        \item A read's value cannot come from a write that has been overwritten by some other write.  
    \end{itemize}
    Axiomatically, we define the above two constraints using binary relations that we derive (also in some sense, take directly) from the specification in Section 2 of this chapter. 
    
\paragraph{Certain Unnecessary Definitions}
    Certain abstract operations are not required to capture the semantics of the model. 
    One such example is shown in Figure~\ref{model:Std2}
    \begin{figure}[H]
        \centering 
        \includegraphics[scale=0.6]{3.ECMAScriptMemoryModel/ECMAScriptStd.pdf}
        \caption{The ECMAScript specification for Compose Write Event Bytes \cite{ECMA}.}
        \label{model:Std2}
    \end{figure}
    Figure~\ref{model:Std2} is the definition of an abstract operation. 
    Understanding this operation would require the meaning of the terms \textit{ModifyOp, Payload, Ws} and \textit{ByteIndex}. 
    In its essence, this operation determines the read-values read by a single event by collecting the values from their corresponding writes. 
    We noticed that one need not know this operation nor understand its function as it is not necessary in the axiomatic semantics of the model. 
    Other such abstract operations which may not be essential are \textit{ValueOfReadEvent} and \textit{ValidChosenReads}\cite{ECMA}. 

\paragraph{Still a bit verbose}
    
    The entire model, though algorithmic in its structure, is still quite verbose in its details, which makes it difficult to understand the model semantics. 
    Figure~\ref{model:Std3} is another \textit{axiom} from the standard. 
    \begin{figure}[H]
        \centering 
        \includegraphics[scale=0.6]{3.ECMAScriptMemoryModel/ECMAScriptStdSeqCnsAt.pdf}
        \caption{The ECMAScript specification for Sequentially Consistent Atomics axiom.}
        \label{model:Std3}
    \end{figure}
    The definition in Figure~\ref{model:Std3}, is not concise enough to reason about it mathematically. 
    In addition, the part after Note1 in Figure~\ref{model:Std3} is not a semantic specification, rather a programming guideline while using 
    atomic memory accesses. 
    We will reduce the above entire axiom into three main patterns using binary relations.

Given the above concerns about the specification in the standard, we found the need to have a concise formal description of the model. 
In the following sections, we define what agents and events are, followed by several binary relations among different events.


%AGENTS----------------------------------------------------------------------------------------------------------------------------------------  
\section{Agents}

    Agents represent threads in a concurrent program. As per the standard, they have more meaning\cite{ECMA} than what we refer to here. 
    However, with respect to the memory consistency model, we can safely abstract them to just represent threads/processes.

    \paragraph{Agent Cluster}
        Collection of agents concurrently communicating with each other through means of shared memory form an agent cluster.  
        There can be multiple agent clusters. 
        However, an agent can only belong to one agent cluster. 
        Agents communicating through message passing may or may not belong to the same agent cluster. 
        For our purpose, we assume just one agent cluster having one shared memory. 

    \paragraph{Agent Event List $(ael)$}
        Every agent is mapped to a list of events. The list represents the order in which the events are evaluated operationally\footnotemark. 
        We define $ael$ as a mapping of each agent to a list of events.
        
        \footnotetext{The standard refers this to be an Event List, but we find it a bit misleading as it does not signify a list for each agent. Hence we name it as Agent Event List.} 

        

            



%Events------------------------------------------------------------------
\section{Events}
        
    Agent execution is modelled in terms of events. An event is either an operation that involves (shared) memory access or that constrains the order of execution of multiple events.

    \subsection{Event Types}
    
        Given an agent cluster, an \textit{event set} \set{E} is a collection of all events from the agent event lists. This set is composed of mainly two distinct subsets as follows: 

        \begin{itemize}
            \item \textbf{Shared Memory (\set{SM}) Events}
                
                This set is composed of two sets of events; those that write to shared memory called Write events (\set{W}) and those that read from shared memory called Read events (\set{R}). Events that belong to both Write and Read events are called Read-Modify-Write. 
            
            \item \textbf{Synchronize (\set{S}) Events} 
                These events only restrict the ordering of execution of events by agents. For instance $lock$ and $unlock$ type of events can be categorized under Synchronize events. However, this is not stated in the specification\footnotemark. 
    
                \footnotetext{The features of $Lock$ and $Unlock$ events is actually not something given to the programmer to use in Javascript. They areused to implement the feature $wait$ and  $notify$ that the programmer can use which adhere to the semantics of $futexes$ inLinux. Hence, in the original standard of the model, the distinction between lock and unlock is not made, and it is simplystated as Synchronize Event.}
        \end{itemize}
        
        \begin{figure}[H]
            \centering 
            \includegraphics[scale=0.7]{4.ECMAScriptMemoryModel/EventTypes.pdf}
            \caption{Different sets of events.}
        \end{figure}

    %Range of events
    \subsection{Range ($\Re$)}
        Each of the \textit{shared memory events} are associated with a contiguous range of memory on which it operates. Range is afunction that maps a shared memory event to the range\footnotemark it operates on. This we represent as a starting index $i$ and a size$s$. So we could represent the range of a write event $w$ as 
                
                \[\Re(w) = (i, s) \]
    
        \footnotetext{The range as per the ECMAScript standard denotes only the set of contiguous byte indices. The starting byte indexis kept separate. We find this to be unnecessary. Hence we define range to have starting index and size.}
        
        We define the two binary operators below on ranges: 
        \begin{enumerate}
            \item Intersection $(\cap{_\Re})$ - Set of byte indices common to both ranges.
            \item Union $(\cup_\Re)$ - A unique set of byte indices that exist in both the ranges.  
        \end{enumerate}
        
        Two Ranges can be \textit{disjoint}, \textit{overlapping} or \textit{equal}. We use the binary operators to define these threepossibilities between ranges of events $e$ and $d$ :
        \begin{enumerate}
            \item Disjoint $\Re(e) \cap_\Re \Re(d) = \phi$ 
            \item Overlapping $(\Re(e)\cap_\Re \Re(d) \neq \phi) \wedge (\Re(e) \cap_\Re  \Re(d) \neq \Re(e) \cup_\Re \Re(d))$ - 
            \item Equal $\Re(e) \cap_\Re  \Re(d) = \Re(e) \cup_\Re \Re(d)$ - In simple terms, we define equality as $\Re(e) = \Re(d)$
        \end{enumerate}
            
%Types of Events Based on Order--------------------------------------------------------------------------------------------------------------------
    
    \subsection{Event Order / Event Access Mode} 
        Order signifies the sequence in which event actions are visible to different agents as well as the order in which they are executed by the agents themselves. In our context, there are mainly three types (in C11 memory model, they are called access modes) for each shared memory event that tells us the kind of ordering that it enforces. 
        
        \begin{enumerate}
            \item \textbf{Sequentially Consistent ($sc$)} - Events of this type are \textit{atomic}\footnotemark  in nature. There is a strict global total ordering of such events which is agreed upon by all agents in the agent cluster. 
            
            \item \textbf{Unordered ($uo$)} - Events of this type are considered \textit{non-atomic} and can occur in different orders for each concurrent process. There is no fixed global order respected by agents for such events. 
            
            \item \textbf{Initialize ($init$)} - Events of this type are used to initialize the values in memory before they are accessed by agent events. 
        \end{enumerate}

        All events of type \textit{init} are writes and all Read-Modify-Write events are of type \textit{sc}.  
        We represent the type of events in the memory consistency rules in the format ``$\textit{event} : \textit{type}$''. 
        When representing events in examples, the type would be represented as a subscript: $\textit{event}_\textit{type}$. 
       
        \begin{figure}[H]
            \centering
            \includegraphics[scale=0.7]{4.ECMAScriptMemoryModel/AccessModes.pdf}
            \caption{Event access modes with its restriction for some events.}
        \end{figure}

        \footnotetext{The word \textit{atomic} does not imply the events are evaluated using just one instruction. For example, a 64-bit sequentially consistent write on a 32-bit system has to be done with two subsequent memory actions. But its intermediate state of write must not be seen by any other agent. In an abstract sense, this event must appear '\textit{atomic}'.The \textit{atomic} here also refers to implications of whether an event's consequence is visible to all other agents in the same global total order or not. The compiler must ensure that for each specific target hardware, such guarantees are satisfied.}

%Tearing factor of events---------------------------------------------------------------------------------------------------------------------------

    \subsection{Tear Free ($tf$) or Tearing $!tf$)}
        Additionally, each shared-memory event is also associated with whether they are tear-free or not. OEvents that tear are non-aligned accesses requiring more than one memory access. Events that are tear-free are aligned and should appear to be serviced in one memory fetch\footnotemark.

        We represent the tearing of events in the memory consistency rules in the format ``$\textit{event} : \textit{tf/!tf}$''. 
        When representing events in examples, the type would be represented as a subscript: $\textit{event}_\textit{tf/!tf}$. 
       
        \footnotetext{It is not clear whether the alignment is with respect to specific hardware or not. The notion of one memory fetch may not be possible for all hardware practically, but it is something that must appear so. We will see a rule for ensuring this in the memory consistency rules.}
                       

%Relation among events----------------------------------------------------------------------------------------------------------------------------
    \section{Relation among events}
        We now describe a set of binary relations between events. These relations help us describe the consistency rules.
        
        \subsection{Read-Write event relations}
        There are two basic relations that assist us in reasoning about read and write events.
        
            %Read bytes from relation 
            \paragraph{Read-Bytes-From $(\stck{_{rbf}})$}
            
            This relation maps every read event to a list of tuples consisting of write event and their corresponding byte index that is read. For instance, consider a read event $r[i...(i+3)]$ and corresponding write events $w_1[i...(i+3)]$, $w_2[i...(i+4)]$. One possible $\stck{_\textit{rbf}}$ relation could be represented as  
                \begin{align*}
                    \reln{e}{\textit{rbf}}{\{(d1, i), (d2, i\!+\!1), (d2, i\!+\!2)\}}     
                \end{align*}   
            or having individual binary relation with each write-index pair as 
            \begin{align*}
                \reln{e}{rbf}{(d1, i)},\ \reln{e}{rbf}{(d2, i\!+\!1)}  \text{ and } \reln{e}{rbf}{(d2, i\!+\!2)}.
            \end{align*}
            
            %Reads from relation
            \paragraph{Reads-From $(\stck{_{rf}})$}
            
            This relation, is similar to the above relation, except that the byte index details are not involved in the composite list. So for the above example, the \textit{rf} relation would be represented either as   
                $\reln{e}{rf}{(d1, d2)}$
            or individual binary read-write relation as 
                $\reln{e}{rf}{d1}$ and $\reln{e}{rf}{d2}$.
            Figure below is an example of a program with its outcome (read values) shown in terms of reads-from relations. 
            \begin{figure}[H]
                \centering
                \includegraphics[scale=0.7]{4.ECMAScriptMemoryModel/ReadsFrom.pdf}
                \caption{An example showing \textit{reads-from} relations.}
                \label{read-from}
            \end{figure}
            
            %Agent sync with relation
        \subsection{Agent-Synchronizes With (\set{ASW})}
        
            It is a list for each agent that consist of ordered tuples of synchronize events. These tuples specify ordering constraints among agents at different points of execution. So such a list for an agent $k$ would be represented like:  
                \[ASW_k = \{ \langle s_1, s_2 \rangle, \langle s_3, s_4 \rangle ...\}\]
        
            For every pair in the list, the second event belongs to the parent agent and the first belongs to another agent it synchronized with\footnotemark.
                \[  
                    \forall{i,j>0},\ 
                    \langle s_1, s_2 \rangle \in ASW_j 
                    \Rightarrow{} 
                    s_2 \in ael(k)                        
                \]

            The figure below shows an example of this relation among two agents. 
            \begin{figure}[H]
                \centering
                \includegraphics[scale=0.7]{4.ECMAScriptMemoryModel/AgentSyncWith.pdf}
                \caption{An example showing \textit{agent-sync-with} relations.}
                \label{agent-sync-with}
            \end{figure}
        
        \footnotetext{This is analogous to the property that every unlock must be paired with a subsequent lock, which enforces the condition that a lock can be acquired only when it has been released.}

%Ordering Relation among Events----------------------------------------------------------------------------------------------------------------------       
\section{Ordering Relations among Events}
        
    We define agent-order, synchronize-with order, happens-before order and memory order using the binary relations defined on events in the previous section.
    %Agent Order
    \subsection{Agent Order ($\stck{_{ao}}$)}
        This is a union of the total orders among events belonging to the same agent event list. 
        It is analogous to \textit{intra-thread} ordering. 
        For example, if two events $e$ and $d$ belong to the same agent event list, then either $\reln{e}{ao}{d}$ or $\reln{d}{ao}{e}$. 
        Figure~\ref{model:agent-order} shows an example of agent order between events(right) composing a program(left).
        \begin{figure}[H]
            \centering
            \includegraphics[scale=0.7]{3.ECMAScriptMemoryModel/AgentOrder.pdf}
            \caption{An example with \textit{agent-order} among events.}
            \label{model:agent-order}
        \end{figure}
    
    %Synchronize With Order
    \subsection{Synchronize-With Order ($\stck{_{sw}} $)}
        This is a binary relation between two events that establish synchronization between multiple agents. 
        It is a composition of two sets: 
        \begin{enumerate}
            \item All pairs belonging to $ASW$ of every agent belongs to this ordering relation. 
                \begin{align*}
                    \langle e_i, e_j \rangle \in ASW \Rightarrow{} \reln{e_i}{sw}{e_j}. 
                \end{align*}
                    
            \item Specific reads-from pairs also belong to this ordering relation\footnotemark. 
                \begin{align*}
                    (\reln{r}{rf}{w}) \ \wedge \ \et{r}{sc} \ \wedge \ \et{w}{sc} \ \wedge \ (\Re(r)\!=\!\Re(w)) \ \Rightarrow{} \
                    (\reln{w}{sw}{r}).
                \end{align*}            
        \end{enumerate}
        Figure~\ref{model:sync-with} shows examples of such orders that can exist between events(right) of a program(left).
        The orange box represents the agent-synchronizes-with set that exists coupled with a possible outcome of the program (the final read value of $a$).
        \begin{figure}[H]
            \centering
            \includegraphics[scale=0.7]{3.ECMAScriptMemoryModel/SynchronizeWith.pdf}
            \caption{An example with \textit{synchronize-with} relations among events.}
            \label{model:sync-with}
        \end{figure}

        \footnotetext{Note that for the second condition, both ranges of events have to be equal. This however, does not mean that the read cannot read from multiple write events(refer the definition for $\stck{_{rbf}}$ in Subsection 3.3).}
        
    %Happens Before order 
    \subsection{Happens Before Order ($\stck{_{hb}}$)}
        This is a transitive order on events, composed of the following:
        \begin{enumerate}
            \item Every agent-ordered relation is also a happens-before relation 
                \begin{align*}
                    (\reln{e}{ao}{d}) \ \Rightarrow{} \ (\reln{e}{hb}{d}).    
                \end{align*}
                
            \item Every synchronize-with relation is also a happens-before relation 
                \begin{align*}
                    (\reln{e}{sw}{d}) \ \Rightarrow{} \ (\reln{e}{hb}{d}).    
                \end{align*}
                 
            \item Initialize type of events happen before all shared memory events that have overlapping or equal ranges between them. 
                \begin{align*}
                    \forall e,d \in SM \ \wedge \ 
                    \et{e}{init} \ \wedge \ 
                    (\Re(e) \cap \Re(d) \neq \phi)
                    \ \Rightarrow{} \ 
                    \reln{e}{hb}{d}.
                \end{align*}          
        \end{enumerate}
        Figure~\ref{model:happens-before} summarizes all possible patterns of happens-before order between events(right) of a program(left).
        THe orange box, again represents the agent-synchronizes-with set that exists coupled with a possible outcome of the program (the final read value of $a$).
        \begin{figure}[H]
            \centering
            \includegraphics[scale=0.7]{3.ECMAScriptMemoryModel/Happens-before.pdf}
            \caption{An example with all the variants of \textit{happens-before} relations between events.}
            \label{model:happens-before}
        \end{figure}
    
    %Memory Order
    \subsection{Memory Order ($\stck{_{mo}}$)}
        This is a \textit{total order} on all events that respect happens-before order\footnotemark. 
        \begin{align*}
            \reln{e}{hb}{d} \Rightarrow{} \reln{e}{mo}{d}.    
        \end{align*}

        \footnotetext{An interesting part is that memory order, though total, is a bit undefined as to how it weaves together this total order given different events.}
        
        Figure~\ref{model:memory-order} is an example of a memory order between events(right) of a program(left).
        \begin{figure}[H]
            \centering
            \includegraphics[scale=0.7]{3.ECMAScriptMemoryModel/MemoryOrder.pdf}
            \caption{An example with \textit{memory-order} (total) among all events.}
            \label{model:memory-order}
        \end{figure}


\section{Helper Definitions}
    
    Before we go into the consistency rules. we define certain preliminary definitions that create a separation based on a program, the axiomatic events and the various ordering relations defined above. This will help us understand where the consistency rules actually apply.    
    
    %What is a program 
    \begin{definition}{Program} 
        A \emph{program} is the source code without abstraction to a set of events and ordering relations. In our context, it is the original Javascript program. 
        
        %Note that here we need to supply a Javascript program using shared array buffers and atomic objects. A thing to do after we give examples for the ones below
    \end{definition}
    
%-----------------------------------------------------------------------------------------------------------------------------------------    
    %What is one run of a program to us?
    \begin{definition}{Candidate}
        A a collection of abstracted set of shared memory events of a program involved in one possible execution, with the $\stck{_\textit{ao}}$ relations. 
        We can think of this as each thread having a set of shared memory events to run in a given intra-thread ordering. 
        Figure~\ref{model:candidate} is an example of a Candidate.
        \begin{figure}[H]
            \centering
            \includegraphics[scale=0.7]{4.ECMAScriptMemoryModel/candidate.pdf}
            \caption{An example of a Candidate.}
            \label{model:candidate}
        \end{figure}
        
    \end{definition}

%-----------------------------------------------------------------------------------------------------------------------------------------  
    \begin{definition}{Candidate Execution}
        A Candidate with the addition of $\stck{_\textit{sw}}$, $\stck{_\textit{hb}}$ and $\stck{_\textit{mo}}$ relations. 
        This can be viewed as the witness/justification of an actual execution of a Program. 
        Note that there can be many Candidate Executions for a given Candidate. 
        Figure~\ref{model:candexec} shows an example of a Candidate Execution.  
        \begin{figure}[H]
            \centering
            \includegraphics[scale=0.7]{4.ECMAScriptMemoryModel/CandidateExecution.pdf}
            \caption{An example of a Candidate Execution based on Candidate above.}
            \label{model:candexec}
        \end{figure}
        
    \end{definition}

 %-----------------------------------------------------------------------------------------------------------------------------------------   
    %What values are read when the program is run
    \begin{definition}{Observable Behavior}
        The set of pairwise $\stck{_\textit{rf}}$/$\stck{_\textit{rbf}}$ relations that result in one execution of the program. 
        Think of this as our outcome of a program execution.
        Figure~\ref{model:observable} shows an example of an observable behavior.
        \begin{figure}[H]
            \centering
            \includegraphics[scale=0.7]{4.ECMAScriptMemoryModel/Observables.pdf}
            \caption{An example of Observable Behavior.}
            \label{model:observable}
        \end{figure}
        
    \end{definition}

%-----------------------------------------------------------------------------------------------------------------------------------------
    

%Valid Execution Rules---------------------------------------------------------------------------------------------------------------------------------
        
    \section{Valid Execution Rules (the Axioms)}
        We now state the memory consistency rules. The rules are on \textit{Candidate Executions} which will place constraints on the possible \textit{Observable behaviors} that may result from it. 
         
%--------------------------------------------------------------------------------------------------------------------------------------  
        %Coherent Reads   
        \subsection{Coherent Reads} 
        
            There are certain restrictions of what a read event cannot see at different points of execution based on $\stck{_{hb}}$ relation with write events. 

            Consider a read event $e$ and a write event $d$ having at least overlapping ranges:
            \begin{align*}
                \event{e}{R} \ \wedge \ 
                \event{d}{W} \ \wedge \
                (\Re(e) \cap_\Re \Re(d) \neq \phi).
            \end{align*}

            \begin{itemize}
                %Rule #1
                \item A read value cannot come from a write that has happened after it 
                    \begin{align*}
                        \reln{e}{hb}{d}\ \Rightarrow{}\ \neg \ \reln{e}{rf}{d}.
                    \end{align*}
                    The figure below pictorially depicts the pattern above hwere $e$ cannot read from $d$.
                    \begin{figure}[H]
                        \centering
                        \includegraphics[scale=0.7]{ECMAScriptMemoryModel/CoherentReads1.pdf}
                        \caption{A read value cannot come from a write that has happened after it}
                    \end{figure}
                %Rule #2    
                \item A read cannot read a specific byte address value from write if there is a write $g$ that happens between them which modifies the exact byte address. Note that this rule would be on the $rbf$ relation among two events. 
                    \begin{align*}
                        \reln{d}{hb}{e}
                        \ \wedge \ 
                        \reln{d}{hb}{g} \ \wedge \  \reln{g}{hb}{e}
                        \ \Rightarrow{} \
                        \forall x \in (\Re(d) \cap_\Re \Re(g) \cap_\Re \Re(e)), \ \neg \ \reln{e}{rbf}{(d,x)}.
                    \end{align*}
                    The figure below pictorially depicts the pattern where $e$ cannot read certain bytes from $d$. 
                    \begin{figure}[H]
                        \centering 
                        \includegraphics[scale=0.7]{ECMAScriptMemoryModel/CoherentReads2.pdf}
                        \caption{A read value cannot come from a write if
                        there is a write that happens between them, writing to the same memory:}
                    \end{figure}
                            
            \end{itemize}
         
%------------------------------------------------------------------------------------------------------------------------------------------

        \subsection{Tear-Free Reads} 
            If two tear free writes $d$ and $g$ and a tear free read $e$ all with equal ranges exist, then $e$ can read only from one of them\footnotemark.
                
            \begin{align*}
                \et{d}{tf}\ \wedge\ \et{g}{tf} \ \wedge \ \et{e}{tf} 
                  \ \wedge \ 
                  (\Re(d) \!=\! \Re(g) \!=\! \Re(e)) 
                  \ \Rightarrow{} \\ 
                      ((\reln{e}{rf}{d}) 
                      \ \wedge \ 
                      (\neg \ \reln{e}{rf}{g})) 
                  \ \vee \  
                      ((\reln{e}{rf}{g}) 
                      \ \wedge \
                      (\neg \ \reln{e}{rf}{d})).
            \end{align*}
                    
            The following figure shows the pattern that is disallowed among all tear-free events. 
            \begin{figure}[H]
                \centering
                \includegraphics[scale=0.7]{ECMAScriptMemoryModel/TearFreeReads.pdf}
                \caption{Pattern of Tear-free reads}
            \end{figure}

            \footnotetext{To recap a tear-free event cannot be separated into multiple small events that do the same operation. However, considering different hardware architectures, the notion of tear-free need not necessarily mean this. (eg: A 64bit tear-free write to be done in a 32bit system). In a more abstract sense, we need an event to appear 'tear-free'.}    
            
        \subsection{Sequentially Consistent Atomics} 
            
            To specifically define how events that are sequentially consistent affects what values a read cannot see, we assume the following memory order among writes $d$ and $g$ and a read $e$ to be the premise for all the rules:  
                \begin{align*}
                    d \stck{_{mo}} g \stck{_{mo}} e.
                \end{align*}
               
            \begin{itemize}
                \item If all three events are of type $sc$ with equal ranges, then $e$ cannot read from $d$
                    \begin{align*}
                        \et{d}{sc}\ \wedge\ \et{g}{sc}\ \wedge\ \et{e}{sc} 
                        \ \wedge \ (\Re(d) \!=\! \Re(g) \!=\! \Re(e))
                        \ \Rightarrow{} \ 
                        \neg \ \reln{e}{rf}{d}.
                    \end{align*} 
                        
                    The figure below depicts pictorially the pattern that is not allowed by this rule.
                    \begin{figure}[H]
                        \centering 
                        \includegraphics[scale=0.7]{ECMAScriptMemoryModel/SequentialAtomics1.pdf}
                        \caption{A read value cannot come from a write, if there exists a write memory ordered between them and all  3 events are sequentially consistent with equal ranges.}
                    \end{figure}
                    
                \item If both writes are of type $sc$ having equal ranges and the read is bound to happen after them, then $e$ cannot read from $d$ 
                    \begin{align*}
                        \et{d}{sc}\ \wedge\ \et{g}{sc}  
                        \ \wedge \ (\Re(d) \!=\! \Re(g)) 
                        \ \wedge \ \reln{d}{hb}{e}
                        \ \wedge \ \reln{g}{hb}{e}
                        \ \Rightarrow{}\  
                        \neg \ \reln{e}{rf}{d}.
                    \end{align*}
                        
                    The figure below depicts pictorially the pattern that is not allowed by this rule.
                    \begin{figure}[H]
                        \centering 
                        \includegraphics[scale=0.7]{ECMAScriptMemoryModel/SequentialAtomics2.pdf}
                        \caption{A read value cannot come from a write, if there exists a write memory ordered between them and both writes are sequentially consistent with equal ranges.}
                    \end{figure}
                
                \item If $g$ and $e$ are sequentially consistent, having equal ranges, and $d$ is bound to happen before them, then $e$ cannot read from $d$
                    \begin{align*}
                        \et{g}{sc}\ \wedge\ \et{e}{sc}  
                        \ \wedge \ (\Re(g) \!= \!\Re(e)) 
                        \ \wedge \ \reln{d}{hb}{g} 
                        \ \wedge \ \reln{d}{hb}{e}
                        \ \Rightarrow \ 
                        \neg \ \reln{e}{rf}{d}.
                    \end{align*}

                    The figure below depicts pictorially the pattern that is not allowed by this rule.
                    \begin{figure}[H]
                        \centering 
                        \includegraphics[scale=0.7]{ECMAScriptMemoryModel/SequentialAtomics3.pdf}
                        \caption{A read value cannot come from a write, if there exists a write memory ordered between them and both this write and the read are sequentially consistent with equal ranges.}
                    \end{figure}
                        
            \end{itemize}

%Races----------------------------------------------------------------------------------------------------------------------------------     
    \section{Race}

        We would want our concurrent programs to be written in such a way that the order in which shared memory is read from or written to is deterministic.
        This, however, may not always be the case. 
        For applications such as client-server interactions, we would rather be more comfortable keeping the client requests being serviced in a non-deterministic fashion. 
        This is because clients can be in millions, and always having a fixed order in which they should be serviced may not be ideal.
        The accesses to shared memory that are involved, which need to be deterministically ordered, but are left to program execution, are said to \textit{race} and are in a \textit{race condition}.
        Pairs of accesses that need to be ordered are those whose order affects the final outcome (observable behavior) of the program.
        In our view, they are simply those pairs, of which at least one of them is a write. 
        
        In the context of relaxed memory accesses, another type of race exists.
        In the normal \textit{race condition}, though accesses are not ordered deterministically by us, every execution will give us an outcome that can be justified using one particular order in which the concurrent accesses were issued. 
        However, there could be cases, i.e. outcomes of program execution, where no ordering among these accesses can justify\footnotemark.
        In such a situation, concurrent accesses are said to have a \textit{data race} among them.    
        
        \footnotetext{This is attributed to the notion of \textit{atomicity}. Atomic accesses, ensure that only the outcomes that result due to their different orderings are possible. Other accesses that do not respect this rule are called \textit{non-atomic}. In the context of our model, they are considered to be of type $uo$.}

        We define the two forms of races formally with respect to the ECMAScript memory model.
        
        \subsection{Race Condition $RC$} 
            We define \set{RC} as the set of all pairs of events that are in a race. 
            Two events $e$ and $d$ are in a race when they are shared memory events:
            \begin{align*}
                (e \in SM)\ \wedge\ (d \in SM).
            \end{align*}
            having overlapping or equal ranges, not ordered by the $\stck{_\textit{hb}}$ relation with each other, and which are either two writes or the two events are involved in a $\stck{_\textit{rf}}$ relation with each other. 
            This can be stated concisely as,
            \begin{align*}
                \neg \ (\reln{e}{hb}{d})\ \wedge\ \neg \ (\reln{d}{hb}{e}) 
                \ \wedge \ 
                (
                (\event{e,d}{W}\  \wedge\ (\Re(d) \cap_{\Re} \Re(e) \neq \phi)) 
                    \  \vee\ (d \stck{_\textit{rf}} e)\ \vee\ (e \stck{_\textit{rf}} d)
                ).
            \end{align*}
                    
        \subsection{Data Race $DR$} 
            We define \set{DR} as the set of all pairs of events that are in a data-race. 
            Two events are in a data race when they are already in a race condition and when the two events are not both of type \textit{sc}, or they have overlapping ranges. 
            This is concisely stated as:  
            \begin{align*}
                \event{e,d}{RC}  \ \wedge \ 
                ((\neg\et{e}{sc} \ \vee \ \neg\et{d}{sc}) \ \vee \ 
                (\Re(e) \cap_{\Re} \Re(d) \neq \Re(e) \cup_{\Re} \Re(d))). 
            \end{align*}
            
        \paragraph{Data-Race-Free (DRF) Programs}
            An execution is considered data-race-free if none of the above conditions for \textit{data-race} occur among events in a candidate execution and its observable behaviors. 
            A program is data-race-free if all its executions are data-race-free.          
            \textit{The memory model guarantees Sequential Consistency for all data-race free programs (SC-DRF).}      

%Consistent Executions-------------------------------------------------------------------------------------------------------------------

   %Consistent Executions-------------------------------------------------------------------------------------------------------------------

   \section{Consistent Executions (Valid Observables)}
      
      Consistent executions are those which should ideally be possible if the program is actually run on some hardware. 
      For a sequential program, we use the semantics of the programming language to understand what can be the outcome of a program. 
      For a concurrent program, since we can have multiple outcomes of the same program being executed (keeping all inputs constant), we need a semantic model to rely on. 
      The memory model is in essence just this semantic model for programs using shared memory.
      
      In our language, a consistent execution maps to a valid observable behavior, as this is what the user can actually record as an outcome of the program. 
   
      As per the standard specification, valid observable behaviour is when\footnotemark:
        \begin{enumerate}
           \item No $\stck{_\textit{rf}}$ relation violates the above memory consistency rules.
           \item $\stck{_\textit{hb}}$ is a strict partial order.
        \end{enumerate} 

        \textit{The memory model guarantees that every program must have at least one valid observable behavior.}

\ \newline
\ \newline  
\hrule 
\ \newline 
\ \newline 
As a summary, this chapter axiomatically defined the ECMAScript memory model. 
We defined binary relations on events and specified the constraints of the model in terms of restricting reads-from relations that can exist between events in a Candidate Execution.
In the next chapter, we use this formal model to reason about the validity of instruction reordering.