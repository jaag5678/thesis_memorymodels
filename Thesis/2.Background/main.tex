This chapter starts with a review of key previous work done in the domain of relaxed memory models.
We start by eliciting research works done in the design of memory models, coupled with works exposing problems due to ill defined or informally specified semantics. 
We then elicit research works done in the context of different memory models concerning validity of program transformations. 
Finally, we end with a list of tutorial works done in the axiomatic style specification of memory models.
\ \newline
\ \newline  
\hrule 
\ \newline 
\ \newline 

A lot of problems emerged with the introduction of concurrent computing. 
Starting with the famously known mutual exclusion problem way back in the 1960s introduced and solved by Djikstra et al.~\cite{Djikstra}, from it stemmed the ideas of semaphores and monitors that we all know today.  
Since then, a lot of well-known problems in concurrency such as The Producer Consumer Problem, Dining Philosopher's problem, Reader-Writer problem have come up, along with different algorithms to implement these in a practical sense, especially in Operating Systems. 

However, the above concerns in concurrency only revolves around the so-called ``critical section''; which requires that every thread trying to do something in a shared memory region must have exclusive access to it before doing anything. 
In recent decades, the need for more performance has increased; multiple hardwares having different features such as read-write buffers, caches, etc exist. 
Reliance solely on critical sections to perform operations on shared memory is today observed as a bottleneck. 
Without its use, however, the possible outcomes of a concurrent program increases and complicates the notion of \textit{atomicity} in actions done on shared memory.

%Start by an intuitive explaination of memory models. What it means for us and why was it introduced.

Sequential Consistency, which was first formulated by Lamport et al.~\cite{Lamport79}, gives programmers a very intuitive way to reason about their programs running in a multiprocessor environment.
However, in the practical sense, Sequential Consistency is too ``strict,'' in the sense that it may impede possible performance benefits of using low level optimization features, such as instruction reordering, or read/write buffers provided by the hardware.
A tutorial by Adve et al.~\cite{AdveG}, summarizes the most common hardware features for relaxed memory that are now available in most hardware. 
What this tutorial also exposed is the difficulty in formalizing such features in a way that we can reason about our programs sanely without getting caught up in the complexity of multiple executions of our programs. 
Unsurprisingly, relaxed memory model specifications for different hardware / high level programming languages are still sometimes written in informal prose format, which lead to a number of problems in implementation~\cite{Sewell}. 

%x86 memory model
Sarkar et al.~\cite{SarkarS} showed that the original x86-CC memory model was fairly informal, which they then formalized in their work. This also exposed inconsistencies between the specification and the implementation in hardware. This was shown in their subsequent work done by Owens et al.~\cite{OwensS}, wherein they proposed a new memory model x86-TSO as a remedy. 
%Java Memory model
Manson et al.~\cite{JeremyM}, showed that the initial specifications of the Java memory model were quite informal and ill defined, and offered a more precise formalization. Recent works such as that done by Bender et al.~\cite{BenderJ}, also shows us that the recent updates to the java Memory model is still relatively unclear, which they again formalize. 
%C++ memory model
%Foundations of C++ memory model. Followed by Batty, then followed by Kyndlain, then conclude by RC11
Similarly, Batty et al.~\cite{BattyM}, clarified the specification of the C11 memory model. 

%Mixed size memory models

ECMAScript has also had some attention in this respect. Watt et al~\cite{WattC} uncovered and fixed a deficiency in the previous version of the model, repairing the model to guarantee SC-DRF.

%Cite that one paper. thats it.
%Also cite perhaps ECMAScript.

\section{Program Transformations under Weak Memory}

    Although programmers usually are responsible to write efficient programs, their performance during execution does not always depend on how well the program is written. 
    Several compiler optimizations, coupled with run-time optimizations by hardware play a big role in the end performance of programs.
    For sequential programs, a lot of well-established ideas exist to enhance the performance of programs, but they do not map well to that of concurrent programs. 
    A large class of optimizations are unsound under concurrent programs employing a memory model such as SC.
    With the introduction of constraints on relaxed memory accesses, quite a few critical program transformations responsible for huge performance gains became possible, but were hard to prove valid in general. 

    \paragraph{A change in data-flow} 
    This situation necessitates a fundamental change in how we employ data-flow analysis to optimize programs in a concurrent context. Naumovich et al.~\cite{NaumovichA} proposed one such analysis which is today commonly known as May-Happen-In-Parallel analysis. 
    Midkiff et al.~\cite{Midkiff} proposed a unified compiler for several memory models, wherein they proposed a new Concurrent Control Flow graph with new edges to perform powerful transformations such as Common Sub-Expression Elimination. 
    These models however, required more of a whole-program-analysis to even do basic local program transformations, which is quite expensive to implement and suffers from imprecision due to the large number of control flow paths implied in a large program. 

    For data-flow analysis, a recent work by Alglave et al.~\cite{Alglave2} show that non-relational data-flow analysis are sound under a category of relaxed memory models. 
    But code transformations are themselves not yet investigated with each one in its own respect, which entail a number of data-flow analyses and particular for code motion.
    
    \paragraph{An Alternative: Go to base!}
    Another way to go about this is to go one step below and observe the transformations in terms of more basic code transformations such as instruction reordering, elimination and introduction.   
    We start by showing how using counterexamples one can show basic transformations are invalidated under SC.

    For instance, the disallowed outcome in Figure~\ref{intro:Example} should be possible; we can simply reorder either the two events in $T1$ or those in $T2$ as they are disjoint memory operations. 
    But from a sequential consistency standpoint, since the outcome is not valid, it also brings with it the question whether such simple program transformations are even valid to perform.
    Figure~\ref{intro:Example2(a)} is an example where we would think it is okay to reorder two independent reads. 
    %Show program 
    \begin{figure}[H]
        \centering
        \includegraphics[scale=0.7]{2.Background/SC_Reordering(a).pdf}
        \caption{Program where reordering of independent reads seems possible with its allowed (green box) and disallowed (red box) outcomes in SC.}
        \label{intro:Example2(a)}
    \end{figure}

    The reordered program can justify a sequential interleaving of events to have the disallowed outcome in the original program. 
    This is shown in Figure~\ref{intro:Example2(b)}.
    \begin{figure}[H]
        \centering
        \includegraphics[scale=0.7]{2.Background/SC_Reordering(b).pdf}
        \caption{The transformed program reordering two reads in T1, with a trace justifying the disallowed outcome under SC.}
        \label{intro:Example2(b)}
    \end{figure}

    Such concerns are not only related to reordering, but also elimination. 
    Consider the program in Figure~\ref{intro:Example3(a)}.
    \begin{figure}[H]
        \centering
        \includegraphics[scale=0.7]{2.Background/SC_Example2(a).pdf}
        \caption{Program with a redundant write to $y$ with its allowed (green box) and disallowed (red box) outcomes under SC.}
        \label{intro:Example3(a)}
    \end{figure}

    In this example, the red box outcome is still not allowed under SC. We can notice though that the write to $y$ in $T2$ is done twice. 
    Naturally, the compiler might think of eliminating one of them under the context of redundant code-elimination. 
    Suppose it eliminates the first write $y=2$. 
    Then the resulting program as shown below in Figure~\ref{intro:Example3(b)}, can justify the outcome under SC which is disallowed in the original program.
    \begin{figure}[H]
        \centering
        \includegraphics[scale=0.7]{2.Background/SC_Example2(b).pdf}
        \caption{The transformed program eliminating the first $y=2$, with a trace justifying the disallowed outcome under SC.}
        \label{intro:Example3(b)}
    \end{figure}

    The above examples show that even simple transformations can be unsound under SC. Complex program transformations such as register allocation, common-sub-expression-elimination, loop-invariant code-motion are some examples which use the above two basic transformations heavily. 
    Having them unsound under SC also implies the compiler is not allowed to do a variety of optimizations without breaking the consistency rules under which the concurrent program is supposed to behave/execute. 
    In addition, hardware had come up with several features as mentioned above, which in principle could be used effectively for performance, but were not so useful for programs respecting SC semantics. 
    In order to effectively critique a model in terms of program transformations, it is quite impractical to come up with exhaustive examples of program to justify their validity.
    A proof in this sense would be required.  

    %Java
    \u{S}ev\u{c}\'{i}k et al.~\cite{SevcikJ} showed that standard compiler optimizations were rendered invalid under the memory model of Java. 
    Simple transformations such read-after-write elimination or redundant read introduction which play a major role in performance based transformations such as common-sub-expression eliminations were unsound. 
    %C11
    Morisset et al.~\cite{Morisset} showed the soundness of optimizations with non-atomic memory accesses in the C11 memory model. 
    Vafeiadis et al.~\cite{VafeiadisV} showed that common compiler optimizations (including those with atomic memory accesses) under C11 memory model were invalid, followed by proposing some changes to allow them. 
    Transformations such as sequentialization, strengthening access modes, and even \textit{roach-motel} reorderings were unsound. Their proposed changes to the model have been incorporated by the standard committee for C11. 

    %General Prog Transformations
    With respect to instruction reordering and redundancy elimination in shared memory programs, \u{S}ev\u{c}\'{i}k et al.~\cite{Sevcik2} gave a proof design on how to show such optimizations are valid. 
    This approach relies on the idea of reconstructing the original execution of a program given the optimized one, while also showing the well known SC-DRF guarantee holds---programs that are \textit{data-race-free} (DRF) must exhibit SC behavior. 
    Our approach is in fact the other way round; we show that the optimized program does not introduce new behaviors, by explicitly using the consistency rules to show that relevant ordering relations are preserved. 
    We do not show it specifically only for \textit{data-race-free} programs as the model that we refer to also requires that programs with races have a defined behavior. 

\section{Axiomatic Style Specifications of Weak Memory}

While most programmling language semantics are defined and analyzed operationally, memory consistency models have been shown to be more effective for analysis using an axiomatic specifcation.
This axiomatic specification typically is in terms of defining partial orders between relaxed memory events (read/write) and then specifying restrictions on the composition of these partial orders. 
A very elaborate literature on axiomatic semantics of weak memory is given by Alglave et al.~\cite{Alglave}. 
This work introduces a new tool called \textit{herd}, which can be used for testing program examples against memory models. The memory models themselves have to be specified in axiomatic format. 

Typically, such an axiomatic approach to something related to concurrency avoids dealing with the state space explosion of operational models, which is often quite difficult to reason with given so many relaxed memory based outcomes. 
We can completely avoid the problems of reasoning with seemingly infinite states of programs by reasoning with partial orders between events in a concurrent program.
We in our approach also rely on this axiomatic perspective, using which we prove the validity of two powerful program transformations used in most compiler optimizations. 

In our literature review, such an axiomatic perspective traces back to times when problems of relaxed memory accesses were being addressed only at the hardware level. 
Sindhu et al.~\cite{Sindhu} specifies an axiomatic framework for specifying the behaviors of shared memory multiprocessors. 
Owens et al.~\cite{OwensS}, Batty et al.\cite{BattyM} specify the respective memory models in an axiomatic format. 




  
\section{Other Concerns}

    To date, memory consistency models are still lacking a very user-friendly specification for use by many programmers. 
    In addition to this, we also showed that related work existed exposing limitations in many models with respect to basic program transformations.
    
    \paragraph{Compilation}
    There is also the big concern of compilation; whether the compilation (code-gen phase mainly) from source and target with different memory models is correct? 
    This question is quite open-ended given a varying class of memory models that exist. 
    While this direction goes beyond the scope of the thesis, works done by Lahav et al.~\cite{Lahav} and Watt et al.~\cite{WattC} on the C11 and JavaScript memory model respectively exposed incorrect compilation done due to which the program exhibited unintended behaviors on execution in particular hardwares (POWER and ARM respectively). 
    The interested reader can refer to Podkopaev et al.~\cite{Anton} to explore further in this direction.
    
    \paragraph{Verification/Model-Checking}
    While the focus of this thesis is not on Formal Verification, there certainly does exist several research works in verifying weak memory programs. 
    The main concern is that because of weak memory accesses, a program has even more outcomes (which increase more exponentially compared to concurrent programs performing operations on shared memory only in \textit{critical sections}.) 
    One of the well known methods in this decade for verifying concurrent programs is using \textit{Stateless Model Checking}. 
    There are two works in this explored during literature review that are exemplary of the use of Stateless Model Checking in weak memory programs.
    Michalis et al.~\cite{Michalis1} provided a model checker named RCMC for performing effective model checking of relaxed memory programs in C11. 
    Michalis et al.~\cite{Michalis2} provided a model checker named GenMC for performing effective model checking parametric to relaxed memory models. 
    
    \paragraph{Out-of-Thin-Air }
    A few well known memory models face the concern which is notoriously known as \textit{out-of-thin-air}; a program can give an outcome that should not have existed in any observable behaviors as per the semantics. %PErhaps give an example here 
    Such behaviors have been shown to be in programs that exhibit \textit{data-races} in their executions.
    The C11 memory model \cite{C11MM} escapes from addressing it by stating that any program with \textit{data-races} has undefined behavior.
    Java on the other hand, relies on a complicated semantics to guarantee that no \textit{out-of-thin-air} values can exist in programs with \textit{data-races}. This to date is still very complicated and subtle to understand in order to use it properly in programs. 
    While one may conclude that \textit{out-of-thin-air} is a bad property that must be avoided at all costs, Verbrugge et al.~\cite{Verbrugge} shows that disallowing them also disallows quite a few potential compiler optimizations.

    \paragraph{Custom Memory Models}
    We also explored certain research works that came up with their own memory model in order to simplify/solve the problems above. 
    Arvind et al.~\cite{Arvind} presented a novel framework to specify memory models using Instruction Reordering and Store atomicity. This work in our eyes, was a better representation of intended behaviors that should be allowed by a memory model. 
    Marino et al.~\cite{Marino} proposed a new memory model named \textit{DRFx}, which they claimed to be quite intuitive for programmers as well as allowing many of the program transformations responsible for major performance benefits.
    Kang et al.~\cite{Kang} proposed a new memory model referred to as \textit{Promising Semantics} to address the well known out-of-thin-air problem that exists in current memory models. 

\ \newline
\ \newline  
\hrule 
\ \newline 
\ \newline 
As a summary, this chapter elicited the key research works done in the domain of relaxed memory models, its specification and its impact of program transformations. 
In the next chapter, we state the problems in the existing specification of the ECMAScript memory model, followed by a more formal specification of the same.