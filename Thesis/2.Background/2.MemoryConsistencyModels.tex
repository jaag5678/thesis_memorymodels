%Start by an intuitive explaination of memory models. What it means for us and why was it introduced.

Sequential Consistency, which was first formulated by Lamport et al.~\cite{Lamport79}, gives programmers a very intuitive way to reason about their programs running in a multiprocessor environment.
However, in the practical sense, Sequential Consistency is too ``strict,'' in the sense that it may impede possible performance benefits of using low level optimization features, such as instruction reordering, or read/write buffers provided by the hardware.
A tutorial by Adve et al.~\cite{AdveG}, summarizes the most common hardware features for relaxed memory that are now available in most hardware. 
What this tutorial also exposed is the difficulty in formalizing such features in a way that we can reason about our programs sanely without getting caught up in the complexity of multiple executions of our programs. 
Unsurprisingly, relaxed memory model specifications for different hardware / high level programming languages are still sometimes written in informal prose format, which lead to a number of problems in implementation~\cite{Sewell}. 

%x86 memory model
Sarkar et al.~\cite{SarkarS} showed that the original x86-CC memory model was fairly informal, which they then formalized in their work. This also exposed inconsistencies between the specification and the implementation in hardware. This was shown in their subsequent work done by Owens et al.~\cite{OwensS}, wherein they proposed a new memory model x86-TSO as a remedy. 
%Java Memory model
Manson et al.~\cite{JeremyM}, showed that the initial specifications of the Java memory model were quite informal and ill defined, and offered a more precise formalization. Recent works such as that done by Bender et al.~\cite{BenderJ}, also shows us that the recent updates to the java Memory model is still relatively unclear, which they again formalize. 
%C++ memory model
%Foundations of C++ memory model. Followed by Batty, then followed by Kyndlain, then conclude by RC11
Similarly, Batty et al.~\cite{BattyM}, clarified the specification of the C11 memory model. 

%Mixed size memory models

ECMAScript has also had some attention in this respect. Watt et al~\cite{WattC} uncovered and fixed a deficiency in the previous version of the model, repairing the model to guarantee SC-DRF.

%Cite that one paper. thats it.
%Also cite perhaps ECMAScript.