%Relation among events----------------------------------------------------------------------------------------------------------------------------
    \section{Relation among events}
        We now describe a set of binary relations between events. These relations help us describe the consistency rules.
        
        \subsection{Read-Write event relations}
            There are two basic relations that assist us in reasoning about read and write events.

            %Read-bytes-from
            \paragraph{Read-Bytes-From $(\stck{_{rbf}})$}
            This relation maps every read event to a list of tuples consisting of write event and their corresponding byte index that is read. 
            For instance, consider a read event $e$ with range $(i, 3)$ and corresponding write events $d1$ and $d2$ with ranges $(i, 3)$ and $(i,4)$ respectively. 
            Possible $\stck{_\textit{rbf}}$ relations could be  
                \begin{align*}
                    \reln{e}{\textit{rbf}}{\{(d1, i), (d2, i\!+\!1), (d2, i\!+\!2)\}}. \\
                    \reln{e}{\textit{rbf}}{\{(d1, i), (d1, i\!+\!1), (d2, i\!+\!2)\}}. \\
                    \reln{e}{\textit{rbf}}{\{(d2, i), (d1, i\!+\!1), (d2, i\!+\!2)\}}.       
                \end{align*}   
            or having individual binary relation with each write-index pair as 
            \begin{align*}
                \reln{e}{rbf}{(d1, i)},\ \reln{e}{rbf}{(d2, i\!+\!1)}  \text{ and } \reln{e}{rbf}{(d2, i\!+\!2)}. \\
                \reln{e}{rbf}{(d1, i)},\ \reln{e}{rbf}{(d1, i\!+\!1)}  \text{ and } \reln{e}{rbf}{(d2, i\!+\!2)}. \\
                \reln{e}{rbf}{(d2, i)},\ \reln{e}{rbf}{(d1, i\!+\!1)}  \text{ and } \reln{e}{rbf}{(d2, i\!+\!2)}. 
            \end{align*}
            
            %Reads from relation
            \paragraph{Reads-From $(\stck{_{rf}})$}
            This relation, is similar to the above relation, except that the byte index details are not involved in the composite list. 
            So for the above examples, the \textit{rf} relation would be represented either as   
                $\reln{e}{rf}{(d1, d2)}$
            or individual binary read-write relation as 
                $\reln{e}{rf}{d1}$ and $\reln{e}{rf}{d2}$.
            Figure~\ref{model:read-from} below is an example of a program with its outcome (read values) shown in terms of reads-from relations.
            The left is a program example with the orange box representing one outcome of a program. 
            \begin{figure}[H]
                \centering
                \includegraphics[scale=0.7]{3.ECMAScriptMemoryModel/ReadsFrom.pdf}
                \caption{An example showing \textit{reads-from} relations.}
                \label{model:read-from}
            \end{figure}
            
        %Agent sync with relation
        \subsection{Agent-Synchronizes-With (\set{ASW})}
            This is a set for each agent that consist of ordered tuples of synchronize events. 
            These tuples specify ordering constraints among agents at different points of execution. 
            So such a list for an agent $k$ would be represented like:  

                \[ASW_k = \{ \langle s1, s2 \rangle, \langle s3, s4 \rangle ...\}\]
        
            For every pair in the list, the second event ($s2$) belongs to the parent agent ($k$) and the first ($s1$) belongs to another agent it synchronized with\footnotemark.
                \[  
                    \langle s1, s2 \rangle \in ASW_k 
                    \Rightarrow{} 
                    s2 \notin ael(k).                        
                \]
            The ordered tuples are acyclic.
            \begin{align*}
                \langle s1, s2 \rangle \Rightarrow \neg \langle s2, s1 \rangle.
            \end{align*}

            Figure~\ref{model:agent-sync-with} below shows an example of this relation among two agents. 
            The left represents a program with synchronize events $s1$ and $s2$. 
            The orange box represents the synchronization established between the two agents.
            \begin{figure}[H]
                \centering
                \includegraphics[scale=0.7]{3.ECMAScriptMemoryModel/AgentSyncWith.pdf}
                \caption{An example showing \textit{asw} relations.}
                \label{model:agent-sync-with}
            \end{figure}
        
        \footnotetext{This is analogous to the property that every unlock must be paired with a subsequent lock, which enforces the condition that a lock can be acquired only when it has been released.}