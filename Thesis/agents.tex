%AGENTS----------------------------------------------------------------------------------------------------------------------------------------  
\section{Agents, Events and their Types}

    \subsection{Agents}
        A concurrent program involves different threads/processes running concurrently. 
        Agents could be thought analogous to different threads/processes. 
        
        \critic{red}{Agents actually have more meaning than what we refer to here. However, in terms of reasoning just with memory consistency, we are safely abstracting them to just represent threads/processes.}
        
        \critic{blue}{Technically, one may not map them directly to individual threads as from an implementation standpoint, a single thread can be allowed to execute multiple concurrent processes. As a way of separating implementation from the specifications, we refer to them as Agents}

        %Agent Clusters
        \paragraph{Agent Cluster ($AC$)}
        Collection of agents running concurrently communicating with each other (directly/ indirectly) form an agent cluster.  There can be multiple agent clusters. However, an agent can only belong to one agent cluster.
        
        %PErhaps give an example here later

        \critic{blue}{Note that for the purpose of reasoning with memory model, we stick to assuming that just one agent cluster exists. So we will refrain from defining a function mapping an agent to its respective agent cluster. We also assume that agents in the cluster communicate only through one common shared memory segment.}
        
        %Agent Event Set
        \paragraph{Agent Event List $(ael)$}
        Every agent is mapped to a list of events appended to it during evaluation. We define $ael$ is a mapping of each agent to a list of events.
        
            \[ael(a) = [e_1, e_2, ... e_k ] \]
        
        \critic{blue}{The standard refers this to be an Event List, but we find it a bit misleading as it does not signify a list for each agent. Hence we name it as Agent Event List}
        
        %Ask whether this is actually required as a notation down the line
        \paragraph{}
            When referring to events in an agent cluster, we use the following notation for an event:
        
            \[ e^i_j \ \longrightarrow \  j^{th}\ \textit{instruction}\ \textit{of}\ i^{th} \textit{Agent} \]  
            
        \critic{blue}{We will sometimes forgo the subscript or superscript wherever it may not play a role in understanding a relation or definition.} 
        
        

        

            

