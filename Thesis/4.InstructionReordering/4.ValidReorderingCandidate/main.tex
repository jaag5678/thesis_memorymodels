
\section{Valid reordering at the Candidate Level}
    
    Our main objective is to ensure that the set of possible observable behaviors of a program, remain unchanged after reordering.
    If that is not feasible, then we would want the set of observable behaviors after reordering at the very least to be a subset.

    \subsection{Reordering of Consecutive Events}
        \begin{theorem}
    \label{WriteElim}
    Consider a candidate $C$ of a program and its possible \textit{Candidate Executions} where $\stck{_\textit{hb}}$ is strictly partial order. 
    Consider two \textbf{write} events $e$ and $d$ in $C$ such that 
    \begin{align*}
        \cons{e}{d} \ \wedge \ \reln{e}{ao}{d}. 
    \end{align*}
    Consider a Candidate $C'$ after eliminating the event $e$ from $C$.  
    If
    \begin{align*}
        \et{e}{uo} \ \wedge \ \Re(e) = \Re(d). 
    \end{align*}
    then the set of Observable behaviors of $C'$ is a subset of $C$.  
\end{theorem}
        \begin{proof}
    Once again, we look at this as a write elimination done on a Candidate Execution of $C$. We start by proving when other happens-before relations remain intact. Followed by identifying relations lost due to elimination and a proof for when these relations do not introduce new observable behaviors. 
    
   \paragraph{1. Preserving \emph{happens-before} relations}
        The relations we want to preserve are those that are dervied through relation with $e$, viz. using the following two relations:
        \begin{tasks}(2)
            \task $\reln{k}{hb}{e}$
            \task $\reln{e}{hb}{k}$
        \end{tasks}

        We can divide the events involved in the above into two sets:
        \begin{align*}
            K_b = \{k \ | \ \reln{k}{hb}{e} \}. \\
            K_a = \{k \ | \ \reln{e}{hb}{k} \}. 
        \end{align*}

        We need to ensure the following relations hold after elimination.
        \begin{align*}
            \forall k_a \in K_a \ \wedge \ \forall k_b \in K_b \ . \ \reln{k_b}{hb}{k_a}
        \end{align*}

        Similar to reordering, we need to have a valid pivot pair $<p_b, p_a>$ such that 
        \begin{align*}
            \forall k_b \neq p_b \in K_b \ . \ \reln{k_b}{hb}{p_b} \\
            \forall k_a \neq p_a \in K_a \ . \ \reln{p_a}{hb}{k_a} 
        \end{align*}

        By Lemma \ref{Lemma1}, $\et{e}{uo}$ is the only case where $p_b$ can be a valid pivot. 
        By Lemma \ref{Lemma2}, $\et{e}{uo} \ \vee \ \et{e}{sc}$ are the cases where $p_a$ can be a valid pivot. 
        We need both the above conditions to be satisfied to have a valid pivot pair. 
        Hence, $\et{e}{uo}$ is the only possibility in which a valid pivot pair can exist. 

        \critic{blue}{Put a figure here to show this pivot role.}
   
    \paragraph{2. The \emph{happens-before} relations lost}

    The relations lost are those attached to the event $e$, which are: 
    \begin{align}
        \reln{k}{hb}{e} \ \vee \ \reln{e}{hb}{k}
    \end{align}
    
    \critic{red}{Do we need to prove that these are the only relations lost? Proof part 1 implicitly shows this.}

   \paragraph{3. Presence of Cycles?}
        
Because no new $\stck{_{hb}}$ relations are introduced, and because original candidate executions have $\stck{_{hb}}$ as a strict partial order, no cycles are introduced after elimination. 

\critic{blue}{Perhaps write this argument a bit better.}

   \paragraph{4. Do the lost relations result in New Observable Behaviors?}

        To answer this, we need to see whether the relations removed had an impact on possible $\stck{_{rf}}$ relations other than those with $e$. 
        We divide our argument into two parts, viz. the two types of relations removed:
        \begin{tasks}(2)
            \task $\reln{k}{hb}{R_{uo}}$. 
            \task $\reln{R_{uo}}{hb}{k}$.
        \end{tasks}

        Figure~\ref{elim_read:case1} shows a breakdown of sub-cases for case (a), varying based
        on the nature of event $k$.
        \begin{figure}[H]
            \centering
            \includegraphics[scale=0.5]{5.Elimination/1.ValidEliminationCandidate/ReadElimProof/ProofParts/Part4_Case1.pdf}
            \caption{The impact of lost relation $\reln{k}{hb}{R_{uo}}$ on observable behaviors.}
            \label{elim_read:case1}
        \end{figure}

        Observations:
        \begin{itemize}
            \item (i) is not a pattern forbidden by the consistency rules.
            \item (ii)(a) is a pattern of Axiom \ref{CoRe}, however, only restricting $\stck{_{rf}}$ relation with $R$ and $W'$(which here is our Unordered Read)
            \item (ii)(b) is a pattern of Axiom \ref{SeqCsAt}, however, once again, only restricting $\stck{_{rf}}$ relation with $R$ and $W'$. 
        \end{itemize}

        Figure~\ref{elim_read:case2} shows a breakdown of sub-cases for case (b), varying based
        on the nature of event $k$.
        \begin{figure}[H]
            \centering
            \includegraphics[scale=0.5]{5.Elimination/1.ValidEliminationCandidate/ReadElimProof/ProofParts/Part4_Case2.pdf}
            \caption{The impact of lost relation $\reln{R_{uo}}{hb}{k}$ on observable behaviors.}
            \label{elim_read:case2}
        \end{figure}

        Observations:
        \begin{itemize}
            \item (i) is not a pattern in any Consistency rules
            \item (ii) is a pattern of Axiom \ref{CoRe}, however, only restricting $\stck{_{rf}}$ relation with $R$ and $W$
        \end{itemize}

        From the above observations, we can infer that the relations removed only have restriction on reads-from relations on the event $e$ we eliminate. 
        Thus, we can conclude that no new observable behaviors are introduced due to the removed $\stck{_{hb}}$ relations. 

\end{proof}

    \subsection{Reordering Non-Consecutive Events}
        \begin{corollary}
    \label{CorolWriteElim}
    Consider a Candidate C of a program and its Candidate Executions which are valid. Consider two events $e$ and $d$ both having equal ranges such that:
    \begin{align*}
        \event{e}{W} \ \wedge \ \event{d}{W} \ \wedge \ \et{e}{uo} \ \wedge \ \reln{e}{ao}{d} \ \wedge \ \neg\cons{e}{d}
    \end{align*} 
    Consider another Candidate C' without the event $e$. If
    \begin{align*}
        \forall k \ \text{s.t.} \ \reln{e}{ao}{k} \wedge \reln{k}{ao}{d} \ , \
        Reord(e, k)
    \end{align*}
    Then, the set of Observable behaviors possible in C' is a subset of C.
\end{corollary}

\begin{proof}
    We prove by induction on the number of events $k$ between $e$ and $d$. We verify that if a $j$ exists that is valid, the Observable behaviors of $C'$ is a subset of $C$.

    \paragraph{Base Case : n = 1}

        We have the case when:
        \begin{align*}
            \reln{e}{ao}{k_1} \ \wedge \ reln{k_1}{ao}{d}
        \end{align*}

        By Theorem of Reordering and Def of consecutive events and agent order, we can reorder $e$ and $k_1$, thus giving us a Candidate $C''$ with :
        \begin{align*}
            \reln{k_1}{ao}{e} \ \wedge \ \reln{e}{ao}{d}
        \end{align*}  
        whose observable behaviors are a subset of $C$.

        By Def of Consecutive instructions and Theorem of Elmination, we can eliminate $e$, thus giving us candidate $C'$  with  
        \begin{align*}
            \reln{k_1}{ao}{d}
        \end{align*} 
        whose observable behaviors are a subset of $C''$.

        By transitive property of subsets, we can conclude that the observable behaviors of $C'$ is a subset of $C$. 
    \paragraph{Inductive Case (n)}

        Let us assume that if the number of events in between are $n$, then the corollary holds. Let us consider the Candidate to be $C_n$ and corresponding candidate after elimination as $C'_n$. The observable behavior of $C'_n$ is a subset of that of $C_n$.

        If we can show the above holds true for $n+1$ events, we are done.
        
        To show this, suppose we have $C_{n+1}$ as the candidate and $C'$ as the one after elimination of $e$. 
        
        Because $\stck{_{ao}}$ is a total order, there is a total order among all $n+1$ events $k$ agent ordered between $e$ and $d$ such that we can label them $k_1, k_2 , ... , k_{n+1}$ with the following properties
        \begin{align*}
            \reln{e}{ao}{\reln{k_1}{ao}{\reln{...}{ao}{\reln{k_{n+1}}{ao}{d}}}} 
            \ \wedge \ 
            cons(e,k_1) \wedge cons(k_1, k_2) \wedge ... \wedge cons(k_{n+1}, d) 
        \end{align*}
        By Theorem of Reordering and Def of consecutive events and agent order, we can reorder $e$ and $k_1$, thus giving a corresponding candidate $C_n$ having observable behaviors as a subset of $C_{n+1}$. 

        By our inductive assumption, we have that the observable behaviors of $C'$ is a subset of $C_n$. By transitive property of subsets, we can then conclude that the observable behaviors of $C'$ are a subset of that of $C_{n+1}$.

\end{proof}

\critic{blue}{The above proof is clear, but it seems to me that I need to label all definitions and lemmas and theorems and corollary so that I can refer them here.}

    For cases where reordering is not safe to do, we also show counter examples of programs where new observable behaviors are introduced.
    This additionally helps gain intuition about the proof given. 
    Note that we do not show examples for cases where $\reln{d}{hb}{e}$ itself is sufficient to show a new observable behavior, as this is a trivial exercise that can be done just using sequential programs.
    We show counterexamples where $\stck{_{hb}}$ relations lost (those across agents specifically), could introduce new observable behaviors. 
    These examples are in Appendix A.1.
