\section{Some Useful Definitions}
Before we go about proving when reordering is valid, we define certain helper definitions for it\footnotemark.

\footnotetext{The following definitions and lemmas are not particular to instruction reordering, so I think we can make it a point to put this in a section that introduces our work on optimizations.}

%Something we need to define for sake of proofs
\begin{definition}{Consecutive pair of events (\emph{cons})}
    \label{Cons}
    We define \emph{cons} as a function, which takes two events as input, and gives us a boolean indicating if they are consecutive pairs. Two events $e$ and $d$ are consecutive if they have a direct $\stck{_\textit{ao}}$ relation among them;  i.e those relations that are not derived through transitive property of $\stck{_\textit{hb}}$. 
    \begin{align*}
        (
        e \stck{_\textit{ao}} d  \ \wedge \ 
        \nexists k \ \textit{s.t.} \ 
        e \stck{_\textit{ao}} k  \ \wedge \
        k \stck{_\textit{ao}} d 
        )
        \ \vee \
        (
            d \stck{_\textit{ao}} e  \ \wedge \ 
            \nexists k \ \textit{s.t.} \ 
            d \stck{_\textit{ao}} k  \ \wedge \
            k \stck{_\textit{ao}} e  
        ).
    \end{align*}
\end{definition}

\begin{definition}{Direct happens-before relation (dir)}
    \label{Dir}
    We define \emph{dir} to take an ordered pair of events $(e,d)$ such that $\reln{e}{hb}{d}$ and gives a boolean value to indicate whether this relation is \textit{direct}, which can be formally stated as follows:
    \begin{align*}
        \nexists k \ \text{s.t.} \ \reln{e}{hb}{k} \wedge \reln{k}{hb}{d}.
    \end{align*}
    
    We can infer some useful things using \emph{dir} based on some information on events $e$ and $d$\footnotemark. 
    \begin{enumerate}
        \item If $\et{e}{uo}$, then $dir(e,d) \ \Rightarrow \ cons(e,d).$ 
        \item If $\et{d}{uo}$, then $dir(e,d) \ \Rightarrow \ cons(e,d).$
        \item If $\et{e}{sc}\ \wedge\ e\!\in\!R$, then $dir(e,d) \ \Rightarrow \ cons(e,d).$
        \item If $\et{e}{sc}\ \wedge\ e\!\in\!W$, then $dir(e,d) \ \Rightarrow \ cons(e,d)\ \vee\ \reln{e}{sw}{d}.$
        \item If $\et{d}{sc}\ \wedge\ d\!\in\!W$, then $dir(e,d) \ \Rightarrow \ cons(e,d).$
        \item If $\et{d}{sc}\ \wedge\ e\!\in\!R$, then $dir(e,d) \ \Rightarrow \ cons(e,d)\ \vee\ \reln{e}{sw}{d}.$
    \end{enumerate}

    \footnotetext{They can be proved trivially using definitions of $\stck{_{hb}}, \stck{_{sw}} \text{and} \stck{_{ao}}$}.
\end{definition}


\begin{definition}{Reorderable Pair (Reord)}
    \label{Reord}
    
    We define a boolean function \emph{Reord} that takes two ordered pair of events $(e,d)$ such that $\reln{e}{ao}{d}$ and gives a boolean value indicating if they are a reorderable pair\footnotemark.   
    \begin{align*}
        Reord(e,d) = \\
        (
        &((\et{e}{uo} \ \wedge \ \et{d}{uo}) \ \wedge \ 
                (   
                    (\event{e}{R} \ \wedge \ \event{d}{R}) \ \vee \ 
                    (\Re(e) \cap_\Re \Re(d) = \phi) 
                )
        ) \\ &\vee \\
        &((\et{e}{sc} \ \wedge \ \et{d}{uo}) \ \wedge \ 
                (
                    (\event{e}{W} \ \wedge \ (\Re(e) \cap_\Re \Re(d) = \phi)) 
                )
        ) \\ &\vee \\
        &((\et{e}{uo} \ \wedge \ \et{d}{sc}) \ \wedge \ 
                (
                    (\event{d}{R} \ \wedge \ (\Re(e) \cap_\Re \Re(d) = \phi)) 
                )
        )
        ).
    \end{align*}

    \footnotetext{We later prove that this exact definition defines when a pair of two consecutive events are reorderable.}
\end{definition}

