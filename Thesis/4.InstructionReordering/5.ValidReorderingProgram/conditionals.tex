\subsection{Addressing programs with Conditionals}

        We first consider programs with conditionals. 
        The following two properties holds for any candidates of programs having conditional branching. 
        \begin{property}{Candidates of Programs with Conditionals}
            \label{CondB1}
            Let $B1$ be the sets of events based on a branch of a conditional in a program $P$. 
            Let $C$ be any Candidate of $P$ and consider $k$ to be a representative event outside the conditional branch. 
            Then $b1 \in B1$ if and only if:
            \begin{align*}
                \exists C \ \text{s.t.} b1 \notin C  
            \end{align*}
            There exists a candidate of the program such that events from the branch cannot be part of it\footnotemark. 
        \end{property}

        \footnotetext{While the property for 1 branch may not always hold (it can be the case that the branch is always taken in any execution) we are defining it for any program. So we assume that every conditional can either be true or false in a program, not just one of them.}

        The above property is general for conditionals, whether it is an ``if-then" (1-branch) clause or ``if-then-else" (2-branch) clause. 
        The latter however, has another property which we define below:
        \begin{property}{Candidates of Programs with Conditionals (2-branch)}
            \label{CondB2}
            Let $B1,B2$ be two sets of events based on each branch of a conditional in a program $P$. 
            Let $C$ be any Candidate of $P$. 
            Then $b1 \in B1 \ \wedge \ b2 \in B2$ if and only if:
            \begin{align*}
                \nexists C \ \text{s.t.} \ b1 \in C \ \wedge \ b2 \in C \\ 
            \end{align*}
            There cannot exist any candidate of the program such that events from both sets can be part of it\footnotemark. 
        \end{property}

        \footnotetext{Note that here we consider every statement in the program unique. So a program like ``if $(c)$ then {$x=1;$} else {$x=1;$}", both the writes to $x$ are unique.}

        Figure~\ref{reord:conditionals} summarizes the two forms of conditionals we can have in any program. 
        \begin{figure}[H]
            \centering 
            \includegraphics[scale=0.7]{4.InstructionReordering/5.ValidReorderingProgram/Conditionals2Form.pdf}
            \caption{Two forms of conditionals.}
            \label{reord:conditionals}
        \end{figure}


        We use the above two properties to state the following lemma 
        \begin{lemma}
            \label{CondBranchLemma}
            Reordering an statement $e$ inside a conditional to outside a conditional violates Property \ref{CondB1} and Property \ref{CondB2}
        \end{lemma}

        \begin{proof}
            The proof is trivial. 
            By removing a statement outside of a conditional branch, we can get a candidate of a program that would violate both properties. 
        \end{proof}

        The above proof also lets us infer that on reordering an event outside a conditional, there are Candidates that exist with a new event belonging to it. 
        We use this insight to state the following corollary for reordering under conditionals. 
        \begin{corollary}
            \label{ReordCond}
            Consider a program $P$ with conditional branches and its candidates $C_1, C_2, ... , C_n$ in which events $e$ and $d$ present in all of them with $\reln{e}{ao}{d}$. 
            Consider the set of corresponding candidates $C'_1, C'_2, ... , C'_n$ after reordering $e$ and $d$ and its corresponding program $P'$. 
            If the following two conditions hold:
            \begin{gather*}
                Reord(e,d) \ \wedge \ 
                ( \forall C_{i \in [1,n]}, \forall k \in C_i \ \text{s.t.} \ \reln{e}{ao}{k} \wedge \reln{k}{ao}{d}, \    
                Reord(e,k) \wedge Reord(k,d) ). \\
                \nexists C \ s.t. \ 
                    (
                        (e \in C \ \wedge \ d \notin C) \ \vee \ 
                        (e \notin C \ \wedge \ d \in C)
                    ). 
            \end{gather*}
            then the set of observable behaviors of $P'$ is a subset of that of $P$. 
        \end{corollary}

        \begin{proof}

            We prove the second condition first. 
            Assume the second condition does not hold. 
            Then we would have
            \begin{align*}
                \exists C \in P \ s.t. \ 
                (
                    (e \in C \ \wedge \ d \notin C) \ \vee \ 
                    (e \notin C \ \wedge \ d \in C)
                ).
            \end{align*}
            
            By Property \ref{CondB1}, $e$ or $d$ must belong to a conditional branch. 
            If $e$ and $d$ are in different branches of same conditional, then by Property \ref{CondB2} there wouldn't exist any candidate $C$ in $P$ where we could reorder $e$ and $d$. 
            If $e$ and $d$ are of the same conditional branch, and neither one of them belong in any conditional branch nested within, then our above assumption does not hold (simple sequential property of conditional branches).
            
            For the other cases, without loss of generality, let us suppose the first condition holds, i.e. 
            \begin{align*}
                \exists C \in P \ s.t. \ 
                (e \in C \ \wedge \ d \notin C).
            \end{align*}

            The cases for the above can be summarized in the Figure~\ref{reord:cond_branch_cases}: 
            \begin{figure}[H]
                \label{CondCases}
                \centering 
                \includegraphics[scale=0.6]{4.InstructionReordering/5.ValidReorderingProgram/ConditionalCases.pdf}
                \caption{Four base cases where $e$ or $d$ are part of some conditional branch.}
                \label{reord:cond_branch_cases}
            \end{figure}

            For cases (i) and (ii), by Lemma \ref{CondBranchLemma}, a new Candidate with event $e$ or $d$ exists without their respective conditional branches being taken.
            For cases (iii) and (iv), by Lemma \ref{CondBranchLemma}, a new Candidate with event $d$ exists without its respective conditional branch being taken.
            Irrespective of $e$ or $d$ being a read or a write, there could be a new $\stck{_{rf}}$ relation be formed with some event $k$ in the Candidate. Thus, we have a new observable behavior\footnotemark. 

            Hence, by contradiction, the second condition must hold.
            
            \footnotetext{Note that this argument is purely in terms of the execution graphs. The new event can possibly have a new reads-from relation established with some event in the graph itself. Since this new node did not exist in the graph before, and since every node in the graph is considered unique, we can infer that a new observable behavior is introduced. Analyzing which such execution graphs are equivalent, would imply drawing equivalence between two different reads-from relations. This could be done as a whole by addressing redundancy introduction optimization. This is not within the scope of the thesis.}

            The first condition holds trivially as it corresponds to Corollary \ref{CorollReord} for each candidate $C_{i\in[1,n]}$. 
            By property of union of sets, we can infer that the set of Observable Behaviors of $P'$ is a subset of that of $P$.

        \end{proof}

  
%------------------------------------------------------------------------------------------------------------------------------------------
    